%%%%%%%%%%%%%%%%%%%%%%%%%%%%%%%%%%%%%%%%%%%%%%%%%%%%%%%%%%%%%%%%%
%                                                               %
%                                                               %
%                                                               %
\chapter{The \aemilia\ Compiler}
%                                                               %
%                                                               %
%                                                               %
%%%%%%%%%%%%%%%%%%%%%%%%%%%%%%%%%%%%%%%%%%%%%%%%%%%%%%%%%%%%%%%%%


%%%%%%%%%%%%%%%%%%%%%%%%%%%%%%%%%%%%%%%%%%%%%%%%%%%%%%%%%%%%%%%%%
%                                                               %
%                                                               %
\section{Introduction}
%                                                               %
%                                                               %
%%%%%%%%%%%%%%%%%%%%%%%%%%%%%%%%%%%%%%%%%%%%%%%%%%%%%%%%%%%%%%%%%

TwoTowers~5.1 accepts only system models that are written in the architectural description language
\aemilia~\cite{BDC,BBS} and are stored in {\tt .aem} files.

An \aemilia\ description represents an architectural type~\cite{BCD,AB}. This is an intermediate abstraction
between a single system and an architectural style~\cite{SG}. It consists of a family of systems sharing
certain constraints on the observable behavior of the system components as well as on the system topology.
As shown in Table~\ref{text}, the description of an architectural type in \aemilia\ starts with the name and
the formal parameters of the architectural type and is composed of three sections.

	\begin{table}[thb]

\[\begin{array}{|ll|}
\hline
{\tt ARCHI\_TYPE} & \triangleleft \textit{name and formal parameters} \triangleright \\[0.2cm]
\hspace*{0.4cm} {\tt ARCHI\_ELEM\_TYPES} & \\
\hspace*{0.8cm} {\tt ELEM\_TYPE} & \triangleleft \textit{definition of the first architectural element type}
\triangleright \\
\hspace*{1.0cm} \vdots & \hspace{0.2cm} \vdots \\
\hspace*{0.8cm} {\tt ELEM\_TYPE} & \triangleleft \textit{definition of the last architectural element type}
\triangleright \\[0.2cm]
\hspace*{0.4cm} {\tt ARCHI\_TOPOLOGY} & \\
\hspace*{0.8cm} {\tt ARCHI\_ELEM\_INSTANCES} & \triangleleft \textit{declaration of the architectural
element instances} \triangleright \\
\hspace*{0.8cm} {\tt ARCHI\_INTERACTIONS} & \triangleleft \textit{declaration of the architectural
interactions} \triangleright \\
\hspace*{0.8cm} {\tt ARCHI\_ATTACHMENTS} & \triangleleft \textit{declaration of the architectural
attachments} \triangleright \\[0.2cm]
\hspace*{0.4cm} [{\tt BEHAV\_VARIATIONS} & \\
\hspace*{0.8cm} [{\tt BEHAV\_HIDINGS} & \triangleleft \textit{declaration of the behavioral hidings}
\triangleright] \\
\hspace*{0.8cm} [{\tt BEHAV\_RESTRICTIONS} & \triangleleft \textit{declaration of the behavioral
restrictions} \triangleright] \\
\hspace*{0.8cm} [{\tt BEHAV\_RENAMINGS} & \triangleleft \textit{declaration of the behavioral renamings}
\triangleright]] \\[0.2cm]
{\tt END} & \\
\hline
\end{array}\]

\caption{Structure of an \aemilia\ description}\label{text}

	\end{table}

The first section defines the types of components that characterize the system family. In order to include
both the computational components and the connectors among them, these types are called architectural
element types (AETs). The definition of an AET starts with its name and formal parameters and consists of
the specification of its behavior and its interactions. The behavior has to be provided in the form of a
list of sequential defining equations written in a verbose variant of the stochastic process algebra
\empagr~\cite{Ber1,BB2,BB1}. The interactions are those \empagr\ action types occurring in the behavior that
act as interfaces for the AET. Each of them has to be equipped with two qualifiers, which establish whether
it is an input or output interaction and the multiplicity of the communications in which it can be involved,
respectively. All the other action types occurring in the behavior are assumed to represent internal
activities.

The second section defines the architectural topology. This is specified in three steps. First we have the
declaration of the instances of the AETs (called AEIs) with their actual parameters, which represent the
real system components and connectors. Then we have the declaration of the architectural (as opposed to
local) interactions, which are those interactions of the AEIs that act as interfaces for the whole system
family. Finally we have the declaration of the directed architectural attachments among the local
interactions of the AEIs, which make the AEIs communicate with each other.

The third section, which is optional, defines some variations of the observable behavior of the system
family. This is accomplished by declaring some action types occurring in the behavior of certain AEIs to be
unobservable, prevented from occurring, or renamed into other action types.



%%%%%%%%%%%%%%%%%%%%%%%%%%%%%%%%%%%%%%%%%%%%%%%%%%%%%%%%%%%%%%%%%
%                                                               %
%                                                               %
\section{Keywords and Comments}\label{keywords}
%                                                               %
%                                                               %
%%%%%%%%%%%%%%%%%%%%%%%%%%%%%%%%%%%%%%%%%%%%%%%%%%%%%%%%%%%%%%%%%

Here is the complete list of the keywords of \aemilia\ that can occur in {\tt .aem} files:

	\begin{verbatim}
  ARCHI_TYPE           BEHAV_HIDINGS      const     max         beta         first
  ARCHI_ELEM_TYPES     HIDE               local     abs         normal       tail
  ELEM_TYPE            INTERNALS          stop      ceil        pareto       concat
  BEHAVIOR             INTERACTIONS       invisible floor       b_pareto     insert
  INPUT_INTERACTIONS   ALL                exp       power       d_uniform    remove
  OUTPUT_INTERACTIONS  BEHAV_RESTRICTIONS inf       epower      bernoulli    length
  UNI                  RESTRICT           choice    loge        binomial     array
  AND                  OBS_INTERNALS      cond      log10       poisson      array_cons
  OR                   OBS_INTERACTIONS   void      sqrt        neg_binomial read
  ARCHI_TOPOLOGY       ALL_OBSERVABLES    prio      sin         geometric    write
  ARCHI_ELEM_INSTANCES BEHAV_RENAMINGS    rate      cos         pascal       record
  ARCHI_INTERACTIONS   RENAME             weight    c_uniform   boolean      record_cons
  ARCHI_ATTACHMENTS    AS                 integer   erlang      true         get
  FROM                 FOR_ALL            real      gamma       false        put
  TO                   IN                 mod       exponential list
  BEHAV_VARIATIONS     END                min       weibull     list_cons
	\end{verbatim}

\noindent where the upper-case keywords in the two leftmost columns refer in general to the three sections
of an \aemilia\ specification, while the lower-case keywords of the other four columns are used within
specific parts of an \aemilia\ specification like the architectural type formal parameters, the AET formal
parameters, the AET behavior, and the AEI actual parameters.

In addition to the keywords above, there are the following keywords that belong to the companion languages
and can occur in {\tt .evr}, {\tt .ltl}, {\tt .sec}, {\tt .sar}, {\tt .rew}, and {\tt .sim} files:

	\begin{verbatim}
      PROPERTY                   DEADLOCK_FREE                   OBS_NRESTR_INTERNALS
      IS                         FOR_ALL_PATHS                   OBS_NRESTR_INTERACTIONS
      TRUE                       FOR_ALL_PATHS_ALL_STATES_SAT    ALL_OBS_NRESTR
      FALSE                      FOR_ALL_PATHS_SOME_STATE_SAT    MEASURE
      NOT                        EXISTS_PATH                     ENABLED
      EXISTS_TRANS               EXISTS_PATH_ALL_STATES_SAT      STATE_REWARD
      EXISTS_WEAK_TRANS          EXISTS_PATH_SOME_STATE_SAT      TRANS_REWARD
      FOR_ALL_TRANS              STRONG_UNTIL                    RUN_LENGTH_ON_EXEC
      FOR_ALL_WEAK_TRANS         WEAK_UNTIL                      RUN_LENGTH
      EXISTS_TRANS_SET           NEXT_STATE_SAT                  RUN_NUMBER
      EXISTS_WEAK_TRANS_SET      ALL_FUTURE_STATES_SAT           MEAN
      FOR_ALL_TRANS_SETS         SOME_FUTURE_STATE_SAT           VARIANCE
      FOR_ALL_WEAK_TRANS_SETS    UNTIL                           DISTRIBUTION
      LABEL                      RELEASES                        REWARD
      MIN_AGGR_REA_PROB          PREV_STATE_SAT                  EXECUTED
      MIN_AGGR_EXP_RATE          ALL_PAST_STATES_SAT             CUMULATIVE
      MIN_AGGR_GEN_PROB          SOME_PAST_STATE_SAT             NON_CUMULATIVE
      REACHED_STATE_SAT          SINCE                           DRAW
      REACHED_STATES_SAT         TRIGGERED                       trc
      MIN_FIXPOINT               HIGH_SECURITY                   FOR_ALL
      MAX_FIXPOINT               LOW_SECURITY                    IN
	\end{verbatim}

\noindent where the keywords in the two leftmost columns (except for the last two keywords) are used within
modal and temporal logic formulas, while the other keywords are used to express security levels as well as
performance measures and simulation experiments.

Comments can be inserted wherever in a {\tt .aem}, {\tt .ltl}, {\tt .sec}, {\tt .rew} or {\tt .sim} file. A
comment starts with symbol {\tt \%} and terminates at the end of the line.



%%%%%%%%%%%%%%%%%%%%%%%%%%%%%%%%%%%%%%%%%%%%%%%%%%%%%%%%%%%%%%%%%
%                                                               %
%                                                               %
\section{Identifiers}
%                                                               %
%                                                               %
%%%%%%%%%%%%%%%%%%%%%%%%%%%%%%%%%%%%%%%%%%%%%%%%%%%%%%%%%%%%%%%%%

The user-defined identifiers denote architectural type names, AET names, AEI names, behavior names, action
type names, formal parameter names, local variable names, record field names, property names, and measure
names. Every user-defined identifier occurring in a {\tt .aem}, {\tt .ltl}, {\tt .sec}, {\tt .rew} or
{\tt .sim} file must be a sequence of upper- and lower-case letters, decimal digits, and underscores, which
starts with a letter and is different from any of the keywords listed in Sect.~\ref{keywords}. Every
user-defined identifier occurring in a {\tt .ltl}, {\tt .sec}, {\tt .rew} or {\tt .sim} file must have
previously occurred in a {\tt .aem} file, unless it denotes a property or a measure.

Except for architectural type names, AET names, and AEI names, every user-defined identifier is internally
represented (and consequently written to output files) through the dot notation by prefixing it with the
name of the context in which it is defined/used, hence the user is free to give the same name to several
entities in different contexts. The internal representation of the type of an action stemming from the
synchronization of several actions is given by the concatenation of the internal representation of the types
of the synchronizing actions, using symbol {\tt \#} as a separator.



%%%%%%%%%%%%%%%%%%%%%%%%%%%%%%%%%%%%%%%%%%%%%%%%%%%%%%%%%%%%%%%%%
%                                                               %
%                                                               %
\section{Data Types, Operators, and Expressions}
%                                                               %
%                                                               %
%%%%%%%%%%%%%%%%%%%%%%%%%%%%%%%%%%%%%%%%%%%%%%%%%%%%%%%%%%%%%%%%%

In this section we provide the syntax for typed identifier declaration and expressions, we define the value
domains for the data types available in \aemilia, and we introduce the related operators by specifying their
precedence and associativity whenever necessary. In order to describe the syntax, we adopt the BNF~notation,
with terminal symbols enclosed within double quotes, non-terminal symbols enclosed within angular
parentheses, and optional parts enclosed within square brackets.


%%%%%%%%%%%%%%%%%%%%%%%%%%%%%%%%%%%%%%%%%%%%%%%%%%%%%%%%%%%%%%%%%
%                                                               %
\subsection{Typed Identifier Declarations and Expressions}\label{id_expr}
%                                                               %
%%%%%%%%%%%%%%%%%%%%%%%%%%%%%%%%%%%%%%%%%%%%%%%%%%%%%%%%%%%%%%%%%

A typed identifier occurring in a {\tt .aem} file represents a constant formal parameter of the
architectural type or one of its AETs, a variable formal parameter or a local variable of a behavior, or an
action priority, rate, or weight. A typed identifier can be declared within the header of an architectural
type, AET, or behavior in the following C-like way:

	\begin{verbatim}
    <data_type> <identifier>
	\end{verbatim}

\noindent with {\tt <data\_type>} being defined by:

	\begin{verbatim}
       <data_type> ::= <normal_type>
                    |  <special_type>
     <normal_type> ::= "integer"
                    |  "integer" "(" <expr> ".." <expr> ")"
                    |  "real"
                    |  "boolean"
                    |  "list" "(" <normal_type> ")"
                    |  "array" "(" <expr> "," <normal_type> ")"
                    |  "record" "(" <field_decl_sequence> ")"
    <special_type> ::= "prio"
                    |  "rate"
                    |  "weight"
	\end{verbatim}

An expression -- denoted by {\tt <expr>} in the following -- is composed of atomic elements, given by typed
identifiers, numeric constants, and truth values, possibly combined through the available operators. The
type of an expression is determined by the type of its atomic elements and the codomain of the operators
occurring in it, while the order in which the infix operators have to be applied to evaluate the expression
is given by their precedence and associativity. The order can be altered using parentheses {\tt ( )}.


%%%%%%%%%%%%%%%%%%%%%%%%%%%%%%%%%%%%%%%%%%%%%%%%%%%%%%%%%%%%%%%%%
%                                                               %
\subsection{Integers, Bounded Integers, and Reals}
%                                                               %
%%%%%%%%%%%%%%%%%%%%%%%%%%%%%%%%%%%%%%%%%%%%%%%%%%%%%%%%%%%%%%%%%

The type {\tt integer} denotes the set of integer numbers that can be represented in the used computer
according to the ANSI~C standard. A special case is given by the bounded integer set defined as follows:

	\begin{verbatim}
    "integer" "(" <expr> ".." <expr> ")"
	\end{verbatim}

\noindent which denotes the set of integers between the value of the first expression and the value of the
second expression. Both expressions must be integer valued, free of undeclared identifiers, and free of
invocations to pseudo-random number generators. Moreover, the value of the first expression cannot be
greater than the value of the second expression.

The type {\tt real} denotes the set of real numbers in fixed-point notation that can be represented in the
used computer according to the ANSI~C standard.


%%%%%%%%%%%%%%%%%%%%%%%%%%%%%%%%%%%%%%%%%%%%%%%%%%%%%%%%%%%%%%%%%
\subsubsection{Arithmetical Operators}
%%%%%%%%%%%%%%%%%%%%%%%%%%%%%%%%%%%%%%%%%%%%%%%%%%%%%%%%%%%%%%%%%

The following four binary arithmetical operators are available in \aemilia:

	\begin{verbatim}
    <expr> ::= <expr> "+" <expr>
            |  <expr> "-" <expr>
            |  <expr> "*" <expr>
            |  <expr> "/" <expr>
	\end{verbatim}

\noindent with the division requiring the second operand to be different from zero and always returning a
real number. All the operators above are left associative, with the multiplicative ones taking precedence
over the additive ones. The unary {\tt -} operator is not explicitly available as its effect can be achieved
through a multiplication by {\tt -1}.


%%%%%%%%%%%%%%%%%%%%%%%%%%%%%%%%%%%%%%%%%%%%%%%%%%%%%%%%%%%%%%%%%
\subsubsection{Relational Operators}
%%%%%%%%%%%%%%%%%%%%%%%%%%%%%%%%%%%%%%%%%%%%%%%%%%%%%%%%%%%%%%%%%

The following six binary relational operators are available in \aemilia:

	\begin{verbatim}
    <expr> ::= <expr> "=" <expr>
            |  <expr> "!=" <expr>
            |  <expr> "<" <expr>
            |  <expr> "<=" <expr>
            |  <expr> ">" <expr>
            |  <expr> ">=" <expr>
	\end{verbatim}

\noindent All of them are non-associative. The arithmetical operators take precedence over them.


%%%%%%%%%%%%%%%%%%%%%%%%%%%%%%%%%%%%%%%%%%%%%%%%%%%%%%%%%%%%%%%%%
\subsubsection{Mathematical Functions}
%%%%%%%%%%%%%%%%%%%%%%%%%%%%%%%%%%%%%%%%%%%%%%%%%%%%%%%%%%%%%%%%%

The following thirteen mathematical functions are available in \aemilia:

	\begin{verbatim}
    <expr> ::= "mod" "(" <expr> "," <expr> ")"
            |  "abs" "(" <expr> ")"
            |  "ceil" "(" <expr> ")"
            |  "floor" "(" <expr> ")"
            |  "min" "(" <expr> "," <expr> ")"
            |  "max" "(" <expr> "," <expr> ")"
            |  "power" "(" <expr> "," <expr> ")"
            |  "epower" "(" <expr> ")"
            |  "loge" "(" <expr> ")"
            |  "log10" "(" <expr> ")"
            |  "sqrt" "(" <expr> ")"
            |  "sin" "(" <expr> ")"
            |  "cos" "(" <expr> ")"
	\end{verbatim}

\noindent where:

	\begin{itemize}

\item {\tt mod} computes the modulus of its first argument with respect to its second argument. Both
arguments must be integer, with the second one greater than zero.

\item {\tt abs} computes the absolute value of its argument.

\item {\tt ceil} (resp.\ {\tt floor}) computes the smallest (resp.\ greatest) integer greater (resp.\
smaller) than or equal to its argument.

\item {\tt min} (resp.\ {\tt max}) computes the minimum (resp.\ maximum) of its two arguments.

\item {\tt power} computes the power of its first argument raised to its second argument. It cannot be
applied to a pair of arguments such that the first one is zero and the second one is not positive, or the
first one is negative and the second one is real.

\item {\tt epower} computes the power of $e$ raised to its argument.

\item {\tt loge} (resp.\ {\tt log10}) computes the natural (resp.\ base-10) logarithm of its argument, which
must be greater than zero.

\item {\tt sqrt} computes the square root of its argument, which cannot be negative.

\item {\tt sin} (resp.\ {\tt cos}) computes the sine (resp.\ cosine) of its argument expressed in radians.

	\end{itemize}


%%%%%%%%%%%%%%%%%%%%%%%%%%%%%%%%%%%%%%%%%%%%%%%%%%%%%%%%%%%%%%%%%
\subsubsection{Pseudo-random Number Generators}
%%%%%%%%%%%%%%%%%%%%%%%%%%%%%%%%%%%%%%%%%%%%%%%%%%%%%%%%%%%%%%%%%

The following sixteen pseudo-random number generators~\cite{Jai} are available in \aemilia:

	\begin{verbatim}
    <expr> ::= "c_uniform" "(" <expr> "," <expr> ")"
            |  "erlang" "(" <expr> "," <expr> ")"
            |  "gamma" "(" <expr> "," <expr> ")"
            |  "exponential" "(" <expr> ")"
            |  "weibull" "(" <expr> "," <expr> ")"
            |  "beta" "(" <expr> "," <expr> ")"
            |  "normal" "(" <expr> "," <expr> ")"
            |  "pareto" "(" <expr> ")"
            |  "b_pareto" "(" <expr> "," <expr> "," <expr> ")"
            |  "d_uniform" "(" <expr> "," <expr> ")"
            |  "bernoulli" "(" <expr> "," <expr> "," <expr> ")"
            |  "binomial" "(" <expr> "," <expr> ")"
            |  "poisson" "(" <expr> ")"
            |  "neg_binomial" "(" <expr> "," <expr> ")"
            |  "geometric" "(" <expr> ")"
            |  "pascal" "(" <expr> "," <expr> ")"
	\end{verbatim}

\noindent where:

	\begin{itemize}

\item {\tt c\_uniform} generates a random number following a continuous uniform distribution between its two
arguments, with the second one greater than the first one:
\cws{10}{18}{\ms{pdf}_{\tt c\_uniform}(x) = {1 \over {\tt expr2} - {\tt expr1}} \hspace{0.6cm} {\tt for}
\hspace{0.2cm} {\tt expr1} \le x \le {\tt expr2}}

\item {\tt erlang} generates a random number following an Erlang distribution with rate parameter given by
its first argument, which must be greater than zero, and shape parameter given by its second argument, which
cannot be less than one:
\cws{10}{18}{\ms{pdf}_{\tt erlang}(x) = {x^{{\tt expr2} - 1} \cdot e^{- {\tt expr1} \cdot x} \over ({\tt
expr2} - 1)! \cdot (1 / {\tt expr1})^{\tt expr2}} \hspace{0.6cm} {\tt for} \hspace{0.2cm} x \ge 0}

\item {\tt gamma} generates a random number following a gamma distribution with rate parameter given by its
first argument, which must be greater than zero, and shape parameter given by its second argument, which
must be greater than zero:
\cws{10}{18}{\ms{pdf}_{\tt gamma}(x) = {({\tt expr1} \cdot x)^{{\tt expr2} - 1} \cdot e^{- {\tt expr1} \cdot
x} \over (1 / {\tt expr1}) \cdot \Gamma({\tt expr2})} \hspace{0.6cm} {\tt for} \hspace{0.2cm} x \ge 0,
\hspace{0.2cm} {\tt where} \hspace{0.2cm} \Gamma(y) = \int_{0}^{\infty} z^{y - 1} \cdot e^{-z} \ms{dz}}

\item {\tt exponential} generates a random number following an exponential distribution with rate parameter
given by its argument, which must be greater than zero:
\cws{12}{20}{\ms{pdf}_{\tt exponential}(x) = {\tt expr} \cdot e^{- {\tt expr} \cdot x} \hspace{0.6cm}
{\tt for} \hspace{0.2cm} x \ge 0}

\item {\tt weibull} generates a random number following a Weibull distribution with rate parameter given by
its first argument, which must be greater than zero, and shape parameter given by its second argument, which
must be greater than zero:
\cws{10}{18}{\ms{pdf}_{\tt weibull}(x) = {{\tt expr2} \cdot x^{{\tt expr2} - 1} \over (1 /
{\tt expr1})^{\tt expr2}} \cdot e^{- ({\tt expr1} \cdot x)^{\tt expr2}} \hspace{0.6cm} {\tt for}
\hspace{0.2cm} x \ge 0}

\item {\tt beta} generates a random number following a beta distribution with shape parameters given by its
two arguments, which must be greater than zero:
\cws{10}{18}{\ms{pdf}_{\tt beta}(x) = {x^{{\tt expr1} - 1} \cdot (1 - x)^{{\tt expr2} - 1} \over
\beta({\tt expr1}, {\tt expr2})} \hspace{0.6cm} {\tt for} \hspace{0.2cm} 0 \le x \le 1, \hspace{0.2cm}
{\tt where} \hspace{0.2cm} \beta(y, z) = {\Gamma(y) \cdot \Gamma(z) \over \Gamma(y + z)}}

\item {\tt normal} generates a random number following a normal distribution with mean given by its first
argument and standard deviation given by its second argument, which must be greater than zero:
\cws{10}{18}{\ms{pdf}_{\tt normal}(x) = {1 \over {\tt expr2} \cdot \sqrt{2 \cdot \pi}} \cdot e^{-{(x -
{\tt expr1})^{2} \over 2 \cdot {\tt expr2}^{2}}}}

\item {\tt pareto} generates a random number following a Pareto distribution with shape parameter given by
its argument, which must be greater than zero:
\cws{12}{20}{\ms{pdf}_{\tt pareto}(x) = {\tt expr} \cdot x^{- ({\tt expr} + 1)} \hspace{0.6cm} {\tt for}
\hspace{0.2cm} x \ge 1}

\item {\tt b\_pareto} generates a random number following a Pareto distribution with shape parameter given
by its first argument, which must be greater than zero, bounded between its other two arguments, with the
second argument not less than one but less than the third argument:
\cws{10}{18}{\ms{pdf}_{\tt b\_pareto}(x) = {{\tt expr1} \cdot {\tt expr2}^{\tt expr1} \over 1 - ({\tt expr2}
/ {\tt expr3})^{\tt expr1}} \cdot x^{- ({\tt expr1} + 1)} \hspace{0.6cm} {\tt for} \hspace{0.2cm}
{\tt expr2} \le x \le {\tt expr3}}

\item {\tt d\_uniform} generates a random number following a discrete uniform distribution between its two
arguments, which must be integer with the second one greater than the first one:
\cws{10}{18}{\ms{pmf}_{\tt d\_uniform}(x) = {1 \over {\tt expr2} - {\tt expr1} + 1} \hspace{0.6cm} {\tt for}
\hspace{0.2cm} {\tt expr1} \le x \le {\tt expr2}}

\item {\tt bernoulli} generates a random number following a Bernoulli distribution where the two possible
values are given by its first two arguments and the probability of choosing the first value is given by its
third argument, which must be in the open interval between zero and one:
\cws{12}{20}{\ms{pmf}_{\tt bernoulli}({\tt expr1}) = {\tt expr3} \hspace{0.2cm} {\tt and} \hspace{0.2cm}
\ms{pmf}_{\tt bernoulli}({\tt expr2}) = 1 - {\tt expr3}}

\item {\tt binomial} generates a random number following a binomial distribution with probability of success
given by its first argument, which must be in the open interval between zero and one, and number of trials
given by its second argument, which must be an integer not less than one:
\cws{10}{18}{\ms{pmf}_{\tt binomial}(x) = \left( \begin{array}{c} {\tt expr2} \\ x \end{array} \right) \cdot
{\tt expr1}^{x} \cdot (1 - {\tt expr1})^{{\tt expr2} - x} \hspace{0.6cm} {\tt for} \hspace{0.2cm} 0 \le x
\le {\tt expr2}}

\item {\tt poisson} generates a random number following a Poisson distribution with mean given by its
argument, which must be greater than zero:
\cws{10}{18}{\ms{pmf}_{\tt poisson}(x) = {{\tt expr}^{x} \over x!} \cdot e^{-{\tt expr}} \hspace{0.6cm}
{\tt for} \hspace{0.2cm} x \ge 0}

\item {\tt neg\_binomial} generates a random number following a negative binomial distribution with
probability of success given by its first argument, which must be in the open interval between zero and one,
and number of successes given by its second argument, which must be an integer not less than one:
\cws{10}{18}{\ms{pmf}_{\tt neg\_binomial}(x) = \left( \begin{array}{c} {\tt expr2} + x - 1 \\ {\tt expr2} -
1 \end{array} \right) \cdot {\tt expr1}^{\tt expr2} \cdot (1 - {\tt expr1})^{x} \hspace{0.6cm} {\tt for}
\hspace{0.2cm} x \ge 0}

\item {\tt geometric} generates a random number following a geometric distribution with probability of
success given by its argument, which must be in the open interval between zero and one:
\cws{12}{20}{\ms{pmf}_{\tt geometric}(x) = {\tt expr} \cdot (1 - {\tt expr})^{x - 1} \hspace{0.6cm}
{\tt for} \hspace{0.2cm} x \ge 1}

\item {\tt pascal} generates a random number following a Pascal distribution with probability of success
given by its first argument, which must be in the open interval between zero and one, and number of
successes given by its second argument, which must be an integer not less than one:
\cws{10}{18}{\ms{pmf}_{\tt pascal}(x) = \left( \begin{array}{c} x - 1 \\ {\tt expr2} - 1 \end{array} \right)
\cdot {\tt expr1}^{\tt expr2} \cdot (1 - {\tt expr1})^{x - {\tt expr2}} \hspace{0.6cm} {\tt for}
\hspace{0.2cm} x \ge {\tt expr2}}

	\end{itemize}


%%%%%%%%%%%%%%%%%%%%%%%%%%%%%%%%%%%%%%%%%%%%%%%%%%%%%%%%%%%%%%%%%
%                                                               %
\subsection{Booleans}
%                                                               %
%%%%%%%%%%%%%%%%%%%%%%%%%%%%%%%%%%%%%%%%%%%%%%%%%%%%%%%%%%%%%%%%%

The type {\tt boolean} denotes the set composed of the two truth values {\tt true} and {\tt false}. The
following three logical operators are available in \aemilia:

	\begin{verbatim}
    <expr> ::= <expr> "&&" <expr>
            |  <expr> "||" <expr>
            |  "!" <expr>
	\end{verbatim}

\noindent with the logical negation ({\tt !}) being right associative and taking precedence over the logical
conjunction ({\tt \&\&}) and the logical disjunction ({\tt ||}), which are left associative and subject to
short-circuitation. The relational operators and the arithmetical operators take precedence over the logical
ones.


%%%%%%%%%%%%%%%%%%%%%%%%%%%%%%%%%%%%%%%%%%%%%%%%%%%%%%%%%%%%%%%%%
%                                                               %
\subsection{Lists}
%                                                               %
%%%%%%%%%%%%%%%%%%%%%%%%%%%%%%%%%%%%%%%%%%%%%%%%%%%%%%%%%%%%%%%%%

The type {\tt list}, which denotes a possibly empty, variable-length sequence of elements of the same type,
is defined as follows:

	\begin{verbatim}
    "list" "(" <normal_type> ")"
	\end{verbatim}

\noindent where {\tt <normal\_type>} is the type of its elements.

The following seven list-related functions are available in \aemilia:

	\begin{verbatim}
    <expr> ::= "list_cons" "(" <pe_expr_sequence> ")"
            |  "first" "(" <expr> ")"
            |  "tail" "(" <expr> ")"
            |  "concat" "(" <expr> "," <expr> ")"
            |  "insert" "(" <expr> "," <expr> ")"
            |  "remove" "(" <expr> "," <expr> ")"
            |  "length" "(" <expr> ")"
	\end{verbatim}

\noindent where {\tt <pe\_expr\_sequence>} is a possibly empty sequence of comma-separated expressions and:

	\begin{itemize}

\item {\tt list\_cons} constructs a possibly empty list composed of the values of the expressions in its
argument, which must be of the same type.

\item {\tt first} returns the first element of its argument, which must be a non-empty list.

\item {\tt tail} returns what follows the first element of its argument, which must be a list.

\item {\tt concat} concatenates its arguments, which must be two lists whose elements are of the same type.

\item {\tt insert} inserts the value of its first argument into its second argument, which must be a list
whose elements are of the same type as the first argument. The position at which the insertion takes place
is established according to the lexicographical order of the elements.

\item {\tt remove} removes the value of the element whose position is given by its first argument, which
must be an integer not less than one, from its second argument, which must be a list with sufficiently many
elements.

\item {\tt length} computes the number of elements of its argument, which must be a list.

	\end{itemize}


%%%%%%%%%%%%%%%%%%%%%%%%%%%%%%%%%%%%%%%%%%%%%%%%%%%%%%%%%%%%%%%%%
%                                                               %
\subsection{Arrays}
%                                                               %
%%%%%%%%%%%%%%%%%%%%%%%%%%%%%%%%%%%%%%%%%%%%%%%%%%%%%%%%%%%%%%%%%

The type {\tt array}, which denotes a non-empty, fixed-length sequence of elements of the same type, is
defined as follows:

	\begin{verbatim}
    "array" "(" <expr> "," <normal_type> ")"
	\end{verbatim}

\noindent where {\tt <expr>} is its length and {\tt <normal\_type>} is the type of its elements. The array
length expression must be integer valued, free of undeclared identifiers, free of invocations to
pseudo-random number generators, and not less than one.

The following three array-related functions are available in \aemilia:

	\begin{verbatim}
    <expr> ::= "array_cons" "(" <expr_sequence> ")"
            |  "read" "(" <expr> "," <expr> ")"
            |  "write" "(" <expr> "," <expr> "," <expr> ")"
	\end{verbatim}

\noindent where {\tt <expr\_sequence>} is a non-empty sequence of comma-separated expressions and:

	\begin{itemize}

\item {\tt array\_cons} constructs an array composed of the values of the expressions in its argument, which
must be of the same type.

\item {\tt read} reads from its second argument, which must be an array with sufficiently many elements, the
value of the element indexed by its first argument, which must be an integer between zero and the length of
the second argument decremented by one.

\item {\tt write} writes to its third argument, which must be an array with sufficiently many elements of
the same type as the second argument, the value of its second argument in the position indexed by its first
argument, which must be an integer between zero and the length of the third argument decremented by one.

	\end{itemize}


%%%%%%%%%%%%%%%%%%%%%%%%%%%%%%%%%%%%%%%%%%%%%%%%%%%%%%%%%%%%%%%%%
%                                                               %
\subsection{Records}
%                                                               %
%%%%%%%%%%%%%%%%%%%%%%%%%%%%%%%%%%%%%%%%%%%%%%%%%%%%%%%%%%%%%%%%%

The type {\tt record}, which denotes a non-empty, fixed-length sequence of named elements of possibly
different types called fields, is defined as follows:

	\begin{verbatim}
    "record" "(" <field_decl_sequence> ")"
	\end{verbatim}

\noindent where {\tt <field\_decl\_sequence>} is a non-empty sequence of comma-separated field declarations,
each of the form defined in Sect.~\ref{id_expr}.

The following three record-related functions are available in \aemilia:

	\begin{verbatim}
    <expr> ::= "record_cons" "(" <expr_sequence> ")"
            |  "get" "(" <identifier> "," <expr> ")"
            |  "put" "(" <identifier> "," <expr> "," <expr> ")"
	\end{verbatim}

\noindent where {\tt <expr\_sequence>} is a non-empty sequence of comma-separated expressions and:

	\begin{itemize}

\item {\tt record\_cons} constructs a record composed of the values of the expressions in its argument.

\item {\tt get} gets from its second argument, which must be a record, the value of the field whose
identifier is given by its first argument, which must belong to the second argument.

\item {\tt put} puts into its third argument, which must be a record, the value of its second argument in
the field whose identifier is given by its first argument, which must belong to the third argument must and
be of the same type as the second argument.

	\end{itemize}


%%%%%%%%%%%%%%%%%%%%%%%%%%%%%%%%%%%%%%%%%%%%%%%%%%%%%%%%%%%%%%%%%
%                                                               %
\subsection{Priorities, Rates, and Weights}
%                                                               %
%%%%%%%%%%%%%%%%%%%%%%%%%%%%%%%%%%%%%%%%%%%%%%%%%%%%%%%%%%%%%%%%%

The type {\tt prio} denotes the set of immediate and passive action priorities, which coincides with the set
of positive integers.

The type {\tt rate} denotes the set of exponentially timed action rates, which coincides with the set of
positive reals.

The type {\tt weight} denotes the set of immediate and passive action weights, which coincides with the set
of positive reals.



%%%%%%%%%%%%%%%%%%%%%%%%%%%%%%%%%%%%%%%%%%%%%%%%%%%%%%%%%%%%%%%%%
%                                                               %
%                                                               %
\section{Architectural Type Header}
%                                                               %
%                                                               %
%%%%%%%%%%%%%%%%%%%%%%%%%%%%%%%%%%%%%%%%%%%%%%%%%%%%%%%%%%%%%%%%%

The architectural type header at the beginning of an \aemilia\ specification has the following syntax:

	\begin{verbatim}
    "ARCHI_TYPE" <identifier> "(" <init_const_formal_par_decl_sequence> ")"
	\end{verbatim}

\noindent where {\tt <identifier>} is the name of the architectural type and
{\tt <init\_const\_formal\_par\_decl\_sequence>} is either {\tt void} or a non-empty sequence of
comma-separated declarations of initialized constant formal parameters, each of the following form:

	\begin{verbatim}
    "const" <data_type> <identifier> ":=" <expr>
	\end{verbatim}

A constant formal parameter represents a formal parameter whose value, which stems in this case from the
evaluation of the assigned expression, cannot change. The assigned expression must be of the same type as
the identifier, free of undeclared identifiers, and free of invocations to pseudo-random number generators.
As a consequence, the only identifiers that can occur in the assigned expression are those for the preceding
constant formal parameters declared in the architectural type header.



%%%%%%%%%%%%%%%%%%%%%%%%%%%%%%%%%%%%%%%%%%%%%%%%%%%%%%%%%%%%%%%%%
%                                                               %
%                                                               %
\section{Architectural Element Types}
%                                                               %
%                                                               %
%%%%%%%%%%%%%%%%%%%%%%%%%%%%%%%%%%%%%%%%%%%%%%%%%%%%%%%%%%%%%%%%%

The first section of an \aemilia\ specification starts with the keyword {\tt ARCHI\_ELEM\_TYPES} and is
composed of a non-empty sequence of AET definitions, each of the following form:

	\begin{verbatim}
    <AET_header> <AET_behavior> <AET_interactions>
	\end{verbatim}


%%%%%%%%%%%%%%%%%%%%%%%%%%%%%%%%%%%%%%%%%%%%%%%%%%%%%%%%%%%%%%%%%
%                                                               %
\subsection{AET Header}
%                                                               %
%%%%%%%%%%%%%%%%%%%%%%%%%%%%%%%%%%%%%%%%%%%%%%%%%%%%%%%%%%%%%%%%%

Similarly to the architectural type header, the header of an AET has the following syntax:

	\begin{verbatim}
    "ELEM_TYPE" <identifier> "(" <const_formal_par_decl_sequence> ")"
	\end{verbatim}

\noindent where {\tt <identifier>} is the name of the AET and {\tt <const\_formal\_par\_decl\_sequence>} is
either {\tt void} or a non-empty sequence of comma-separated declarations of constant formal parameters,
each of the following form:

	\begin{verbatim}
    "const" <data_type> <identifier>
	\end{verbatim}

\noindent The value of each such formal constant parameter is defined upon declaration of the instances of
the AET in the architectural topology section.


%%%%%%%%%%%%%%%%%%%%%%%%%%%%%%%%%%%%%%%%%%%%%%%%%%%%%%%%%%%%%%%%%
%                                                               %
\subsection{AET Behavior: \empagr\ Operators and Actions}\label{aet_behavior}
%                                                               %
%%%%%%%%%%%%%%%%%%%%%%%%%%%%%%%%%%%%%%%%%%%%%%%%%%%%%%%%%%%%%%%%%

The behavior of an AET has the following syntax:

	\begin{verbatim}
    "BEHAVIOR" <behav_equation_sequence>
	\end{verbatim}

\noindent where {\tt <behav\_equation\_sequence>} is a non-empty sequence of semicolon-separated \empagr\
behavioral equations, each of the following form:

	\begin{verbatim}
    <behav_equation_header> "=" <process_term>
	\end{verbatim}

\noindent The first behavioral equation in the sequence represents the initial behavior for the AET. Each of
the other possible behavioral equations in the sequence must describe a behavior that can be directly or
indirectly invoked by the initial one.


%%%%%%%%%%%%%%%%%%%%%%%%%%%%%%%%%%%%%%%%%%%%%%%%%%%%%%%%%%%%%%%%%
\subsubsection{Behavioral Equation Header}
%%%%%%%%%%%%%%%%%%%%%%%%%%%%%%%%%%%%%%%%%%%%%%%%%%%%%%%%%%%%%%%%%

The header of the first behavioral equation has the following syntax:

	\begin{verbatim}
    <identifier> "(" <init_var_formal_par_decl_sequence> ";" <local_var_decl_sequence> ")"
	\end{verbatim}

\noindent whereas the header of any subsequent behavioral equation has the following syntax:

	\begin{verbatim}
    <identifier> "(" <var_formal_par_decl_sequence> ";" <local_var_decl_sequence> ")"
	\end{verbatim}

\noindent In both headers, {\tt <identifier>} is the name of the behavioral equation, while
{\tt <local\_var\_decl\_sequence>} is either {\tt void} or a non-empty sequence of comma-separated
declarations of local variables, each of the following form:

	\begin{verbatim}
    "local" <normal_type> <identifier>
	\end{verbatim}

A local variable is typically used to store one of the values received when synchronizing an input action of
an instance of the AET with an output action of another AEI.

In the header of the first behavioral equation, {\tt <init\_var\_formal\_par\_decl\_sequence>} is either
{\tt void} or a non-empty sequence of comma-separated declarations of initialized variable formal
parameters, each of the following form:

	\begin{verbatim}
    <normal_type> <identifier> ":=" <expr>
	\end{verbatim}

In the header of the subsequent behavioral equations, {\tt <var\_formal\_par\_decl\_sequence>} is either
{\tt void} or a non-empty sequence of comma-separated declarations of variable formal parameters, each of
the following form:

	\begin{verbatim}
    <normal_type> <identifier>
	\end{verbatim}

A variable formal parameter represents a formal parameter whose value can change and, in the case of the
first behavioral equation, is initialized by evaluating the assigned expression. The assigned expression
must be of the same type as the identifier and free of undeclared identifiers. The only identifiers that can
occur in the assigned expression are those for the constant formal parameters declared in the AET header. No
initializing expression is needed for the variable formal parameters of each subsequent behavioral equation,
as they will be assigned the values of the actual parameters contained in the invocations of the related
behavioral equation.


%%%%%%%%%%%%%%%%%%%%%%%%%%%%%%%%%%%%%%%%%%%%%%%%%%%%%%%%%%%%%%%%%
\subsubsection{Process Terms}
%%%%%%%%%%%%%%%%%%%%%%%%%%%%%%%%%%%%%%%%%%%%%%%%%%%%%%%%%%%%%%%%%

The syntax for the process term following the behavioral equation header is a verbose variant of the syntax
for the \empagr\ dynamic operators -- stop, action prefix, alternative composition, and behavioral equation
invocation:

	\begin{verbatim}
      <process_term> ::= "stop"
                      |  <action> "." <process_term_1>
                      |  "choice" "{" <process_term_2_sequence> "}"
    <process_term_1> ::= <process_term>
                      |  <identifier> "(" <actual_par_sequence> ")"
    <process_term_2> ::= ["cond" "(" <expr> ")" "->"] <process_term>
	\end{verbatim}

Constant {\tt stop} represents the process term that cannot execute any action. The action prefix operator
({\tt .}) represents a process term that can execute an action given by its first operand and then behaves
as the process term given by its second operand. The alternative composition operator ({\tt choice})
represents a process term that behaves as one of the elements of {\tt <process\_term\_2\_sequence>}, which
is a sequence of at least two comma-separated process terms, each possibly preceded by a boolean expression
establishing the condition under which it is available. The behavioral equation invocation represents a
process term that behaves as the behavioral equation whose name is given by {\tt <identifier>}, when passing
a possibly empty sequence of expressions represented by {\tt <actual\_par\_sequence>}. The actual parameters
must match by number, order and type with the variable formal parameters of the invoked behavioral equation.
Note that a behavioral equation invocation can occur only immediately after an action prefix operator.


%%%%%%%%%%%%%%%%%%%%%%%%%%%%%%%%%%%%%%%%%%%%%%%%%%%%%%%%%%%%%%%%%
\subsubsection{Actions}
%%%%%%%%%%%%%%%%%%%%%%%%%%%%%%%%%%%%%%%%%%%%%%%%%%%%%%%%%%%%%%%%%

The syntax for an action occurring in the process term of a behavioral equation is as follows:

	\begin{verbatim}
         <action> ::= "<" <action_type> "," <action_rate> ">"
    <action_type> ::= <identifier>
                   |  <identifier> "?" "(" <local_var_sequence> ")"
                   |  <identifier> "!" "(" <expr_sequence> ")"
    <action_rate> ::= "exp" "(" <expr> ")"
                   |  "inf" "(" <expr> "," <expr> ")"
                   |  "inf"
                   |  "_" "(" <expr> "," <expr> ")"
                   |  "_"
	\end{verbatim}

The action type is simply an identifier (unstructured action), an identifier followed by symbol {\tt ?}\ and
a non-empty sequence of local variables (input action), or an identifier followed by symbol {\tt !}\ and a
non-empty sequence of expressions (output action). Whenever a local variable occurs in an expression within
an output action, a behavioral equation invocation, or a boolean guard without previously occurring in an
input action, it evaluates to zero, false, empty list, null array, or null record depending on its type.

The rate of an exponentially timed action ({\tt exp}) is given by an expression, whose value must be a
positive real, that is interpreted as the rate of the exponentially distributed random variable describing
the action duration. The rate of an immediate action ({\tt inf}) is expressed through a priority, given by
an expression whose value must be an integer not less than one, and a weight, given by an expression whose
value must be a positive real. The rate of a passive action ({\tt \_}) is again expressed through two
expressions denoting a priority and a weight, respectively. If not specified, the values of the priority and
the weight of an immediate or passive action are assumed to be one.

There are three constraints to which the actions are subject. First, within the behavior of an AET, the
actions in which an action type identifier occur must all be unstructured, input with the same number,
order, and type of local variables, or output with the same number, order, and type of expressions. Second,
within the behavior of an AET, the actions in which an action type identifier occur must all be
exponentially timed, immediate with the same priority, or passive with the same priority. Third, every input
action must be passive.

If several actions are simultaneously enabled in the current AET behavior, as in the case of the alternative
composition, the one to be executed is selected as follows~\cite{BB2}. If all the considered actions are
exponentially timed, then the race policy applies: each considered action is selected with a probability
proportional to its rate. If some of the considered actions are immediate, then such immediate actions take
precedence over the exponentially timed ones and the generative preselection policy applies: each considered
immediate action with the highest priority is selected with a probability proportional to its weight. If
some of the considered actions are passive, then the reactive preselection policy applies to them: for every
action type, each considered passive action of that type with the highest priority level is selected with a
probability proportional to its weight (the choice among passive actions of different types is
nondeterministic).

An action of an AEI can synchronize with an action of another AEI, possibly exchanging values. Besides other
constraints that we shall see in Sect.~\ref{attachments}, in accordance with the generative-reactive
paradigm~\cite{BB2} it must be the case that at most one of the two actions is not passive, with both
actions being unstructured, or both being structured with at most one of them being an output action and
their parameters matching by number, order, and type. If both actions are passive, then the resulting
synchronizing action is passive as well, otherwise the rate of the resulting action is the (possibly
normalized) rate of the involved non-passive action. The identifier of the resulting action type is
internally represented by concatenating the two original identifiers, using symbol {\tt \#} as a separator.
If one of the two involved actions is an output action, then the resulting action is an output action with
the same non-empty sequence of expressions as the original output action. If instead both involved actions
are input actions, then the resulting action is an input action with the same non-empty sequence of local
variables as one of the two original input actions.


%%%%%%%%%%%%%%%%%%%%%%%%%%%%%%%%%%%%%%%%%%%%%%%%%%%%%%%%%%%%%%%%%
%                                                               %
\subsection{AET Interactions}
%                                                               %
%%%%%%%%%%%%%%%%%%%%%%%%%%%%%%%%%%%%%%%%%%%%%%%%%%%%%%%%%%%%%%%%%

The identifiers of the types of the actions occurring in the behavior of an AET through which the instances
of that AET can communicate with other AEIs are declared to be interactions as follows:

	\begin{verbatim}
    "INPUT_INTERACTIONS" <input_interactions> "OUTPUT_INTERACTIONS" <output_interactions>
	\end{verbatim}

Every interaction has two qualifiers associated with it. First, an interaction is classified to be either
an input or an output interaction based on its communication direction, i.e.\ whether it tries to establish
a communication or is willing to be involved in a communication. In the particular case of an interaction
given by an action type identifier occurring in input (resp.\ output) actions, the action type must be
declared to be an input (resp.\ output) interaction.

Second, an input or output interaction is classified to be a uni-, and- or or-interaction depending on the
multiplicity of the communications in which it can be involved. Syntactically speaking, each of
{\tt <input\_interactions>} and {\tt <output\_interactions>} either is {\tt void} or has the following
format:

	\begin{verbatim}
    <uni_interactions> <and_interactions> <or_interactions>
	\end{verbatim}

\noindent with at least one of the three elements, which basically represent sequences of action type
identifiers, being non-empty.

A uni-interaction of an instance of an AET can communicate only with one interaction of another AEI
(point-to-point communication). If not empty, {\tt <uni\_interactions>} has the following syntax:

	\begin{verbatim}
    "UNI" <identifier_sequence>
	\end{verbatim}

\noindent where {\tt <identifier\_sequence>} is a non-empty sequence of semicolon-separated action type
identifiers.

An and-interaction of an instance of an AET can simultaneously communicate with several interactions of
other AEIs (broadcast communication). If not empty, {\tt <and\_interactions>} has the following syntax:

	\begin{verbatim}
    "AND" <identifier_sequence>
	\end{verbatim}

\noindent where {\tt <identifier\_sequence>} is a non-empty sequence of semicolon-separated action type
identifiers. Due to the adoption of the generative-reactive paradigm, the identifier of an action type
occurring in input actions cannot be declared to be an (input) and-interaction.

An or-interaction of an instance of an AET can communicate with one of several interactions of other AEIs
(server-clients communication). If not empty, {\tt <or\_interactions>} has the following syntax:

	\begin{verbatim}
    "OR" <identifier_sequence>
	\end{verbatim}

\noindent where {\tt <identifier\_sequence>} is a non-empty sequence of semicolon-separated action type
identifiers. Internally, every occurrence of an or-interaction is replaced within the AET behavior by a
choice among as many fresh uni-interactions as there are AEIs with which the original or-interaction can
communicate. Each such fresh uni-interaction is represented through the identifier of the original
or-interaction augmented with a dot followed by a unique index.



%%%%%%%%%%%%%%%%%%%%%%%%%%%%%%%%%%%%%%%%%%%%%%%%%%%%%%%%%%%%%%%%%
%                                                               %
%                                                               %
\section{Architectural Topology}
%                                                               %
%                                                               %
%%%%%%%%%%%%%%%%%%%%%%%%%%%%%%%%%%%%%%%%%%%%%%%%%%%%%%%%%%%%%%%%%

The second section of an \aemilia\ specification has the following syntax:

	\begin{verbatim}
    "ARCHI_TOPOLOGY" <AEIs> <architectural_interactions> <architectural_attachments>
	\end{verbatim}


%%%%%%%%%%%%%%%%%%%%%%%%%%%%%%%%%%%%%%%%%%%%%%%%%%%%%%%%%%%%%%%%%
%                                                               %
\subsection{Architectural Element Instances}
%                                                               %
%%%%%%%%%%%%%%%%%%%%%%%%%%%%%%%%%%%%%%%%%%%%%%%%%%%%%%%%%%%%%%%%%

The instances of the AETs defined in the first section of an \aemilia\ specification are declared as
follows:

	\begin{verbatim}
    "ARCHI_ELEM_INSTANCES" <AEI_decl_sequence>
	\end{verbatim}

\noindent where {\tt <AEI\_decl\_sequence>} is a non-empty sequence of semicolon-separated AEI declarations,
each of the following form:

	\begin{verbatim}
    <AEI_decl> ::= <identifier> ["[" <expr> "]"] ":" <identifier> "(" <pe_expr_sequence> ")"
                |  "FOR_ALL" <identifier> "IN" <expr> ".." <expr>
                   <identifier> "[" <expr> "]" ":" <identifier> "(" <pe_expr_sequence> ")"
	\end{verbatim}

In its simpler form, an AEI declaration contains the identifier of the AEI, a possible integer-valued
expression enclosed in square brackets, which represents a selector and must be free of undeclared
identifiers and invocations to pseudo-random number generators, the identifier of the related AET, which
must have been defined in the first section of the \aemilia\ specification, and a possibly empty sequence of
expressions free of invocations to pseudo-random number generators, which provide the actual values for the
constant formal parameters of the AET and must match with them by number, order, and type. The only
identifiers that can occur in the possible selector expression and in the actual parameters are the ones of
the constant formal parameters declared in the architectural type header.

The second form is useful to concisely declare several instances of the same AET through an indexing
mechanism. This additionally requires the specification of the index identifier, which can then occur in the
selector expression and in the actual parameters, together with its range, which is given by two
integer-valued expressions. These two expressions must be free of undeclared identifiers and invocations to
pseudo-random number generators, with the value of the first expression being not greater than the value of
the second expression.

We observe that the identifier of an AEI can be augmented with a selector expression also in the simpler
form of AEI declaration. This is useful whenever it is desirable to declare a set of indexed instances of
the same AET, but only some of them have a common selector expression.


%%%%%%%%%%%%%%%%%%%%%%%%%%%%%%%%%%%%%%%%%%%%%%%%%%%%%%%%%%%%%%%%%
%                                                               %
\subsection{Architectural Interactions}
%                                                               %
%%%%%%%%%%%%%%%%%%%%%%%%%%%%%%%%%%%%%%%%%%%%%%%%%%%%%%%%%%%%%%%%%

The architectural interactions are declared through the following syntax:

	\begin{verbatim}
    "ARCHI_INTERACTIONS" <pe_architectural_interaction_decl>
	\end{verbatim}

\noindent where {\tt <pe\_architectural\_interaction\_decl>} is either {\tt void} or a non-empty sequence of
semicolon-separated architectural interaction declarations, each of the following form:

	\begin{verbatim}
    <architectural_interaction_decl> ::= <identifier> ["[" <expr> "]"] "." <identifier>
                                      |  "FOR_ALL" <identifier> "IN" <expr> ".." <expr>
                                         <identifier> "[" <expr> "]" "." <identifier>
	\end{verbatim}

In its simpler form, an architectural interaction declaration contains the identifier of the AEI to which
the interaction belongs, a possible integer-valued expression enclosed in square brackets, which represents
a selector and must be free of undeclared identifiers and invocations to pseudo-random number generators,
and the identifier of the interaction. Both the AEI and the interaction for the type of the AEI, whose
identifier concatenation through the dot notation gives rise to the name of the architectural interaction,
must have been previously declared. The only identifiers that can occur in the possible selector expression
are the ones of the constant formal parameters declared in the architectural type header.

The second form is useful to concisely declare several architectural interactions through an indexing
mechanism. This additionally requires the specification of the index identifier, which can then occur in the
selector expression, together with its range, which is given by two integer-valued expressions. These two
expressions must be free of undeclared identifiers and invocations to pseudo-random number generators, with
the value of the first expression being not greater than the value of the second expression.


%%%%%%%%%%%%%%%%%%%%%%%%%%%%%%%%%%%%%%%%%%%%%%%%%%%%%%%%%%%%%%%%%
%                                                               %
\subsection{Architectural Attachments}\label{attachments}
%                                                               %
%%%%%%%%%%%%%%%%%%%%%%%%%%%%%%%%%%%%%%%%%%%%%%%%%%%%%%%%%%%%%%%%%

The architectural attachments are declared through the following syntax:

	\begin{verbatim}
    "ARCHI_ATTACHMENTS" <pe_architectural_attachment_decl>
	\end{verbatim}

\noindent where {\tt <pe\_architectural\_attachment\_decl>} is either {\tt void} or a non-empty sequence of
semicolon-separated architectural attachment declarations, each of the following form:

	\begin{verbatim}
    <architectural_attachment_decl> ::= "FROM" <identifier> ["[" <expr> "]"] "." <identifier>
                                        "TO" <identifier> ["[" <expr> "]"] "." <identifier>
                                     |  "FOR_ALL" <identifier> "IN" <expr> ".." <expr>
                                        ["AND" "FOR_ALL" <identifier> "IN" <expr> ".." <expr>]
                                        "FROM" <identifier> ["[" <expr> "]"] "." <identifier>
                                        "TO" <identifier> ["[" <expr> "]"] "." <identifier>
	\end{verbatim}

In its simpler form, an architectural attachment declaration contains the indication of an output
interaction followed by the indication of an input interaction. Each of the two interactions is expressed in
dot notation through the identifier of the AEI to which the interaction belongs, a possible integer-valued
expression enclosed in square brackets, which represents a selector and must be free of undeclared
identifiers and invocations to pseudo-random number generators, and the identifier of the interaction. Both
the AEI and the interaction for the type of the AEI must have been previously declared, with the interaction
not being architectural. The two AEIs must be different from each other. At least one of the two
interactions must be a uni-interaction, and at least one of them must occur in passive actions within the
behavior of the AEI to which it belongs. The actions in which the two interactions occur within the behavior
of the two AEIs must all be either unstructured or structured; in the latter case, the expressions of the
output interaction must match with the local variables of the input interaction by number, order, and type.
The only identifiers that can occur in the possible selector expression are the ones of the constant formal
parameters declared in the architectural type header.

The second form is useful to concisely declare several architectural attachments through an indexing
mechanism. This additionally requires the specification of up to two different index identifiers, which can
then occur in the selector expressions, together with their ranges, each of which is given by two
integer-valued expressions. These two expressions must be free of undeclared identifiers and invocations to
pseudo-random number generators, with the value of the first expression being not greater than the value of
the second expression.

All the non-architectural interactions should be involved at least in one architectural attachment, with the
non-architectural uni-interactions being involved at most in one architectural attachment. All the
uni-interactions attached to the same and- or or-interaction must belong to different AEIs. Among all the
uni-interactions attached to the same passive and-interaction, at most one is admitted that occurs in
non-passive actions within the behavior of the AEI to which it belongs. No isolated groups of AEIs should be
present.



%%%%%%%%%%%%%%%%%%%%%%%%%%%%%%%%%%%%%%%%%%%%%%%%%%%%%%%%%%%%%%%%%
%                                                               %
%                                                               %
\section{Behavioral Variations}
%                                                               %
%                                                               %
%%%%%%%%%%%%%%%%%%%%%%%%%%%%%%%%%%%%%%%%%%%%%%%%%%%%%%%%%%%%%%%%%

The third section of an \aemilia\ specification has the following syntax:

	\begin{verbatim}
    ["BEHAV_VARIATIONS" [<behav_hidings>] [<behav_restrictions>] [<behav_renamings>]]
	\end{verbatim}

\noindent This section is optional. If present, at least one of its three optional subsections must be
there.


%%%%%%%%%%%%%%%%%%%%%%%%%%%%%%%%%%%%%%%%%%%%%%%%%%%%%%%%%%%%%%%%%
%                                                               %
\subsection{Behavioral Hidings}
%                                                               %
%%%%%%%%%%%%%%%%%%%%%%%%%%%%%%%%%%%%%%%%%%%%%%%%%%%%%%%%%%%%%%%%%

The behavioral hidings are declared through the following syntax:

	\begin{verbatim}
    "BEHAV_HIDINGS" <behav_hiding_decl_sequence>
	\end{verbatim}

\noindent where {\tt <behav\_hiding\_decl\_sequence>} is a non-empty sequence of semicolon-separated
behavioral hiding declarations, each of the following form:

	\begin{verbatim}
    <behav_hiding_decl> ::= "HIDE" "INTERNALS"
                         |  "HIDE" "INTERACTIONS"
                         |  "HIDE" "ALL"
                         |  "HIDE" <identifier> ["[" <expr> "]"] "." <action_type_set_h>
                         |  "FOR_ALL" <identifier> "IN" <expr> ".." <expr>
                            "HIDE" <identifier> "[" <expr> "]" "." <action_type_set_h>
    <action_type_set_h> ::= <identifier>
                         |  "INTERNALS"
                         |  "INTERACTIONS"
                         |  "ALL"
	\end{verbatim}

In its simpler form, a behavioral hiding declaration consists of making unobservable all the action types
that are internal to the AEIs of the \aemilia\ specification, all the non-architectural interactions of the
AEIs of the \aemilia\ specification, or both of them. Alternatively, it is possible to hide a set of action
types of a specific AEI. In this case, the behavioral hiding declaration contains the identifier of the AEI
to which the action types to be hidden belong, a possible integer-valued expression enclosed in square
brackets, which represents a selector and must be free of undeclared identifiers and invocations to
pseudo-random number generators, and the identifier of the action type to be hidden or one of the three
shorthands above for sets of action types to be hidden. If specified, the AEI must have been previously
declared. If specified, the action type to be hidden must occur in the behavior of the AEI and cannot be an
architectural interaction. The only identifiers that can occur in the possible selector expression are the
ones of the constant formal parameters declared in the architectural type header.

The more complex form is useful to concisely hide some of the action types of several AEIs through an
indexing mechanism. This additionally requires the specification of the index identifier, which can then
occur in the selector expression, together with its range, which is given by two integer-valued expressions.
These two expressions must be free of undeclared identifiers and invocations to pseudo-random number
generators, with the value of the first expression being not greater than the value of the second
expression.

If an internal action type is hidden, it is converted to the special action type {\tt invisible}. If a
non-architectural interaction is hidden, all the synchronizing action types in which it is involved are
converted to the special action type {\tt invisible}. In both cases, all the possible action type parameters
are dropped.


%%%%%%%%%%%%%%%%%%%%%%%%%%%%%%%%%%%%%%%%%%%%%%%%%%%%%%%%%%%%%%%%%
%                                                               %
\subsection{Behavioral Restrictions}
%                                                               %
%%%%%%%%%%%%%%%%%%%%%%%%%%%%%%%%%%%%%%%%%%%%%%%%%%%%%%%%%%%%%%%%%

The behavioral restrictions are declared through the following syntax:

	\begin{verbatim}
    "BEHAV_RESTRICTIONS" <behav_restriction_decl_sequence>
	\end{verbatim}

\noindent where {\tt <behav\_restriction\_decl\_sequence>} is a non-empty sequence of semicolon-separated
behavioral restriction declarations, each of the following form:

	\begin{verbatim}
    <behav_restriction_decl> ::= "RESTRICT" "OBS_INTERNALS"
                              |  "RESTRICT" "OBS_INTERACTIONS"
                              |  "RESTRICT" "ALL_OBSERVABLES"
                              |  "RESTRICT" <identifier> ["[" <expr> "]"] "." <action_type_set_r>
                              |  "FOR_ALL" <identifier> "IN" <expr> ".." <expr>
                                 "RESTRICT" <identifier> "[" <expr> "]" "." <action_type_set_r>
         <action_type_set_r> ::= <identifier>
                              |  "OBS_INTERNALS"
                              |  "OBS_INTERACTIONS"
                              |  "ALL_OBSERVABLES"
	\end{verbatim}

In its simpler form, a behavioral restriction declaration consists of preventing the execution of all the
observable action types that are internal to the AEIs of the \aemilia\ specification, all the observable,
non-architectural interactions of the AEIs of the \aemilia\ specification, or both of them. Alternatively,
it is possible to restrict a set of action types of a specific AEI. In this case, the behavioral restriction
declaration contains the identifier of the AEI to which the action types to be restricted belong, a possible
integer-valued expression enclosed in square brackets, which represents a selector and must be free of
undeclared identifiers and invocations to pseudo-random number generators, and the identifier of the action
type to be restricted or one of the three shorthands above for sets of action types to be restricted. If
specified, the AEI must have been previously declared. If specified, the action type to be hidden must occur
in the behavior of the AEI and cannot be an architectural interaction or hidden. The only identifiers that
can occur in the possible selector expression are the ones of the constant formal parameters declared in the
architectural type header.

The more complex form is useful to concisely restrict some of the action types of several AEIs through an
indexing mechanism. This additionally requires the specification of the index identifier, which can then
occur in the selector expression, together with its range, which is given by two integer-valued expressions.
These two expressions must be free of undeclared identifiers and invocations to pseudo-random number
generators, with the value of the first expression being not greater than the value of the second
expression.

If an observable internal action type is restricted, it cannot be executed. If an observable,
non-architectural interaction is restricted, none of the synchronizing action types in which it is involved
can be executed.


%%%%%%%%%%%%%%%%%%%%%%%%%%%%%%%%%%%%%%%%%%%%%%%%%%%%%%%%%%%%%%%%%
%                                                               %
\subsection{Behavioral Renamings}
%                                                               %
%%%%%%%%%%%%%%%%%%%%%%%%%%%%%%%%%%%%%%%%%%%%%%%%%%%%%%%%%%%%%%%%%

The behavioral renamings are declared through the following syntax:

	\begin{verbatim}
    "BEHAV_RENAMINGS" <behav_renaming_decl_sequence>
	\end{verbatim}

\noindent where {\tt <behav\_renaming\_decl\_sequence>} is a non-empty sequence of semicolon-separated
behavioral renaming declarations, each of the following form:

	\begin{verbatim}
    <behav_renaming_decl> ::= "RENAME" <identifier> ["[" <expr> "]"] "." <identifier>
                              "AS" <identifier> ["[" <expr> "]"]
                           |  "FOR_ALL" <identifier> "IN" <expr> ".." <expr>
                              "RENAME" <identifier> "[" <expr> "]" "." <identifier>
                              "AS" <identifier> ["[" <expr> "]"]
	\end{verbatim}

In its simpler form, a behavioral renaming declaration contains the identifier of the AEI to which the
action type to be renamed belongs, a possible integer-valued expression enclosed in square brackets, which
represents a selector and must be free of undeclared identifiers and invocations to pseudo-random number
generators, and the identifier of the renaming action type possibly followed by another selector expression.
The AEI must have been previously declared. The action type to be renamed must occur in the behavior of the
AEI and cannot be hidden or restricted. The only identifiers that can occur in the possible selector
expressions are the ones of the constant formal parameters declared in the architectural type header.

The mroe complex form is useful to concisely rename some of the action types of several AEIs through an
indexing mechanism. This additionally requires the specification of the index identifier, which can then
occur in the selector expressions, together with its range, which is given by two integer-valued
expressions. These two expressions must be free of undeclared identifiers and invocations to pseudo-random
number generators, with the value of the first expression being not greater than the value of the second
expression.

An observable, non-restricted action type can be renamed to a single action type. If an observable,
non-restricted, internal action type or an architectural interaction is renamed, it is converted to the
specified renaming action type. If an observable, non-restricted, non-architectural interaction is renamed,
all the synchronizing action types in which it is involved are converted to the specified renaming action
type. Note that the renaming action type is not expressed in dot notation, which may turn out to be useful
when checking for equivalence two \aemilia\ specifications whose AEIs have different identifiers. In the
model checking case, instead, any renaming action type must be different from the NuSMV~2.2.5 keywords,
which are listed below:

	\begin{verbatim}
                A               EG              LTLSPEC      Z
                ABF             EVAL            LTLWFF       apropos
                ABG             EX              MAX          array
                AF              F               MIN          boolean
                AG              FAIRNESS        MODULE       case
                ASSIGN          FALSE           O            else
                ASYNC           FORMAT          OUTPUT       esac
                AX              G               RESET        if
                BU              GOTO            S            in
                CTLWFF          H               SIMPWFF      init
                COMPASSION      IMPLEMENTS      SPEC         mod
                COMPUTE         IN              STEP         next
                COMPWFF         INIT            T            of
                CONSTANT        INPUT           TRANS        process
                CONSTRAINT      INVAR           TRUE         self
                DEFINE          INVARSPEC       U            sigma
                E               ISA             V            then
                EBF             IVAR            VAR          union
                EBG             JUSTICE         X            xnor
                EF              LET             Y            xor
	\end{verbatim}



%%%%%%%%%%%%%%%%%%%%%%%%%%%%%%%%%%%%%%%%%%%%%%%%%%%%%%%%%%%%%%%%%
%                                                               %
%                                                               %
\section{Compiling \aemilia\ Specifications}
%                                                               %
%                                                               %
%%%%%%%%%%%%%%%%%%%%%%%%%%%%%%%%%%%%%%%%%%%%%%%%%%%%%%%%%%%%%%%%%

In this section we briefly describe how correct \aemilia\ specifications are compiled into finite semantic
models suited for analysis.


%%%%%%%%%%%%%%%%%%%%%%%%%%%%%%%%%%%%%%%%%%%%%%%%%%%%%%%%%%%%%%%%%
%                                                               %
\subsection{Parsing}
%                                                               %
%%%%%%%%%%%%%%%%%%%%%%%%%%%%%%%%%%%%%%%%%%%%%%%%%%%%%%%%%%%%%%%%%

While parsing an \aemilia\ (or companion) specification, a {\tt .lis} file is generated in which each line
of the specification is reported by having it preceded by its line number. The parser is able to catch about
300~types of lexical, syntax and static semantic error or warning, which are signalled through suitable
messages in the {\tt .lis} file. The {\tt .lis} file is terminated with the indication of the total number
of errors and warnings that have been detected. A specification with no errors is said to be correct.


%%%%%%%%%%%%%%%%%%%%%%%%%%%%%%%%%%%%%%%%%%%%%%%%%%%%%%%%%%%%%%%%%
%                                                               %
\subsection{Semantic Models}
%                                                               %
%%%%%%%%%%%%%%%%%%%%%%%%%%%%%%%%%%%%%%%%%%%%%%%%%%%%%%%%%%%%%%%%%

A correct \aemilia\ specification can be compiled into three semantic models: the integrated semantic model,
the functional semantic model, and the performance semantic model. All of them are state transition graphs,
whose states are in correspondence with the vectors of the current behaviors of the AEIs. More precisely,
the integrated semantic model is a state transition graph whose transitions are labeled with the type and
the rate of the corresponding actions, with the lower priority transitions being pruned. The functional
semantic model is a state transition graph in which only the action types label the transitions. The
performance semantic model can be extracted in the form of a Markov chain~\cite{Ste1} only if the \aemilia\
specification is performance closed, i.e.\ its integrated semantic model has no passive transitions and no
non-determinism arises because of some boolean condition -- occurring in a behavioral choice -- that cannot
be statically evaluated. In such a case, if the integrated semantic model has only exponentially timed
transitions or both immediate and exponentially timed transitions, the performance semantic model is a
continuous-time Markov chain obtained after removing the possible immediate transitions, with the
transitions labeled with the rates of the corresponding actions. If instead the integrated semantic model
has only immediate transitions, they are interpreted as taking one time unit and the performance semantic
model is a discrete-time Markov chain, with the transitions labeled with the probabilities of the
corresponding actions.

Due to the process term syntax within the behavior of the AETs, which rules out static operators (hence
recursion over them) as well as behavioral equation invocations outside the scope of an action prefix, the
semantic model of an \aemilia\ specification with no variable formal parameters and local variables is
guaranteed to be finitely branching and finite state.

When printed to the related {\tt .ism}, {\tt .fsm} or {\tt .psm} file, the semantic model is a sequence of
numbered global states with their outgoing transitions. Each global state is described through the vector of
local states representing the current behavior of the AEIs. In the case of the integrated and functional
semantic models, the global state numbered with one is the initial global state, which is composed of the
local states representing the initial behavior of the AEIs. In the case of the performance semantic model,
the concept of initial global state is replaced with an initial global state probability distribution -- as
the initial global state may not be unique -- whose values are reported in the {\tt .psm} file for every
global state. Each transition is represented through its action-based label and the number of its target
global state.

The {\tt .ism}, {\tt .fsm} or {\tt .psm} file is terminated with the indication of the total number of
global states and transitions and their classification. The global states are divided into tangible (having
exponentially timed transitions), vanishing (having immediate transitions), open (having passive
transitions), and deadlocked (having no transitions). In the case of the performance semantic model, the
global states with no outgoing transitions are called absorbing rather than deadlocked as they might have
self-looping transitions. Based on their action types, the transitions are divided into observable and
invisible. Based on their action rates, the transitions are divided into exponentially timed, immediate, and
passive.

The same global state and transition classification is reported, together with the related numbers, in a
{\tt .siz} file whenever the user is interested only in the size of the semantic model.


%%%%%%%%%%%%%%%%%%%%%%%%%%%%%%%%%%%%%%%%%%%%%%%%%%%%%%%%%%%%%%%%%
%                                                               %
\subsection{Concrete and Symbolic Representation of Data Values}
%                                                               %
%%%%%%%%%%%%%%%%%%%%%%%%%%%%%%%%%%%%%%%%%%%%%%%%%%%%%%%%%%%%%%%%%

In the case in which variable formal parameters and local variables are present in the behavioral equations
of the AETs of an \aemilia\ specification, they must be compiled in a way that keeps the underlying semantic
models both finitely branching and finite state. This can be achieved in two different ways.

If each of the occurring variable formal parameters and local variables has a finite value domain, i.e.\ its
type is bounded integer, boolean, or array or record based on the two previous basic types, then a concrete
treatment is applied. This means that every expression -- occurring in a behavioral equation invocation, a
boolean guard, or an output action -- can be statically evaluated, after replacing every input action with a
choice among as many instances of it as needed to instantiate all of its local variables in every possible
way according to their finite value domains. In this case, the synchronization between an output action and
an input action is possible only if their parameters have pairwise the same value. In the {\tt .ism},
{\tt .fsm} or {\tt .psm} file, the current value of every variable formal parameter and local variable is
represented through a concrete assignment -- an assignment whose right-hand side does not contain any
variable -- printed immediately after the current behavior of the AEI to which the variable formal parameter
or local variable belongs.

If instead there is at least one variable formal parameter or local variable with an infinite value domain,
i.e.\ whose type is integer, real, list, or array or record based on the three previous types, then a
concrete treatment is not possible. To keep the semantic models finite, a symbolic treatment is
applied~\cite{Ber2}. Every transition label is augmented with an expression and a sequence of symbolic
assignments. The expression, which arises from the possible boolean guards of the alternative composition
operator, represents the condition under which the transition can be executed. The symbolic assignments
establish how the values of the variable formal parameters and of the local variables must be updated after
executing the transition, due to behavioral equation invocations in the target global state or value passing
between the synchronized structured actions involved in the transition. In the {\tt .ism}, {\tt .fsm} or
{\tt .psm} file, every transition is printed together with its boolean guard and its sequence of symbolic
assignments. Moreover, a list of initial symbolic assignments, based on the initialization of the variable
formal parameters in the header of the first behavioral equation of every AEI, is printed at the beginning
of the file. We remind that the performance semantic model cannot be generated in the case of symbolic
treatment if there is at least one boolean guard that cannot be statically evaluated.

We conclude by recalling that equivalence verification, model checking, security analysis, and
Markov-chain-based performance evaluation can be applied only to correct \aemilia\ specifications in which
all the possible variable formal parameters and local variables can be treated concretely. Simulation,
instead, can be applied to any correct \aemilia\ specification (with no open and deadlock states), because
the possible expressions -- including guards and symbolic assignments -- are directly evaluated while
generating the portion of the integrated semantic model that is necessary to make the simulation advance.


%%%%%%%%%%%%%%%%%%%%%%%%%%%%%%%%%%%%%%%%%%%%%%%%%%%%%%%%%%%%%%%%%
%                                                               %
\subsection{Compile-Time Crashes}\label{ct_crashes}
%                                                               %
%%%%%%%%%%%%%%%%%%%%%%%%%%%%%%%%%%%%%%%%%%%%%%%%%%%%%%%%%%%%%%%%%

The compilation of an \aemilia\ specification can be interrupted because of lack of memory, inability to
open a file, illegal value of a symbolic action rate, or an error occurred during the evaluation of a
concrete assignment, like out-of-range bounded integer, symbolic array length mismatch, or invalid argument
of an arithmetical operator, a mathematical function, a pseudo-random number generator, or a list- or
array-related function. In each of these cases, which cannot be detected at parsing time, a suitable message
is displayed.



%%%%%%%%%%%%%%%%%%%%%%%%%%%%%%%%%%%%%%%%%%%%%%%%%%%%%%%%%%%%%%%%%
%                                                               %
%                                                               %
\section{Example A: The Alternating Bit Protocol}
%                                                               %
%                                                               %
%%%%%%%%%%%%%%%%%%%%%%%%%%%%%%%%%%%%%%%%%%%%%%%%%%%%%%%%%%%%%%%%%

In this section we introduce a simple communication protocol and we present three different \aemilia\
specifications for it, which exemplify the syntax of the language and illustrate its expressiveness.


%%%%%%%%%%%%%%%%%%%%%%%%%%%%%%%%%%%%%%%%%%%%%%%%%%%%%%%%%%%%%%%%%
%                                                               %
\subsection{Informal Description}
%                                                               %
%%%%%%%%%%%%%%%%%%%%%%%%%%%%%%%%%%%%%%%%%%%%%%%%%%%%%%%%%%%%%%%%%

The alternating bit protocol~\cite{BSW} is a data-link-level communication protocol that establishes a means
whereby two stations, one acting as a sender and the other acting as a receiver, connected by a full-duplex, FIFO communication channel that may lose messages, can cope with message loss. The name of the protocol
stems from the fact that each message is augmented with an additional bit. Since consecutive messages that
are not lost are tagged with additional bits that are pairwise complementary, it is easy to distinguish
between an original message and its possible duplicates.

Initially, if the data link level of the sender obtains a message from the upper level in the protocol
stack, it augments the message with an additional bit set to~0, sends the tagged message to the receiver,
and starts a timer. If an acknowledgment tagged with~0 is received before the timeout expires, then the
subsequent message obtained from the upper level will be sent with an additional bit set to~1, otherwise the
current tagged message is sent again. On the other side, the data link level of the receiver waits for a
message tagged with~0. If it receives such a tagged message for the first time, then it passes the message
to the upper level in the protocol stack, sends an acknowledgment tagged with~0 back to the sender, and
waits for a message tagged with~1. On the contrary, if it receives a duplicate tagged message -- due to
message loss, acknowledgment loss, or propagation taking an exceedingly long time -- then it sends an
acknowledgment tagged with the same additional bit back to the sender and keeps waiting.


%%%%%%%%%%%%%%%%%%%%%%%%%%%%%%%%%%%%%%%%%%%%%%%%%%%%%%%%%%%%%%%%%
%                                                               %
\subsection{Pure \aemilia\ Description with Markovian Delays}\label{abp}
%                                                               %
%%%%%%%%%%%%%%%%%%%%%%%%%%%%%%%%%%%%%%%%%%%%%%%%%%%%%%%%%%%%%%%%%

We show below a pure \aemilia\ specification of the alternating bit protocol called {\tt abp.aem}, where
pure means that no variable formal parameters and local variables are used:

	\begin{verbatim}
    ARCHI_TYPE ABP_Type(const rate   msg_gen_rate  := 5,
                        const rate   timeout_rate  := 1,
                        const rate   prop_rate     := 9.375,
                        const weight delivery_prob := 0.95)

    ARCHI_ELEM_TYPES

      ELEM_TYPE Sender_Type(const rate msg_gen_rate,
                            const rate timeout_rate)

        BEHAVIOR

          Sender_0(void; void) =
            <generate_msg, exp(msg_gen_rate)> . <transmit_msg_0, inf> . Sender_0_Waiting();

          Sender_0_Waiting(void; void) =
            choice
            {
              <receive_ack_0, _> . Sender_1(),
              <receive_ack_1, _> . Sender_0_Waiting(),
              <timeout, exp(timeout_rate)> . Sender_0_Retransmitting()
            };

          Sender_0_Retransmitting(void; void) =
            choice
            {
              <transmit_msg_0, inf> . Sender_0_Waiting(),
              <receive_ack_0, _> . Sender_1(),
              <receive_ack_1, _> . Sender_0_Retransmitting()
            };

          Sender_1(void; void) =
            <generate_msg, exp(msg_gen_rate)> . <transmit_msg_1, inf> . Sender_1_Waiting();

          Sender_1_Waiting(void; void) =
            choice
            {
              <receive_ack_1, _> . Sender_0(),
              <receive_ack_0, _> . Sender_1_Waiting(),
              <timeout, exp(timeout_rate)> . Sender_1_Retransmitting()
            };

          Sender_1_Retransmitting(void; void) =
            choice
            {
              <transmit_msg_1, inf> . Sender_1_Waiting(),
              <receive_ack_1, _> . Sender_0(),
              <receive_ack_0, _> . Sender_1_Retransmitting()
            }

        INPUT_INTERACTIONS

          UNI generate_msg;
              receive_ack_0;
              receive_ack_1

        OUTPUT_INTERACTIONS

          UNI transmit_msg_0;
              transmit_msg_1

      ELEM_TYPE Line_Type(const rate   prop_rate,
                          const weight delivery_prob)

        BEHAVIOR

          Line(void; void) =
            choice
            {
              <receive_0, _> . <propagate_0, exp(prop_rate)> .
                choice
                {
                  <keep_0, inf(1, delivery_prob)> . <deliver_0, inf> . Line(),
                  <lose_0, inf(1, 1 - delivery_prob)> . Line()
                },
              <receive_1, _> . <propagate_1, exp(prop_rate)> .
                choice
                {
                  <keep_1, inf(1, delivery_prob)> . <deliver_1, inf> . Line(),
                  <lose_1, inf(1, 1 - delivery_prob)> . Line()
                }
            }

        INPUT_INTERACTIONS

          UNI receive_0;
              receive_1

        OUTPUT_INTERACTIONS

          UNI deliver_0;
              deliver_1

      ELEM_TYPE Receiver_Type(void)

        BEHAVIOR

          Receiver_0(void; void) =
            choice
            {
              <receive_msg_0, _> . <consume_msg, inf> . <transmit_ack_0, inf> . Receiver_1(),
              <receive_msg_1, _> . <transmit_ack_1, inf> . Receiver_0()
            };

          Receiver_1(void; void) =
            choice
            {
              <receive_msg_1, _> . <consume_msg, inf> . <transmit_ack_1, inf> . Receiver_0(),
              <receive_msg_0, _> . <transmit_ack_0, inf> . Receiver_1()
            }

        INPUT_INTERACTIONS

          UNI receive_msg_0;
              receive_msg_1

        OUTPUT_INTERACTIONS

          UNI consume_msg;
              transmit_ack_0;
              transmit_ack_1

    ARCHI_TOPOLOGY

      ARCHI_ELEM_INSTANCES

        S  : Sender_Type(msg_gen_rate,
                         timeout_rate);
        LM : Line_Type(prop_rate,
                       delivery_prob);
        LA : Line_Type(prop_rate,
                       delivery_prob);
        R  : Receiver_Type()

      ARCHI_INTERACTIONS

        S.generate_msg;
        R.consume_msg

      ARCHI_ATTACHMENTS

        FROM S.transmit_msg_0 TO LM.receive_0;
        FROM S.transmit_msg_1 TO LM.receive_1;
        FROM LM.deliver_0     TO R.receive_msg_0;
        FROM LM.deliver_1     TO R.receive_msg_1;
        FROM R.transmit_ack_0 TO LA.receive_0;
        FROM R.transmit_ack_1 TO LA.receive_1;
        FROM LA.deliver_0     TO S.receive_ack_0;
        FROM LA.deliver_1     TO S.receive_ack_1

    END
	\end{verbatim}

The \aemilia\ specification above is parameterized with respect to the message generation rate, the timeout
rate, the propagation rate of a single message, and the delivery probability of a single message. In order
to allow for a Markov-chain-based performance evaluation, the timeout delay and the message propagation
delay are assumed to be exponentially distributed. Supposed that the protocol uses two 9.6~Kbps lines and
that the average length of a message is 1024~bits, the propagation rate is 9.375~messages per second. We
also assume that the generation rate is 5~messages per second, the timeout delay is 1~second on average, and
the delivery probability is~0.95.

We have three AETs: one for the sender, one for a half-duplex communication line, and one for the receiver.
As far as {\tt Sender\_Type} is concerned, initially a message is generated (i.e.\ arrives from the upper
level), which is then tagged with~0 and transmitted. Three cases arise. If an acknowledgment tagged with~0
is received, then the same behavior is repeated for the next generated message, which will be tagged with~1.
If an acknowledgment tagged with~1 is received, the acknowledgment is simply ignored. If the timeout
expires, then the message is retransmitted unless an acknowledgment tagged with~0 is received in the
meanwhile.

{\tt Line\_Type} waits for the transmission of a message/acknowledgment tagged with~0 or~1, which is then
propagated along the line. With probability~0.95 the message/acknowledgment reaches its destination, while
with probability~0.05 the message/acknowledgment is lost. Afterwards, this behavior is repeated. Note that
the types of the actions representing the fact that a message/acknowledgment is kept or lost are not
declared to be interactions, as these events are not under the control of the protocol.

{\tt Receiver\_Type} initially waits for a message tagged with~0. If it is received, then the message tagged
with~0 is passed to the upper level, an acknowledgment tagged with~0 is sent back, and this behavior is
repeated for the next expected message, which shall be tagged with~1. As long as a message tagged with~1 is
received instead, the message is ignored and an acknowledgment tagged with~1 is sent back.

The architectural topology section contains the declaration of one instance of {\tt Sender\_Type}, two
instances of {\tt Line\_Type} -- one for the messages and one for the acknowledgments -- and one instance of
{\tt Receiver\_Type}. The interactions {\tt S.generate\_msg} and {\tt R.consume\_msg} are declared to be
architectural, as they are the access points for the upper levels of the protocol stack both at the sender
side and at the receiver side.

Here is the size of the semantic models of {\tt abp.aem}:

	\begin{verbatim}
    Size of the integrated semantic model underlying ABP_Type:

    - 302 states: 
      - 76 tangible, 
      - 226 vanishing, 
      - 0 open, 
      - 0 deadlocked;

    - 464 transitions:
      - 464 observable,
      - 0 invisible;
      - 140 exponentially timed,
      - 324 immediate,
      - 0 passive.

    Size of the functional semantic model underlying ABP_Type:

    - 302 states: 
      - 302 nondeadlocked, 
      - 0 deadlocked;

    - 464 transitions:
      - 464 observable,
      - 0 invisible.

    Size of the homogeneous continuous-time Markov chain underlying ABP_Type:

    - 76 states: 
      - 76 nonabsorbing, 
      - 0 absorbing;

    - 204 transitions.
	\end{verbatim}


%%%%%%%%%%%%%%%%%%%%%%%%%%%%%%%%%%%%%%%%%%%%%%%%%%%%%%%%%%%%%%%%%
%                                                               %
\subsection{Value Passing \aemilia\ Description with Markovian Delays}\label{abp_vp}
%                                                               %
%%%%%%%%%%%%%%%%%%%%%%%%%%%%%%%%%%%%%%%%%%%%%%%%%%%%%%%%%%%%%%%%%

A more concise \aemilia\ description can be obtained if the tagging bit is encoded through a boolean
variable that is passed across the components, as shown in the following {\tt abp\_vp.aem}:

	\begin{verbatim}
    ARCHI_TYPE ABP_VP_Type(const boolean starting_bit  := false,
                           const rate    msg_gen_rate  := 5,
                           const rate    timeout_rate  := 1,
                           const rate    prop_rate     := 9.375,
                           const weight  delivery_prob := 0.95)

    ARCHI_ELEM_TYPES

      ELEM_TYPE Sender_Type(const boolean starting_bit,
                            const rate    msg_gen_rate,
                            const rate    timeout_rate)

        BEHAVIOR

          Sender(boolean sent_bit := starting_bit;
                 void) =
            <generate_msg, exp(msg_gen_rate)> . <transmit_msg!(sent_bit), inf> .
              Sender_Waiting(sent_bit);

          Sender_Waiting(boolean       sent_bit;
                         local boolean received_bit) =
            choice
            {
              <receive_ack?(received_bit), _> . Sender_Checking(sent_bit,
                                                                received_bit),
              <timeout, exp(timeout_rate)> . Sender_Retransmitting(sent_bit)
            };

          Sender_Checking(boolean sent_bit,
                          boolean received_bit;
                          void) =
            choice
            {
              cond(received_bit = sent_bit)  ->
                <check_bit, inf> . Sender(!sent_bit),
              cond(received_bit != sent_bit) ->
                <check_bit, inf> . Sender_Waiting(sent_bit)
            };

          Sender_Retransmitting(boolean       sent_bit;
                                local boolean received_bit) =
            choice
            {
              <transmit_msg!(sent_bit), inf> . Sender_Waiting(sent_bit),
              <receive_ack?(received_bit), _> . Sender_Checking(sent_bit,
                                                                received_bit)
            }

        INPUT_INTERACTIONS

          UNI generate_msg;
              receive_ack

        OUTPUT_INTERACTIONS

          UNI transmit_msg

      ELEM_TYPE Line_Type(const rate   prop_rate,
                          const weight delivery_prob)

        BEHAVIOR

          Line(void;
               local boolean tagging_bit) =
            <receive?(tagging_bit), _> . <propagate, exp(prop_rate)> .
              choice
              {
                <keep, inf(1, delivery_prob)> . <deliver!(tagging_bit), inf> . Line(),
                <lose, inf(1, 1 - delivery_prob)> . Line()
              }

        INPUT_INTERACTIONS

          UNI receive

        OUTPUT_INTERACTIONS

          UNI deliver

      ELEM_TYPE Receiver_Type(const boolean starting_bit)

        BEHAVIOR

          Receiver(boolean       expected_bit := starting_bit;
                   local boolean received_bit) =
            <receive_msg?(received_bit), _> .
              choice
              {
                cond(received_bit = expected_bit)  ->
                  <consume_msg, inf> . <transmit_ack!(received_bit), inf> .
                    Receiver(!expected_bit),
                cond(received_bit != expected_bit) ->
                  <transmit_ack!(received_bit), inf> . Receiver(expected_bit)
              }

        INPUT_INTERACTIONS

          UNI receive_msg

        OUTPUT_INTERACTIONS

          UNI consume_msg;
              transmit_ack

    ARCHI_TOPOLOGY

      ARCHI_ELEM_INSTANCES

        S  : Sender_Type(starting_bit,
                         msg_gen_rate,
                         timeout_rate);
        LM : Line_Type(prop_rate,
                       delivery_prob);
        LA : Line_Type(prop_rate,
                       delivery_prob);
        R  : Receiver_Type(starting_bit)

      ARCHI_INTERACTIONS

        S.generate_msg;
        R.consume_msg

      ARCHI_ATTACHMENTS

        FROM S.transmit_msg TO LM.receive;
        FROM LM.deliver     TO R.receive_msg;
        FROM R.transmit_ack TO LA.receive;
        FROM LA.deliver     TO S.receive_ack

    END
	\end{verbatim}

Since all the occurring variable formal parameters and local variables are of type boolean, the concrete
treatment of data values applies. Here is the size of the semantic models of {\tt abp\_vp.aem}:

	\begin{verbatim}
    Size of the integrated semantic model underlying ABP_VP_Type:

    - 366 states: 
      - 76 tangible, 
      - 290 vanishing, 
      - 0 open, 
      - 0 deadlocked;

    - 556 transitions:
      - 556 observable,
      - 0 invisible;
      - 140 exponentially timed,
      - 416 immediate,
      - 0 passive.

    Size of the functional semantic model underlying ABP_VP_Type:

    - 366 states: 
      - 366 nondeadlocked, 
      - 0 deadlocked;

    - 556 transitions:
      - 556 observable,
      - 0 invisible.

    Size of the homogeneous continuous-time Markov chain underlying ABP_VP_Type:

    - 76 states: 
      - 76 nonabsorbing, 
      - 0 absorbing;

    - 214 transitions.
	\end{verbatim}


%%%%%%%%%%%%%%%%%%%%%%%%%%%%%%%%%%%%%%%%%%%%%%%%%%%%%%%%%%%%%%%%%
%                                                               %
\subsection{Value Passing \aemilia\ Description with General Delays}\label{abp_gd}
%                                                               %
%%%%%%%%%%%%%%%%%%%%%%%%%%%%%%%%%%%%%%%%%%%%%%%%%%%%%%%%%%%%%%%%%

The value passing features are not only necessary to express data-driven computations and useful to obtain
more concise \aemilia\ specifications. They also allow for the representation of systems in which some
activities have generally distributed durations. As an example, in the case of the alternating bit protocol,
it is reasonable to assume that the message generation time is exponentially distributed, while it is more
realistic to describe the message propagation time through e.g.\ a normal distribution and the timeout
period through a fixed duration.

This is accomplished using a sub-language of \aemilia\ in which there are no exponentially timed actions. In
other words, a system with generally distributed delays is represented through a discrete-time model, in
which an explicit clock process marks the discrete-time steps for the whole system. The timed events are
treated by means of suitable list-typed variables that store the occurrence times of such events, where the
related occurrence times are expressed through suitable invocations to pseudo-random number generators.
Timer variables are then used in the specification of the various system components to detect when the
occurrence time of a timed event has come, in order to enable the particular action representing the
occurrence of the event itself.

The guidelines above are followed in {\tt abp\_gd.aem} shown below, in which the delays are expressed in
milliseconds, the propagation time is described through a normal distribution, and the timeout period is
described through a fixed duration:

	\begin{verbatim}

    ARCHI_TYPE ABP_GD_Type(const boolean starting_bit      := false,
                           const real    msg_gen_rate      := 0.005,
                           const integer timeout_period    := 1000,
                           const real    prop_delay_mean   := 107.0,
                           const real    prop_delay_st_dev := 7.0,
                           const weight  delivery_prob     := 0.95)

    ARCHI_ELEM_TYPES

      ELEM_TYPE Msg_Gen_Type(const real msg_gen_rate)

        BEHAVIOR

          Msg_Gen(integer time_to_gen_next := ceil(exponential(msg_gen_rate)),
                  integer msg_to_send      := 0;
                  void) =
            choice
            {
              cond(msg_to_send >= 1) ->
                choice
                {
                  <generate_msg, inf(3, 1)> . Msg_Gen_Updating(time_to_gen_next,
                                                               msg_to_send - 1),
                  <idle, inf> . Msg_Gen_Updating(time_to_gen_next,
                                                 msg_to_send)
                },
              cond(msg_to_send = 0)  ->
                <idle, inf> . Msg_Gen_Updating(time_to_gen_next,
                                               msg_to_send)
            };

          Msg_Gen_Updating(integer time_to_gen_next,
                           integer msg_to_send;
                           void) =
            choice
            {
              cond(time_to_gen_next = 0) ->
                <elapse_tick, _> . Msg_Gen(ceil(exponential(msg_gen_rate)),
                                           msg_to_send + 1),
              cond(time_to_gen_next > 0) ->
                <elapse_tick, _> . Msg_Gen(time_to_gen_next - 1,
                                           msg_to_send)
            }

        INPUT_INTERACTIONS

          UNI elapse_tick

        OUTPUT_INTERACTIONS

          UNI generate_msg


      ELEM_TYPE Sender_Type(const boolean starting_bit,
                            const integer timeout_period)

        BEHAVIOR

          Sender(boolean       sent_bit     := starting_bit;
                 local boolean received_bit) =
            choice
            {
              <get_msg, _> . Sender_Transmitting(sent_bit),
              <receive_ack?(received_bit), _> . <elapse_tick, _> . Sender(sent_bit),
              <elapse_tick, _> . Sender(sent_bit)
            };

          Sender_Transmitting(boolean       sent_bit;
                              local boolean received_bit) =
            choice
            {
              <transmit_msg!(sent_bit), inf(2, 1)> . <elapse_tick, _> .
                Sender_Waiting(sent_bit,
                               timeout_period),
              <receive_ack?(received_bit), _> . <elapse_tick, _> .
                Sender_Transmitting(sent_bit),
              <elapse_tick, _> . Sender_Transmitting(sent_bit)
            };

          Sender_Waiting(boolean       sent_bit,
                         integer       time_to_timeout;
                         local boolean received_bit) =
            choice
            {
              cond(time_to_timeout > 0) ->
                choice
                {
                  <receive_ack?(received_bit), _> . Sender_Checking_1(sent_bit,
                                                                      time_to_timeout,
                                                                      received_bit),
                  <elapse_tick, _> . Sender_Waiting(sent_bit,
                                                    time_to_timeout - 1)
                },
              cond(time_to_timeout = 0) ->
                choice
                {
                  <timeout, inf(2, 1)> . Sender_Retransmitting(sent_bit),
                  <receive_ack?(received_bit), _> . Sender_Checking_2(sent_bit,
                                                                      received_bit)
                }
            };

          Sender_Checking_1(boolean sent_bit,
                            integer time_to_timeout,
                            boolean received_bit;
                            void) =
            choice
            {
              cond(received_bit = sent_bit)  ->
                <elapse_tick, _> . Sender(!sent_bit),
              cond(received_bit != sent_bit) ->
                <elapse_tick, _> . Sender_Waiting(sent_bit,
                                                  time_to_timeout - 1)
            };

          Sender_Checking_2(boolean sent_bit,
                            boolean received_bit;
                            void) =
            choice
            {
              cond(received_bit = sent_bit)  ->
                <elapse_tick, _> . Sender(!sent_bit),
              cond(received_bit != sent_bit) ->
                <idle, inf> . Sender_Retransmitting(sent_bit)
            };

          Sender_Retransmitting(boolean       sent_bit;
                                local boolean received_bit) =
            choice
            {
              <transmit_msg!(sent_bit), inf(2, 1)> . <elapse_tick, _> .
                Sender_Waiting(sent_bit,
                               timeout_period),
              <receive_ack?(received_bit), _> . Sender_Checking_2(sent_bit,
                                                                  received_bit),
              <elapse_tick, _> . Sender_Retransmitting(sent_bit)
            }

        INPUT_INTERACTIONS

          UNI get_msg;
              receive_ack;
              elapse_tick

        OUTPUT_INTERACTIONS

          UNI transmit_msg

      ELEM_TYPE Line_Type(const real   prop_delay_mean,
                          const real   prop_delay_st_dev,
                          const weight delivery_prob)

        BEHAVIOR

          Line(list(record(integer time_to_delivery,
                           boolean tag))             prop_queue := list_cons();
               local boolean                         tagging_bit) =
            choice
            {
              <receive?(tagging_bit), _> .
                choice
                {
                  <keep, inf(2, delivery_prob)> .
                    Line_Delivering(concat(prop_queue,
                                           list_cons(record_cons(ceil(normal(prop_delay_mean,
                                                                             prop_delay_st_dev)),
                                                                 tagging_bit)))),
                  <lose, inf(2, 1 - delivery_prob)> . Line_Delivering(prop_queue)
                },
              <idle, inf> . Line_Delivering(prop_queue)
            };

          Line_Delivering(list(record(integer time_to_delivery,
                                      boolean tag))             prop_queue;
                          void) =
            choice
            {
              cond((prop_queue != list_cons()) &&
                   get(time_to_delivery,
                       first(prop_queue)) <= 0) ->
                <deliver!(get(tag,
                              first(prop_queue))), inf(3, 1)> .
                  Line_Propagating(tail(prop_queue)),
              cond((prop_queue = list_cons()) ||
                   get(time_to_delivery,
                       first(prop_queue)) > 0) ->
                <idle, inf> . Line_Propagating(prop_queue)
            };

          Line_Propagating(list(record(integer time_to_delivery,
                                       boolean tag))             prop_queue;
                           void) =
            choice
            {
              cond(prop_queue != list_cons()) ->
                <propagate, inf(2, 1)> . Line_Updating(prop_queue,
                                                       list_cons()),
              cond(prop_queue = list_cons())  ->
                <elapse_tick, _> . Line(prop_queue)
            };

          Line_Updating(list(record(integer time_to_delivery,
                                    boolean tag))             prop_queue,
                        list(record(integer time_to_delivery,
                                    boolean tag))             new_prop_queue;
                        void) =
            choice
            {
              cond(prop_queue != list_cons()) ->
                <compute_time_to_delivery, inf(2, 1)> .
                  Line_Updating(tail(prop_queue),
                                concat(new_prop_queue,
                                       list_cons(record_cons(get(time_to_delivery,
                                                                 first(prop_queue)) - 1,
                                                             get(tag,
                                                                 first(prop_queue)))))),
              cond(prop_queue = list_cons())  ->
                <elapse_tick, _> . Line(new_prop_queue)
            }

        INPUT_INTERACTIONS

          UNI receive;
              elapse_tick

        OUTPUT_INTERACTIONS

          UNI deliver

      ELEM_TYPE Receiver_Type(const boolean starting_bit)

        BEHAVIOR

          Receiver(boolean       expected_bit := starting_bit;
                   local boolean received_bit) =
            choice
            {
              <receive_msg?(received_bit), _> . Receiver_Checking(expected_bit,
                                                                  received_bit),
              <elapse_tick, _> . Receiver(expected_bit)
            };

          Receiver_Checking(boolean expected_bit,
                            boolean received_bit;
                            void) =
            choice
            {
              cond(received_bit = expected_bit)  ->
                <consume_msg, inf(2, 1)> . Receiver_Transmitting(!expected_bit,
                                                                 received_bit),
              cond(received_bit != expected_bit) ->
                <idle, inf> . Receiver_Transmitting(expected_bit,
                                                    received_bit)
            };

          Receiver_Transmitting(boolean expected_bit,
                                boolean received_bit;
                                void) =
            choice
            {
              <transmit_ack!(received_bit), inf(2, 1)> . <elapse_tick, _> .
                Receiver(expected_bit),
              <elapse_tick, _> . Receiver_Transmitting(expected_bit,
                                                       received_bit)
            }

        INPUT_INTERACTIONS

          UNI receive_msg;
              elapse_tick

        OUTPUT_INTERACTIONS

          UNI consume_msg;
              transmit_ack

      ELEM_TYPE Clock_Type(void)

        BEHAVIOR

          Clock(void; void) =
            <elapse_tick, inf> . Clock()

        INPUT_INTERACTIONS

          void

        OUTPUT_INTERACTIONS

          AND elapse_tick

    ARCHI_TOPOLOGY

      ARCHI_ELEM_INSTANCES

        MG : Msg_Gen_Type(msg_gen_rate);
        S  : Sender_Type(starting_bit,
                         timeout_period);
        LM : Line_Type(prop_delay_mean,
                       prop_delay_st_dev,
                       delivery_prob);
        LA : Line_Type(prop_delay_mean,
                       prop_delay_st_dev,
                       delivery_prob);
        R  : Receiver_Type(starting_bit);
        C  : Clock_Type()

      ARCHI_INTERACTIONS

        R.consume_msg

      ARCHI_ATTACHMENTS

        FROM MG.generate_msg TO S.get_msg;
        FROM S.transmit_msg  TO LM.receive;
        FROM LM.deliver      TO R.receive_msg;
        FROM R.transmit_ack  TO LA.receive;
        FROM LA.deliver      TO S.receive_ack;
        FROM C.elapse_tick   TO MG.elapse_tick;
        FROM C.elapse_tick   TO S.elapse_tick;
        FROM C.elapse_tick   TO LM.elapse_tick;
        FROM C.elapse_tick   TO LA.elapse_tick;
        FROM C.elapse_tick   TO R.elapse_tick

    END
	\end{verbatim}

\noindent Note that {\tt C.elapse\_tick} is an and-interaction, as it must simultaneously synchronize with
the {\tt elapse\_tick} interactions of all the other AEIs in order to mark the discrete time steps for the
whole system. Observe also that such an and-interaction is given the lowest priority, so that all the system
activities logically belonging to the same time step can be completed before the clock ticks.

Because of the presence of integer-, integer-based-, and list-typed variable formal parameters and local
variables, only the symbolic treatment of data values applies. Moreover, the symbolic performance semantic
model of {\tt abp\_gd.aem} cannot be generated as the specification is not performance closed, due to
non-determinism arising from the boolean guards within the alternative compositions that cannot be
statically evaluated. Here is the size of the other two semantic models of {\tt abp\_gd.aem}:

	\begin{verbatim}
    Size of the integrated semantic model underlying ABP_GD_Type:

    - 701 states: 
      - 0 tangible, 
      - 701 vanishing, 
      - 0 open, 
      - 0 deadlocked;

    - 2781 transitions:
      - 2781 observable,
      - 0 invisible;
      - 0 exponentially timed,
      - 2781 immediate,
      - 0 passive.

    Size of the functional semantic model underlying ABP_GD_Type:

    - 701 states: 
      - 701 nondeadlocked, 
      - 0 deadlocked;

    - 2781 transitions:
      - 2781 observable,
      - 0 invisible.
	\end{verbatim}



%%%%%%%%%%%%%%%%%%%%%%%%%%%%%%%%%%%%%%%%%%%%%%%%%%%%%%%%%%%%%%%%%
%                                                               %
%                                                               %
\section{Example B: The NRL Pump}
%                                                               %
%                                                               %
%%%%%%%%%%%%%%%%%%%%%%%%%%%%%%%%%%%%%%%%%%%%%%%%%%%%%%%%%%%%%%%%%

In this section we illustrate an extension of the alternating bit protocol used in a trusted device to
guarantee a suitable replication of information from low security users to high security users.


%%%%%%%%%%%%%%%%%%%%%%%%%%%%%%%%%%%%%%%%%%%%%%%%%%%%%%%%%%%%%%%%%
%                                                               %
\subsection{Informal Description}
%                                                               %
%%%%%%%%%%%%%%%%%%%%%%%%%%%%%%%%%%%%%%%%%%%%%%%%%%%%%%%%%%%%%%%%%

The NRL Pump~\cite{KMM} is a hardware device that interfaces a high security level LAN with a low security
level LAN. In essence, the pump places a buffer between the low LAN and the high LAN, pumps data from the
low LAN to the high LAN, and probabilistically modulates the timing of the acknowledgments from the high LAN
to the low LAN on the basis of the average transmission delay from the high LAN to the pump. The low level
and high level enclaves communicate with the pump through special interfacing software called wrappers,
which implement the pump protocol. Each wrapper is made of an application-dependent part, which supports the
set of functionalities that satisfy application-specific requirements, and a pump-dependent part, which is a
library of routines that implement the pump protocol. For security reasons, each process that uses the pump
must register its address with the pump administrator.

In order to establish a connection, the low LAN has to send a connection request message to the main thread
of the pump, which identifies the sending process and the address of the final destination. If both
addresses are valid (i.e., they have been previously registered), the main thread sends back a connection
valid message, otherwise it sends back a connection reject message. In the first case, the connection is
managed by a trusted low thread and a trusted high thread, which are created during the connection setup
phase to interact with the low LAN and the high LAN, respectively. Registered high processes are always
ready to accept a connection from the pump through the same handshake mechanism seen above. Once the
connection has been established, the pump sends a connection grant message to both LANs with initialization
parameters for the communication. During the connection, the trusted low thread receives data messages from
the low LAN, stores them in the connection buffer, and sends back the related acknowledgments by introducing
an additional stochastic delay computed on the basis of the average rate at which the trusted high thread
consumes messages. On the other hand, the trusted high thread delivers to the high LAN any data message
contained in the connection buffer, and the high LAN has to send back to the trusted high thread the related
acknowledgments before the expiration of a timeout. If the high LAN violates this protocol, the trusted high
thread aborts the connection. In such a case, as soon as the trusted low thread detects that the trusted
high thread is dead, it immediately sends to the low LAN all the remaining acknowledgments and a connection
exit message. Another special message is connection close, which is sent by the low LAN to the pump at the
end of a normal connection.


%%%%%%%%%%%%%%%%%%%%%%%%%%%%%%%%%%%%%%%%%%%%%%%%%%%%%%%%%%%%%%%%%
%                                                               %
\subsection{\aemilia\ Description}\label{nrl_pump}
%                                                               %
%%%%%%%%%%%%%%%%%%%%%%%%%%%%%%%%%%%%%%%%%%%%%%%%%%%%%%%%%%%%%%%%%

The following \aemilia\ specification called {\tt nrl\_pump.aem} describes the scenario illustrated above,
which comprises a low wrapper, a main thread, a trusted high thread, a trusted low thread, a connection
buffer, a high channel, and a high wrapper:

	\begin{verbatim}
    ARCHI_TYPE NRL_Pump_Type(const integer buffer_size     := 1,
                             const rate    conn_gen_rate   := 10,
                             const rate    conn_init_rate  := 62.5,
                             const rate    data_trans_rate := 125,
                             const rate    ack_trans_rate  := 1306.12,
                             const rate    ack_delay_rate  := 435.37,
                             const rate    timeout_rate    := 57.04,
                             const weight  valid_prob      := 0.99)

    ARCHI_ELEM_TYPES

      ELEM_TYPE LW_Type(const rate conn_gen_rate,
                        const rate data_trans_rate)

        BEHAVIOR

          LW_Beh(void; void) =
            <send_conn_request, exp(conn_gen_rate)>.
              choice
              {
                <receive_conn_valid, _> . <receive_conn_grant, _> .
                  <send_msg, exp(data_trans_rate)> . <receive_low_ack, _> .
                    choice
                    {
                      <receive_conn_exit, _> . LW_Beh(),
                      <send_conn_close, exp(data_trans_rate)> . LW_Beh()
                    },
                <receive_conn_reject, _> . LW_Beh()
              }

        INPUT_INTERACTIONS

          UNI receive_conn_valid;
              receive_conn_grant;
              receive_conn_reject;
              receive_low_ack;
              receive_conn_exit

        OUTPUT_INTERACTIONS

          UNI send_conn_request;
              send_msg;
              send_conn_close

      ELEM_TYPE MT_Type(const rate   data_trans_rate,
                        const weight valid_prob)

        BEHAVIOR

          MT_Beh(void; void) =
             <receive_conn_request, _> .
               choice
               {
                 <conn_is_valid, inf(1, valid_prob)> . <wakeup_tht, inf> .
                   <send_conn_valid, exp(data_trans_rate)> . MT_Beh(),
                 <conn_not_valid, inf(1, 1 - valid_prob)> .
                   <send_conn_reject, exp(data_trans_rate)> . MT_Beh()
               }

        INPUT_INTERACTIONS

          UNI receive_conn_request

        OUTPUT_INTERACTIONS

          UNI wakeup_tht;
              send_conn_valid;
              send_conn_reject

      ELEM_TYPE THT_Type(const rate conn_init_rate,
                         const rate timeout_rate)

        BEHAVIOR

          THT_Beh(void; void) =
            choice
            {
              <receive_high_wakeup, _> . <init_high_conn, exp(conn_init_rate)> .
                <wakeup_tlt, inf> . THT_Beh(),
              <read_msg, _> . <forward_msg, inf> .
                choice
                {
                  <receive_high_ack, _> . <delete_msg, inf> .
                    <send_ok_to_tlt, inf> . THT_Beh(),
                  <wait_for_timeout, exp(timeout_rate)> . <comm_timeout, inf> .
                    <delete_msg, inf> . <send_abort_to_tlt, inf> . THT_Beh()
                }
            }

        INPUT_INTERACTIONS

          UNI receive_high_wakeup;
              read_msg;
              receive_high_ack

        OUTPUT_INTERACTIONS

          UNI wakeup_tlt;
              forward_msg;
              delete_msg;
              send_ok_to_tlt;
              comm_timeout;
              send_abort_to_tlt

      ELEM_TYPE TLT_Type(const rate data_trans_rate,
                         const rate ack_trans_rate,
                         const rate ack_delay_rate)

        BEHAVIOR

          TLT_Beh(void; void) =
            <receive_low_wakeup, _> . <send_conn_grant, exp(data_trans_rate)> .
              <receive_msg, _> . <store_msg, inf> .
                choice
                {
                  <wait_delay, exp(ack_delay_rate)> . <send_low_ack, exp(ack_trans_rate)> .
                    choice
                    {
                      <receive_abort_from_tht, _> . <send_conn_exit, exp(data_trans_rate)> .
                        TLT_Beh(),
                      <receive_ok_from_tht, _> . <receive_conn_close, _> . TLT_Beh()
                    },
                  <receive_abort_from_tht, _> . <send_low_ack, exp(ack_trans_rate)> .
                    <send_conn_exit, exp(data_trans_rate)> . TLT_Beh(),
                  <receive_ok_from_tht, _> . <wait_delay, exp(ack_delay_rate)> .
                    <send_low_ack, exp(ack_trans_rate)> . <receive_conn_close, _> . TLT_Beh()
                }

        INPUT_INTERACTIONS

          UNI receive_low_wakeup;
              receive_msg;
              receive_abort_from_tht;
              receive_ok_from_tht;
              receive_conn_close

        OUTPUT_INTERACTIONS

          UNI send_conn_grant;
              store_msg;
              send_low_ack;
              send_conn_exit

      ELEM_TYPE Buffer_Type(const integer buffer_size)

        BEHAVIOR

          Buffer_Beh(integer(0..buffer_size) msg_num := 0;
                     void) =
            choice
            {
              cond(msg_num < buffer_size) ->
                <accept_msg, _> . Buffer_Beh(msg_num + 1),
              cond(msg_num > 0) ->
                choice
                {
                  <read_msg, inf> . Buffer_Beh(msg_num),
                  <delete_msg, _> . Buffer_Beh(msg_num - 1)
                }
            }

        INPUT_INTERACTIONS

          UNI accept_msg;
              delete_msg

        OUTPUT_INTERACTIONS

          UNI read_msg

      ELEM_TYPE HC_Type(const rate data_trans_rate,
                        const rate ack_trans_rate)

        BEHAVIOR

          HC_Beh(void; void) =
            <accept_msg, _> .
              choice
              {
                <receive_timeout, _> . HC_Beh(),
                <transmit_msg, exp(data_trans_rate)> . <accept_high_ack, _> .
                  choice
                  {
                    <receive_timeout, _> . HC_Beh(),
                    <transmit_high_ack, exp(ack_trans_rate)> . HC_Beh()
                  }
              }

        INPUT_INTERACTIONS

          UNI accept_msg;
              receive_timeout;
              accept_high_ack

        OUTPUT_INTERACTIONS

          UNI transmit_msg;
              transmit_high_ack

      ELEM_TYPE HW_Type(void)

        BEHAVIOR

          HW_Beh(void; void) =
            <receive_msg, _> . <send_high_ack, inf> . HW_Beh()

        INPUT_INTERACTIONS

          UNI receive_msg

        OUTPUT_INTERACTIONS

          UNI send_high_ack

    ARCHI_TOPOLOGY

      ARCHI_ELEM_INSTANCES

        LW  : LW_Type(conn_gen_rate,
                      data_trans_rate);
        MT  : MT_Type(data_trans_rate,
                      valid_prob);
        THT : THT_Type(conn_init_rate,
                       timeout_rate);
        TLT : TLT_Type(data_trans_rate,
                       ack_trans_rate,
                       ack_delay_rate);
        B   : Buffer_Type(buffer_size);
        HC  : HC_Type(data_trans_rate,
                      ack_trans_rate);
        HW  : HW_Type()

      ARCHI_INTERACTIONS

        void

      ARCHI_ATTACHMENTS

        FROM LW.send_conn_request  TO MT.receive_conn_request;
        FROM MT.send_conn_valid    TO LW.receive_conn_valid;
        FROM MT.send_conn_reject   TO LW.receive_conn_reject;
        FROM MT.wakeup_tht         TO THT.receive_high_wakeup;
        FROM THT.wakeup_tlt        TO TLT.receive_low_wakeup;
        FROM TLT.send_conn_grant   TO LW.receive_conn_grant;
        FROM LW.send_msg           TO TLT.receive_msg;
        FROM TLT.store_msg         TO B.accept_msg;
        FROM TLT.send_low_ack      TO LW.receive_low_ack;
        FROM B.read_msg            TO THT.read_msg;
        FROM THT.forward_msg       TO HC.accept_msg;
        FROM HC.transmit_msg       TO HW.receive_msg;
        FROM THT.comm_timeout      TO HC.receive_timeout;
        FROM HW.send_high_ack      TO HC.accept_high_ack;
        FROM HC.transmit_high_ack  TO THT.receive_high_ack;
        FROM THT.delete_msg        TO B.delete_msg;
        FROM THT.send_abort_to_tlt TO TLT.receive_abort_from_tht;
        FROM THT.send_ok_to_tlt    TO TLT.receive_ok_from_tht;
        FROM TLT.send_conn_exit    TO LW.receive_conn_exit;
        FROM LW.send_conn_close    TO TLT.receive_conn_close

    END
	\end{verbatim}

The formal parameters of the \aemilia\ specification represent the connection buffer capacity, the
connection generation rate (one connection every 100~msec), the connection initialization rate (corresponds
to the round-trip delay), the data and acknowledgment transmission rates, the acknowledgment delay rate
(corresponds to the delay for transmitting 3~acknowledgments), the timeout rate (corresponds to twice the
delay for transmitting a data message and the related acknowledgment), and the probability that a connection
request is valid, respectively. The values of the rates are a consequence of the assumption that the channel
capacity is 64~Kbps, the data message length is 512~bits, and the acknowledgment length is 49~bits.

{\tt LW\_Type} represents the wrapper at the sending site, which periodically generates a connection
request. After sending the request, {\tt LW\_Type} is ready to accept either a connection reject message or
a connection valid message followed by a connection grant message. If a connection is established,
{\tt LW\_Type} sends a data message to the trusted low thread and then waits for the related acknowledgment.
Afterwards, {\tt LW\_Type} can either receive a connection exit message in the case that the connection is
aborted by the high LAN, or send a connection close message in the case that the connection is correctly
terminated.

{\tt MT\_Type} listens to the port of the NRL Pump to which the low LAN sends connection request messages.
We abstract away from the verification of the validity of an incoming request by assuming to know the
probability of receiving a valid request. If a valid connection must be established, the main thread awakens
the trusted high thread and sends a connection valid message to the low LAN, otherwise it sends a connection
reject message to the low LAN.

{\tt THT\_Type}, which is spawned by the main thread during the setup phase, initializes the connection
towards the high LAN. For the sake of simplicity, we assume that the high wrapper cannot refuse the
connection request and we abstract away from the handshaking with the high wrapper through a single
exponentially timed action. Afterwards, {\tt THT\_Type} awakens the trusted low thread. When the connection
becomes active, {\tt THT\_Type} checks the buffer for new incoming data messages. Upon reading a message
from the buffer, {\tt THT\_Type} outputs it to the high communication channel. An immediate action is used
to express the synchronization between {\tt THT\_Type} and the high communication channel, which in turn
explicitly models the message transmission delay. Then, {\tt THT\_Type} waits for the reception of the
related acknowledgment from the high LAN. The arrival of an acknowledgment competes with the timeout fixed
by {\tt THT\_Type}. In particular, if the acknowledgment is received before the end of the timeout,
{\tt THT\_Type} removes the message from the buffer and informs the trusted low thread about the successful
transmission. On the other hand, if the timeout expires before the reception of the acknowledgment, the
trusted high thread notifies the timeout expiration, removes the message from the buffer, and informs the
trusted low thread about the aborted connection.

{\tt TLT\_Type} waits for the trusted high thread to awaken it, then establishes the connection from the low
LAN to the pump by sending a connection grant message to the low LAN. At that moment, {\tt TLT\_Type} is
ready to receive a data message from the low LAN. Upon receiving a data message, {\tt TLT\_Type} stores it
in the connection buffer, then delays the transmission of the acknowledgment to the low LAN through an
exponentially timed action. After the expiration of such a delay, {\tt TLT\_Type} sends the acknowledgment
to the low LAN. At any moment, {\tt TLT\_Type} may receive a message from the trusted high thread concerning
the status of the connection. In particular, in the case of trusted high thread failure, {\tt TLT\_Type}
must send a connection exit message to the low LAN. On the contrary, if the trusted high thread is correctly
working, {\tt TLT\_Type} can accept a connection close message from the low LAN. If {\tt TLT\_Type} detects
the trusted high thread failure before sending the acknowledgment to the low LAN, then {\tt TLT\_Type}
immediately transmits the acknowledgment and the connection exit message to the low LAN.

{\tt Buffer\_Type}, which is initially empty, can be accessed by the trusted low thread and the trusted high
thread only. When {\tt Buffer\_Type} is not full, a new data message can be accepted from the trusted low
thread. When {\tt Buffer\_Type} is not empty, a data message can be read or deleted by the trusted high
thread.

{\tt HC\_Type} models the communications between the trusted high thread and the high LAN. We need an
explicit component to express the transmission delay of messages along that link, because the round-trip
delay of a communication between the trusted high thread and the high LAN must compete with the timeout set
by the trusted high thread. Initially, {\tt HC\_Type} is ready to accept a data message from the trusted
high thread. An incoming message is transmitted to the high LAN according to an exponentially distributed
delay. After the delivery of the message, {\tt HC\_Type} waits for the related acknowledgment to be
transmitted back to the trusted high thread. {\tt HC\_Type} is always willing to accept a timeout
notification from the trusted high thread, in which case the connection will be aborted and all the pending
messages will be dropped.

Finally, {\tt HW\_Type} represents the wrapper at the receving site, which accepts data messages from the
high channel and sends back the related acknowledgments.

We conclude by reporting the size of the semantic models of {\tt nrl\_pump.aem}:

	\begin{verbatim}
    Size of the integrated semantic model underlying NRL_Pump_Type:

    - 46 states: 
      - 20 tangible, 
      - 26 vanishing, 
      - 0 open, 
      - 0 deadlocked;

    - 58 transitions:
      - 58 observable,
      - 0 invisible;
      - 31 exponentially timed,
      - 27 immediate,
      - 0 passive.

    Size of the functional semantic model underlying NRL_Pump_Type:

    - 46 states: 
      - 46 nondeadlocked, 
      - 0 deadlocked;

    - 58 transitions:
      - 58 observable,
      - 0 invisible.

    Size of the homogeneous continuous-time Markov chain underlying NRL_Pump_Type:

    - 20 states: 
      - 20 nonabsorbing, 
      - 0 absorbing;

    - 32 transitions.
	\end{verbatim}



%%%%%%%%%%%%%%%%%%%%%%%%%%%%%%%%%%%%%%%%%%%%%%%%%%%%%%%%%%%%%%%%%
%                                                               %
%                                                               %
\section{Example C: Dining Philosophers}
%                                                               %
%                                                               %
%%%%%%%%%%%%%%%%%%%%%%%%%%%%%%%%%%%%%%%%%%%%%%%%%%%%%%%%%%%%%%%%%

In this section we present a simple randomized distributed algorithm for the solution of a classical mutual
exclusion problem, which illustrates the declaration of a parameterized architectural topology.


%%%%%%%%%%%%%%%%%%%%%%%%%%%%%%%%%%%%%%%%%%%%%%%%%%%%%%%%%%%%%%%%%
%                                                               %
\subsection{Informal Description}
%                                                               %
%%%%%%%%%%%%%%%%%%%%%%%%%%%%%%%%%%%%%%%%%%%%%%%%%%%%%%%%%%%%%%%%%

Suppose there are $n$~philosophers sitting at a round table, each with a plate in front of him/her, and
$n$~chopsticks on the table, each shared by two neighbor philosophers. Every philosopher alternately thinks
and eats. In order to get the rice at the center of the table, a philosopher needs both the chopstick on
his/her right and the chopstick on his/her left. The problem is that of defining a set of rules ensuring
deadlock freedom, which the philosophers should follow whenever they are hungry in order to get the
chopsticks they share with their neighbors.

The Lehmann-Rabin algorithm~\cite{LR} provides a symmetric solution to the problem, in the sense that all
the philosophers behave according to the same protocol. Whenever a philosopher is hungry, he/she flips a
fair coin to decide which chopstick to pick up first, waits for that chopstick to become free, gets it, then
tries to get the other chopstick. It is free, then the philosopher picks it up and starts eating, otherwise
the philosopher puts down the first chopstick and restarts the whole process.


%%%%%%%%%%%%%%%%%%%%%%%%%%%%%%%%%%%%%%%%%%%%%%%%%%%%%%%%%%%%%%%%%
%                                                               %
\subsection{\aemilia\ Description}\label{dp}
%                                                               %
%%%%%%%%%%%%%%%%%%%%%%%%%%%%%%%%%%%%%%%%%%%%%%%%%%%%%%%%%%%%%%%%%

The following \aemilia\ specification called {\tt dining\_philosophers.aem} describes the Lehmann-Rabin
algorithm, assuming an exponential timing for the two activities carried out by every philosopher:

	\begin{verbatim}
    ARCHI_TYPE LR_Dining_Philosophers_Type(const integer philosopher_num := 3,
                                           const rate    think_rate      := 4.50,
                                           const rate    eat_rate        := 0.75)

    ARCHI_ELEM_TYPES

      ELEM_TYPE Philosopher_Type(const rate think_rate,
                                 const rate eat_rate)

        BEHAVIOR

          Philosopher_Thinking(void; void) =
            <think, exp(think_rate)> . Philosopher_Picking();

          Philosopher_Picking(void; void) =
            choice
            {
              <flip_head, inf(1, 0.5)> . <pick_up_right_first, inf> .
                choice
                {
                  <pick_up_left_then, inf(2, 1)> . Philosopher_Eating(),
                  <put_down_right, inf> . Philosopher_Picking()
                },
              <flip_tail, inf(1, 0.5)> . <pick_up_left_first, inf> .
                choice
                {
                  <pick_up_right_then, inf(2, 1)> . Philosopher_Eating(),
                  <put_down_left, inf> . Philosopher_Picking()
                }
            };

          Philosopher_Eating(void; void) =
            <eat, exp(eat_rate)> . <put_down_right, inf> . <put_down_left, inf> .
              Philosopher_Thinking()

        INPUT_INTERACTIONS

          void

        OUTPUT_INTERACTIONS

          UNI pick_up_right_first;
              pick_up_right_then;
              put_down_right;
              pick_up_left_first;
              pick_up_left_then;
              put_down_left

      ELEM_TYPE Chopstick_Type(void)

        BEHAVIOR

          Chopstick_Picking(void; void) =
            choice
            {
              <pick_up_first, _> . Chopstick_Putting(),
              <pick_up_then, _> . Chopstick_Putting()
            };

          Chopstick_Putting(void; void) =
            <put_down, _> . Chopstick_Picking()

        INPUT_INTERACTIONS

          OR pick_up_first;
             pick_up_then;
             put_down

        OUTPUT_INTERACTIONS

          void

    ARCHI_TOPOLOGY

      ARCHI_ELEM_INSTANCES

        FOR_ALL i IN 0..philosopher_num - 1
          P[i] : Philosopher_Type(think_rate,
                                  eat_rate);
        FOR_ALL i IN 0..philosopher_num - 1
          C[i] : Chopstick_Type()

      ARCHI_INTERACTIONS

        void

      ARCHI_ATTACHMENTS

        FOR_ALL i IN 0..philosopher_num - 1
          FROM P[i].pick_up_right_first TO C[i].pick_up_first;
        FOR_ALL i IN 0..philosopher_num - 1
          FROM P[i].pick_up_right_then TO C[i].pick_up_then;
        FOR_ALL i IN 0..philosopher_num - 1
          FROM P[i].put_down_right TO C[i].put_down;
        FOR_ALL i IN 0..philosopher_num - 1
          FROM P[i].pick_up_left_first TO C[mod(i + 1, philosopher_num)].pick_up_first;
        FOR_ALL i IN 0..philosopher_num - 1
          FROM P[i].pick_up_left_then TO C[mod(i + 1, philosopher_num)].pick_up_then;
        FOR_ALL i IN 0..philosopher_num - 1
          FROM P[i].put_down_left TO C[mod(i + 1, philosopher_num)].put_down

    END
	\end{verbatim}

As far as the behavior is concerned, whenever both {\tt pick\_up\_left\_then} (resp.\
{\tt pick\_up\_right\_then}) and {\tt put\_down\_right} (resp.\ {\tt put\_down\_left}) are enabled, then the
first one is executed as it has higher priority. We also observe that all the input interactions of the
chopsticks are or-interactions, because each chopstick is shared by two neighbor philosophers, but only one
of them at a time can use the chopstick.

On the topology side, we note that the indexing mechanism for the declaration of AEIs and attachments
provides a concise and parameterized way to describe the application of the algorithm to an arbitrary number
of philosophers.

Here is the size of the semantic models of {\tt dining\_philosophers.aem} for $n = 3$:

	\begin{verbatim}
    Size of the integrated semantic model underlying LR_Dining_Philosophers_Type:

    - 109 states: 
      - 13 tangible, 
      - 96 vanishing, 
      - 0 open, 
      - 0 deadlocked;

    - 147 transitions:
      - 147 observable,
      - 0 invisible;
      - 27 exponentially timed,
      - 120 immediate,
      - 0 passive.

    Size of the functional semantic model underlying LR_Dining_Philosophers_Type:

    - 109 states: 
      - 109 nondeadlocked, 
      - 0 deadlocked;

    - 147 transitions:
      - 147 observable,
      - 0 invisible.

    Size of the homogeneous continuous-time Markov chain underlying LR_Dining_Philosophers_Type:

    - 13 states: 
      - 13 nonabsorbing, 
      - 0 absorbing;

    - 30 transitions.
	\end{verbatim}

\noindent The size of the state space grows by a multiplicative factor between three and four with respect
to the number $n$ of philosophers. As an example, for $n = 10$ the integrated semantic model of
{\tt dining\_philosophers.aem} has 175887~states and 282530~transitions.
