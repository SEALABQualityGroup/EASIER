%%%%%%%%%%%%%%%%%%%%%%%%%%%%%%%%%%%%%%%%%%%%%%%%%%%%%%%%%%%%%%%%%
%                                                               %
%                                                               %
%                                                               %
\pagenumbering{arabic}
\chapter{Tool Description}
%                                                               %
%                                                               %
%                                                               %
%%%%%%%%%%%%%%%%%%%%%%%%%%%%%%%%%%%%%%%%%%%%%%%%%%%%%%%%%%%%%%%%%


%%%%%%%%%%%%%%%%%%%%%%%%%%%%%%%%%%%%%%%%%%%%%%%%%%%%%%%%%%%%%%%%%
%                                                               %
%                                                               %
\section{What TwoTowers~5.1 Is}
%                                                               %
%                                                               %
%%%%%%%%%%%%%%%%%%%%%%%%%%%%%%%%%%%%%%%%%%%%%%%%%%%%%%%%%%%%%%%%%

TwoTowers~5.1 is an open-source software tool for the functional verification, security analysis, and
performance evaluation of computer, communication and software systems modeled in the architectural
description language \aemilia~\cite{BCD,AB,BDC,BBS}, which is based on the stochastic process algebra
\empagr~\cite{Ber1,BB2,BB1}.

The study of the properties of the \aemilia\ specifications is conducted in TwoTowers~5.1 through 
equivalence verification with diagnostics~\cite{CS}, symbolic model checking with diagnostics~\cite{CGP} via
NuSMV~2.2.5~\cite{CCOPR}, information flow analysis with diagnostics~\cite{FG}, reward Markov chain
solution~\cite{Ste1,How}, and discrete event simulation~\cite{Wel}.



%%%%%%%%%%%%%%%%%%%%%%%%%%%%%%%%%%%%%%%%%%%%%%%%%%%%%%%%%%%%%%%%%
%                                                               %
%                                                               %
\section{Architecture of TwoTowers~5.1}
%                                                               %
%                                                               %
%%%%%%%%%%%%%%%%%%%%%%%%%%%%%%%%%%%%%%%%%%%%%%%%%%%%%%%%%%%%%%%%%

TwoTowers~5.1 is composed of about 45,000 lines of ANSI~C~\cite{KR} code organized as depicted in
Fig.~\ref{archi}.

TwoTowers~5.1 is equipped with a simple graphical user interface written in Tcl/Tk~\cite{Ous} through which
the user can invoke the analysis routines by means of suitable menus. Each routine needs input files of
certain types and writes its results onto files of other types. The graphical user interface takes care of
the integrated management of the various file types needed by the different routines, which belong to the
\aemilia\ compiler, the equivalence verifier, the model checker, the security analyzer, and the performance
evaluator.

The compiler is in charge of parsing \aemilia\ specifications stored in {\tt .aem} files and signalling
possible lexical, syntax and static semantic errors through a {\tt .lis} file. Based on the translation
semantics for \aemilia\ into \empagr\ and the operational semantics for \empagr, if an \aemilia\
specification is correct the compiler can generate its integrated, functional or performance semantic model,
which is written to a {\tt .ism}, {\tt .fsm} or {\tt .psm} file, respectively. As a faster option that does
not require printing the state space onto a file, the compiler can show only the size -- in terms of number
of states and transitions -- of the semantic model, which is written to a {\tt .siz} file. The integrated
semantic model of an \aemilia\ specification for a given system is a state transition graph whose
transitions are labeled with the name and the duration/priority/probability of the corresponding system
activities. The functional semantic model is a state transition graph in which only the activity names label
the transitions. The performance semantic model, which can be extracted only if the \aemilia\ specification
is performance closed, is a continuous- or discrete-time Markov chain~\cite{Ste1}.

The equivalence verifier checks through the application of the Kanellakis-Smolka algorithm~\cite{KS}
whether two correct, finite-state \aemilia\ specifications are equivalent according to one of four different
behavioral equivalences: strong bisimulation equivalence, weak bisimulation equivalence, strong (extended)
Markovian bisimulation equivalence, and weak (extended) Markovian bisimulation
equivalence~\cite{Mil,BB2,Ber3}. The result of the verification is written to a {\tt .evr} file. In the case
of non-equivalence a distinguishing modal logic formula is reported as well, which is computed on the basis
of the algorithm of~\cite{Cle} and is expressed in a verbose variant of the Hennessy-Milner logic~\cite{HM}
or one of its probabilistic extensions~\cite{LS,CGH}.

The equivalence verifier allows a comparative study of two \aemilia\ specifications to be conducted, aiming
at establishing whether they possess the same functional, security and performance properties in general.
Should the two \aemilia\ specifications be equivalent, in order to know whether they satisfy a particular
functional property, security requirement, or performance guarantee, it is necessary to apply to one of the
two \aemilia\ specifications the model checker, the security analyzer, or the performance evaluator,
respectively.

	\begin{figure}[thb]

\centerline{\epsfbox{twotowers.eps}}
\caption{Tool architecture}\label{archi}

	\end{figure}

The model checker verifies through the BDD-based routines of NuSMV~2.2.5~\cite{CCOPR} whether a set of
functional properties expressed through verbose LTL formulas~\cite{CGP}, which are stored in a {\tt .ltl}
file, are satisfied by a correct, finite-state \aemilia\ specification. The result of the check, together
with a counterexample for each property that is not met, is written to a {\tt .mcr} file.

The security analyzer checks through the equivalence verifier whether a correct, finite-state \aemilia\
specification satisfies non-interference or non-deducibility on composition~\cite{FG}, both of which
establish the absence of illegal information flows from high security system components to low security
system components. This requires the classification in an additional {\tt .sec} file of the system
activities that are high and low with respect to the security level. The result of the analysis is written
to a {\tt .sar} file, together with a modal logic formula expressed in a verbose variant of the
Hennessy-Milner logic to explain a possible security violation.

Finally, the performance evaluator assesses the quantitative characteristics of correct, finite-state and
performance closed \aemilia\ specifications. First, it can calculate the stationary/transient probability
distribution for the state space of the performance semantic model of an \aemilia\ specification. The
distribution is written to a {\tt .dis} file. Second, the performance evaluator can calculate for an
\aemilia\ specification a set of instant-of-time, stationary/transient performance measures specified
through state and transition rewards~\cite{How} stored in a {\tt .rew} file. The values of the measures are
written to a {\tt .val} file. The solution methods implemented for the stationary case are Gaussian
elimination and an adaptive variant of symmetric stochastic over-relaxation, while for the transient case
uniformization is available~\cite{Ste1}. Third, the performance evaluator can estimate via discrete event
simulation the mean, variance or distribution of a set of performance measures specified through an
extension of state and transition rewards, which are stored in a {\tt .sim} file together with the number
and the length of the simulation runs. The simulation, which can be applied also to \aemilia\ specifications
with infinitely many states and general distributions, is based on the method of independent replications
with 90\% confidence level~\cite{Wel} and can be trace driven, in which case the traces are stored in
{\tt .trc} files. The result of the simulation is written to a {\tt .est} file.



%%%%%%%%%%%%%%%%%%%%%%%%%%%%%%%%%%%%%%%%%%%%%%%%%%%%%%%%%%%%%%%%%
%                                                               %
%                                                               %
\section{What TwoTowers~5.1 Offers}
%                                                               %
%                                                               %
%%%%%%%%%%%%%%%%%%%%%%%%%%%%%%%%%%%%%%%%%%%%%%%%%%%%%%%%%%%%%%%%%

From the user viewpoint, TwoTowers~5.1 supplies the following capabilities:

	\begin{itemize}

\item Component-oriented modeling with \aemilia:

		\begin{itemize}

\item Separation of system component behavior specification from system topology specification.

\item Support for the parameterization of system component behavior.

\item Indexing mechanism for defining parameterized system topologies.

\item Several data types: integer, bounded integer, real, boolean, list, array, and record.

\item Representation of continuous- and discrete-time systems.

\item Component activities with exponentially distributed, zero or unspecified durations.

\item Random number generators for sampling durations/probabilities from other distributions~\cite{Jai}.

\item Prioritized and probabilistic choices among component activities.

\item Generative-reactive synchronizations among component activities~\cite{BB2}.

\item Value passing across components (compiled concretely -- if possible -- or symbolically~\cite{Ber2}).

		\end{itemize}

\item Companion languages for the parameterized specification of:

		\begin{itemize}

\item Functional properties expressed in a verbose variant of LTL.

\item Security levels.

\item Performance measures based on state and transition rewards~\cite{BB1}.

\item Simulation experiments.

		\end{itemize}

\item Parsing of \aemilia\ and companion specifications with the generation of listings that pinpoint
lexical, syntactical and static semantical errors.

\item Compilation of \aemilia\ specifications into state transition graphs that are shown in a readable
format, through the indication for each global state of the constituting local states of the components.

\item Integrated framework for the study of functional, security and performance properties of \aemilia\
specifications and the provision of diagnostic information in the case of negative outcome.

	\end{itemize}



%%%%%%%%%%%%%%%%%%%%%%%%%%%%%%%%%%%%%%%%%%%%%%%%%%%%%%%%%%%%%%%%%
%                                                               %
%                                                               %
\section{Case Studies}
%                                                               %
%                                                               %
%%%%%%%%%%%%%%%%%%%%%%%%%%%%%%%%%%%%%%%%%%%%%%%%%%%%%%%%%%%%%%%%%

A complete and updated list of the case studies conducted with TwoTowers~5.1 (or earlier versions) is
maintained at \verb+http://www.sti.uniurb.it/bernardo/twotowers/+, where their \aemilia\ and companion
specifications as well as the related papers can be found.



%%%%%%%%%%%%%%%%%%%%%%%%%%%%%%%%%%%%%%%%%%%%%%%%%%%%%%%%%%%%%%%%%
%                                                               %
%                                                               %
\section{History of TwoTowers}
%                                                               %
%                                                               %
%%%%%%%%%%%%%%%%%%%%%%%%%%%%%%%%%%%%%%%%%%%%%%%%%%%%%%%%%%%%%%%%%

The development of TwoTowers started in~1996, then restarted from scratch in~1997. Its name stems from the
two medieval towers -- Asinelli (97~mt.) and Garisenda (48~mt.) -- that are the symbol of Bologna, the city
where the implementation of the tool started.

Version~1.0 was distributed in July~2001 for Linux and Unix operating systems only. Its input language was
\empagr, enriched with the symbolically treated data types integer, real, boolean, list, array, and record.
Its graphical user interface contained a menu for a parser and a compiler of \empagr\ specifications into
state transition graphs, an integrated analyzer for equivalence checking (strong Markovian bisimulation
equivalence), a functional analyzer based on CWB-NC~1.2~\cite{CS} for model checking ($\mu$-calculus and
GCTL*~\cite{CGP}) and equivalence/preorder checking with diagnostics (strong/weak bisimulation equivalence,
may/must testing equivalence and preorder~\cite{DH}), and a performance analyzer in charge of reward-based
Markovian analysis through the solution methods of the commercial tool MarCA~3.0~\cite{Ste2} as well as
discrete event simulation.

Version~2.0 was distributed in November~2002. As a faster compilation option not requiring any possibly huge
file to be written, it provided the capability to report only the size of the semantic models underlying an
\empagr\ specification. Unlike the previous version, it no longer relied on MarCA~3.0, as it had three
built-in analysis routines -- Gaussian elimination, an adaptive variant of symmetric stochastic
over-relaxation, and uniformization -- for the solution of reward Markov chains of arbitrary size. This
allowed TwoTowers to be distributed free of charge to educational and non-profit organizations.

Version~3.0 was completed in October~2003 but not distributed. In this version \empagr\ was replaced by
\aemilia, thus adopting a component-oriented modeling style leading to more confidence in the correctness of
the system specifications as well as a higher degree of parameterization and reuse. Also the companion
languages for the specification of functional properties and performance measures were modified according to
the adopted component-oriented style and became more verbose, thus increasing their ease of use. The
component-orientation reflected on a more readable state representation, as each global state could be
described through its constituting local states corresponding to the components. On the data type side,
bounded integers were introduced and the concrete treatment of data values of type different from integer,
real, and list was implemented in addition to the original symbolic treatment.

Version~4.0 was completed in January~2004 but not distributed. Its graphical user interface was reorganized
in order to contain a menu for equivalence verification with diagnostics fully based on built-in routines
(strong/weak functional/Markovian bisimulation equivalence), a menu for model checking (based on CWB-NC~1.2
as before), a novel menu for security analysis, and a menu for performance evaluation (as before). The
routines for security analysis required a companion language for the specification of the security levels of
the component activities, and employed weak bisimulation equivalence checking to assess non-interference and
non-deducibility on composition.

Version~5.0 was distributed in May~2004. In this version a symbolic model checker was adopted by replacing
the one of CWB-NC~1.2 with NuSMV~2.1.2. Consequently, the companion language for the specification of
functional properties was modified to express LTL formulas instead of $\mu$-calculus and GCTL* formulas.

Version~5.1 is being distributed since January~2006. Some minor modifications were done in order to make the
tool available for the Windows operating system as well. Moreover the use of NuSMV~2.1.2 was replaced by the
use of NuSMV~2.2.5.



%%%%%%%%%%%%%%%%%%%%%%%%%%%%%%%%%%%%%%%%%%%%%%%%%%%%%%%%%%%%%%%%%
%                                                               %
%                                                               %
\section{Acknowledgments}
%                                                               %
%                                                               %
%%%%%%%%%%%%%%%%%%%%%%%%%%%%%%%%%%%%%%%%%%%%%%%%%%%%%%%%%%%%%%%%%

The following people worked with Marco Bernardo for the definition of \aemilia\ and \empagr, the development
of the case studies, and the integration of TwoTowers with other tools: Pietro Abate, Andrea Acquaviva,
Alessandro Aldini, Simonetta Balsamo, Alessandro Bogliolo, Edoardo Bont\`a, Mario Bravetti, Nadia Busi,
Paolo Ciancarini, Alessandro Cimatti, Rance Cleaveland, Marcello Colucci, Lorenzo Donatiello, Francesco
Franz\`e, Roberto Gorrieri, Emanuele Lattanzi, Simone Mecozzi, Claudio Premici, Marina Ribaudo, Marco
Roccetti, Marta Simeoni, Steve Sims, and Billy Stewart.
