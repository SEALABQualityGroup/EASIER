%%%%%%%%%%%%%%%%%%%%%%%%%%%%%%%%%%%%%%%%%%%%%%%%%%%%%%%%%%%%%%%%%
%                                                               %
%                                                               %
%                                                               %
\chapter{Tool Installation and Execution}
%                                                               %
%                                                               %
%                                                               %
%%%%%%%%%%%%%%%%%%%%%%%%%%%%%%%%%%%%%%%%%%%%%%%%%%%%%%%%%%%%%%%%%


%%%%%%%%%%%%%%%%%%%%%%%%%%%%%%%%%%%%%%%%%%%%%%%%%%%%%%%%%%%%%%%%%
%                                                               %
%                                                               %
\section{Introduction}
%                                                               %
%                                                               %
%%%%%%%%%%%%%%%%%%%%%%%%%%%%%%%%%%%%%%%%%%%%%%%%%%%%%%%%%%%%%%%%%

In this chapter we explain how to install and run TwoTowers~5.1 on a computer with the Linux or Windows
operating system.



%%%%%%%%%%%%%%%%%%%%%%%%%%%%%%%%%%%%%%%%%%%%%%%%%%%%%%%%%%%%%%%%%
%                                                               %
%                                                               %
\section{Source Distribution}
%                                                               %
%                                                               %
%%%%%%%%%%%%%%%%%%%%%%%%%%%%%%%%%%%%%%%%%%%%%%%%%%%%%%%%%%%%%%%%%

TwoTowers~5.1 is distributed through the compressed file {\tt TwoTowers.tar.gz}. After moving this
compressed file into a new directory, the source files can be extracted together with the related
documentation and utilities through the following two commands (symbol {\tt >} denotes the prompt of the
operating system shell):

	\begin{verbatim}
  > gunzip TwoTowers.tar.gz
  > tar -xvf TwoTowers.tar
	\end{verbatim}

\noindent which should result in the following directory structure:

	\begin{verbatim}
  . bin
  |_. TTKernel.exe
  . docs
  |_. license.txt
    . manual.pdf
    . readme.txt
  . gui
  |_. TTGUI
  . src
  |_. Makefile
    . compiler
    |_. Makefile
      . aemilia_compiler.c
      . aemilia_parser.y
      . aemilia_scanner.l
      . listing_generator.c
      . ltl_parser.y
      . ltl_scanner.l
      . rew_parser.y
      . rew_scanner.l
      . sec_parser.y
      . sec_scanner.l
      . sim_parser.y
      . sim_scanner.l
      . symbol_handler.c
    . driver
    |_. Makefile
      . driver.c
    . equivalence_verifier
    |_. Makefile
      . equivalence_verifier.c
    . headers
    |_. Library.h
      . aemilia_compiler.h
      . aemilia_parser.h
      . aemilia_scanner.h
      . driver.h
      . equivalence_verifier.h
      . file_handler.h
      . list_handler.h
      . listing_generator.h
      . ltl_parser.h
      . ltl_scanner.h
      . markov_solver.h
      . memory_handler.h
      . number_handler.h
      . nusmv_connector.h
      . rew_parser.h
      . rew_scanner.h
      . sec_parser.h
      . sec_scanner.h
      . security_analyzer.h
      . sim_parser.h
      . sim_scanner.h
      . simulator.h
      . string_handler.h
      . symbol_handler.h
      . table_handler.h
    . model_checker
    |_. Makefile
      . nusmv_connector.c
    . performance_evaluator
    |_. Makefile
      . markov_solver.c
      . simulator.c
    . security_analyzer
    |_. Makefile
      . security_analyzer.c
    . utilities
    |_. Makefile
      . file_handler.c
      . list_handler.c
      . memory_handler.c
      . number_handler.c
      . string_handler.c
      . table_handler.c
  . win_utils
  |_. cp.bat
    . mv.bat
    . rm.bat
    . tt_compile.bat
    . tt_exec.bat
	\end{verbatim}



%%%%%%%%%%%%%%%%%%%%%%%%%%%%%%%%%%%%%%%%%%%%%%%%%%%%%%%%%%%%%%%%%
%                                                               %
%                                                               %
\section{Installation Procedure}
%                                                               %
%                                                               %
%%%%%%%%%%%%%%%%%%%%%%%%%%%%%%%%%%%%%%%%%%%%%%%%%%%%%%%%%%%%%%%%%

The procedure for installing TwoTowers~5.1 comprises a couple of quick and easy steps.


%%%%%%%%%%%%%%%%%%%%%%%%%%%%%%%%%%%%%%%%%%%%%%%%%%%%%%%%%%%%%%%%%
%                                                               %
\subsection{Linux}
%                                                               %
%%%%%%%%%%%%%%%%%%%%%%%%%%%%%%%%%%%%%%%%%%%%%%%%%%%%%%%%%%%%%%%%%

On a Linux machine, make sure that the following packages are available:

	\begin{verbatim}
  flex     (lexical analyzer generator,
            http://www.gnu.org/software/flex/flex.html)
  bison    (parser generator,
            http://www.gnu.org/software/bison/bison.html)
  make     (program maintainance utility,
            http://www.gnu.org/software/make/make.html)
  gcc      (C compiler,
            http://www.gnu.org/software/gcc/gcc.html)
	\end{verbatim}

\noindent
The first step consists of compiling the ANSI C source files through the following commands:

	\begin{verbatim}
  > cd <TwoTowers 5.1 directory>/src/
  > make
  > make clean
	\end{verbatim}

\noindent
which should result in the following executable file:

	\begin{verbatim}
  <TwoTowers 5.1 directory>/bin/TTKernel
	\end{verbatim}

\noindent
The second step consists of creating a symbolic link to the above executable file through the following
command:

	\begin{verbatim}
  > ln -s <TwoTowers 5.1 directory>/bin/TTKernel TTKernel
	\end{verbatim}

\noindent
given in a directory whose pathname occurs in the shell variable {\tt path}.


%%%%%%%%%%%%%%%%%%%%%%%%%%%%%%%%%%%%%%%%%%%%%%%%%%%%%%%%%%%%%%%%%
%                                                               %
\subsection{Windows}
%                                                               %
%%%%%%%%%%%%%%%%%%%%%%%%%%%%%%%%%%%%%%%%%%%%%%%%%%%%%%%%%%%%%%%%%

The executable file for Windows is already available at:

	\begin{verbatim}
  <TwoTowers 5.1 directory>\bin\TTKernel.exe
	\end{verbatim}

\noindent
Should you need to generate it again, make sure that the following packages are available in \linebreak
\verb+\+\verb+Program Files\GnuWin32+:

	\begin{verbatim}
  flex     (lexical analyzer generator,
            http://gnuwin32.sourceforge.net/packages/flex.htm)
  bison    (parser generator,
            http://gnuwin32.sourceforge.net/packages/bison.htm)
	\end{verbatim}

\noindent
and that the following packages are available as well:

	\begin{verbatim}
  make     (program maintainance utility,
            http://www.mingw.org/)
  gcc      (C compiler,
            http://www.mingw.org/)
	\end{verbatim}

\noindent
Then compile the C source files through the following commands:

	\begin{verbatim}
  <double click> <TwoTowers 5.1 directory>\win_utils\tt_compile
  > make
  > make clean
	\end{verbatim}

\noindent
which should create the following executable file:

	\begin{verbatim}
  <TwoTowers 5.1 directory>\bin\TTKernel.exe
	\end{verbatim}



%%%%%%%%%%%%%%%%%%%%%%%%%%%%%%%%%%%%%%%%%%%%%%%%%%%%%%%%%%%%%%%%%
%                                                               %
%                                                               %
\section{Running the Tool}
%                                                               %
%                                                               %
%%%%%%%%%%%%%%%%%%%%%%%%%%%%%%%%%%%%%%%%%%%%%%%%%%%%%%%%%%%%%%%%%

Running the tool is very simple.


%%%%%%%%%%%%%%%%%%%%%%%%%%%%%%%%%%%%%%%%%%%%%%%%%%%%%%%%%%%%%%%%%
%                                                               %
\subsection{Linux}
%                                                               %
%%%%%%%%%%%%%%%%%%%%%%%%%%%%%%%%%%%%%%%%%%%%%%%%%%%%%%%%%%%%%%%%%

On a Linux machine, make sure that the following packages are available:

	\begin{verbatim}
  wish           (windowing shell for Tcl/Tk 8.0 or higher,
                  http://www.tcl.tk/software/tcltk/8.0.tml)
  NuSMV 2.2.5    (symbolic model checker,
                  http://nusmv.irst.itc.it/)
	\end{verbatim}

\noindent
Then type the following command to run the tool:

	\begin{verbatim}
  > wish <TwoTowers 5.1 directory>/gui/TTGUI &
	\end{verbatim}

\noindent
To simplify this, we suggest to define an alias like the following:

	\begin{verbatim}
  alias tt 'wish <TwoTowers 5.1 directory>/gui/TTGUI &'
	\end{verbatim}

\noindent
so that the command to run the tool simply becomes:

	\begin{verbatim}
  > tt
	\end{verbatim}

\noindent
In order to be able to use to model checker, we also suggest to make sure that the following symbolic links:

	\begin{verbatim}
  > ln -s <NuSMV 2.2.5 directory>/NuSMV NuSMV
  > ln -s <NuSMV 2.2.5 directory>/src/ltl/ltl2smv/ltl2smv ltl2smv
	\end{verbatim}

\noindent
have been created in a directory whose pathname occurs in the shell variable {\tt path}.


%%%%%%%%%%%%%%%%%%%%%%%%%%%%%%%%%%%%%%%%%%%%%%%%%%%%%%%%%%%%%%%%%
%                                                               %
\subsection{Windows}
%                                                               %
%%%%%%%%%%%%%%%%%%%%%%%%%%%%%%%%%%%%%%%%%%%%%%%%%%%%%%%%%%%%%%%%%

On a Windows machine, make sure that the following package is available in \verb+\Program Files\Tcl+:

	\begin{verbatim}
  wish           (windowing shell for Tcl/Tk 8.0 or higher,
                  http://www.tcl.tk/software/tcltk/8.0.tml)
	\end{verbatim}

\noindent
and that the following package is available in \verb+\Program Files\NuSMV-2.2.5+:

	\begin{verbatim}
  NuSMV 2.2.5    (symbolic model checker,
                  http://nusmv.irst.itc.it/)
	\end{verbatim}

\noindent
Then give the following command to run the tool:

	\begin{verbatim}
  <double click> <TwoTowers 5.1 directory>\win_utils\tt_exec
	\end{verbatim}
