%%%%%%%%%%%%%%%%%%%%%%%%%%%%%%%%%%%%%%%%%%%%%%%%%%%%%%%%%%%%%%%%%
%                                                               %
%                                                               %
%                                                               %
\chapter{The Security Analyzer}
%                                                               %
%                                                               %
%                                                               %
%%%%%%%%%%%%%%%%%%%%%%%%%%%%%%%%%%%%%%%%%%%%%%%%%%%%%%%%%%%%%%%%%


%%%%%%%%%%%%%%%%%%%%%%%%%%%%%%%%%%%%%%%%%%%%%%%%%%%%%%%%%%%%%%%%%
%                                                               %
%                                                               %
\section{Introduction}
%                                                               %
%                                                               %
%%%%%%%%%%%%%%%%%%%%%%%%%%%%%%%%%%%%%%%%%%%%%%%%%%%%%%%%%%%%%%%%%

TwoTowers~5.1 is able to check whether certain information flow properties, which are related to the
security levels of the system activities as expressed in a {\tt .sec} file, are satisfied by a correct
\aemilia\ specification, in which all the possible variable formal parameters and local variables are of
type bounded integer, boolean, or array or record based on the two previous basic types. The analysis is
carried out by means of the equivalence verifier and the result of the check is written to a {\tt .sar}
file, together with a modal logic formula expressed in a verbose variant of the Hennessy-Milner
logic~\cite{HM} (see Sect.~\ref{disting_formula}) to explain a possible security violation.



%%%%%%%%%%%%%%%%%%%%%%%%%%%%%%%%%%%%%%%%%%%%%%%%%%%%%%%%%%%%%%%%%
%                                                               %
%                                                               %
\section{Security Properties}
%                                                               %
%                                                               %
%%%%%%%%%%%%%%%%%%%%%%%%%%%%%%%%%%%%%%%%%%%%%%%%%%%%%%%%%%%%%%%%%

Two different security properties can be analyzed with TwoTowers~5.1: non-interference and non-deducibility
on composition~\cite{FG}.

Supposing that low security users observe public operations only while high security users perform
confidential operations only, an interference from high security users to low security users occurs if what
the high users can do is reflected in what the low users can observe. Formally, given an \aemilia\
specification and classified each of its action types as being high, low or irrelevant from the security
viewpoint, non-interference is satisfied if the functional semantic model of the \aemilia\ specification
with all the high action types being hidden is weakly bisimulation equivalent to the functional semantic
model of the \aemilia\ specification with all the high action types being restricted (the irrelevant action
types are hidden in both models). In this case the low users cannot infer the behavior of the high users by
observing the public view of the system, because the low users are not able to distinguish between the
situation in which the high users are carrying out some confidential operation and the opposite situation in
which the high users are not doing anything. This means that the system does not leak any secret information
from the high users to the low users.

Non-deducibility on composition is about the capability of altering or not the system view of the low users
when considering each of their potential interactions with the high users. Formally, given an \aemilia\
specification and the usual security-based classification of its action types, non-deducibility on
composition is satisfied if, for each pair of states of the functional semantic model of the \aemilia\
specification such that the first one has a transition that reaches the second one and is labeled with a
high action type, the two states are weakly bisimulation equivalent after restricting all the high action
types (and hiding all the irrelevant ones). This means that the low users are not able to note any
difference in the system behavior before and after each interaction with the high users.

The second property is more restrictive than the first one. Whenever an \aemilia\ specification satisfies
non-deducibility on composition, then it satisfies non-interference as well.



%%%%%%%%%%%%%%%%%%%%%%%%%%%%%%%%%%%%%%%%%%%%%%%%%%%%%%%%%%%%%%%%%
%                                                               %
%                                                               %
\section{Syntax of {\tt .sec} Specifications}
%                                                               %
%                                                               %
%%%%%%%%%%%%%%%%%%%%%%%%%%%%%%%%%%%%%%%%%%%%%%%%%%%%%%%%%%%%%%%%%

A {\tt .sec} specification has the following syntax

	\begin{verbatim}
    "HIGH_SECURITY" <security_decl_sequence> "LOW_SECURITY" <security_decl_sequence>
	\end{verbatim}

\noindent where {\tt <security\_decl\_sequence>} is a non-empty sequence of semicolon-separated security
declarations, each of the following form:

        \begin{verbatim}
        <security_decl> ::= "OBS_NRESTR_INTERNALS"
                         |  "OBS_NRESTR_INTERACTIONS"
                         |  "ALL_OBS_NRESTR"
                         |  <identifier> ["[" <expr> "]"] "." <action_type_set_s>
                         |  "FOR_ALL" <identifier> "IN" <expr> ".." <expr>
                            <identifier> "[" <expr> "]" "." <action_type_set_s>
    <action_type_set_s> ::= <identifier>
                         |  "OBS_NRESTR_INTERNALS"
                         |  "OBS_NRESTR_INTERACTIONS"
                         |  "ALL_OBS_NRESTR"
        \end{verbatim}

In its simpler form, a security declaration consists of associating a high or low security level with all
the observable, non-restricted action types that are internal to the AEIs of the \aemilia\ specification,
all the observable, non-restricted interactions of the AEIs of the \aemilia\ specification, or both of them.
Alternatively, it is possible to associate a high or low security level with a set of action types of a
specific AEI. In this case, the security declaration contains the identifier of the AEI to which the
high/low action types belong, a possible integer-valued expression enclosed in square brackets, which
represents a selector and must be free of undeclared identifiers and invocations to pseudo-random number
generators, and the identifier of the high/low security action type or one of the three shorthands above for
sets of high/low security action types. If specified, the AEI must be declared in the \aemilia\
specification to which the security analysis is applied. If specified, the high/low security action type
must occur in the behavior of the type of the AEI and cannot be hidden or restricted. Moreover, a high
security action type cannot be redeclared to be low security. The only identifiers that can occur in the
possible selector expression are the ones of the constant formal parameters declared in the architectural
type header of the \aemilia\ specification.

The more complex form is useful to concisely declare several action types to be at the same security level
through an indexing mechanism. This additionally requires the specification of the index identifier, which
can then occur in the selector expression, together with its range, which is given by two integer-valued
expressions. These two expressions must be free of undeclared identifiers and invocations to pseudo-random
number generators, with the value of the first expression being not greater than the value of the second
expression.



%%%%%%%%%%%%%%%%%%%%%%%%%%%%%%%%%%%%%%%%%%%%%%%%%%%%%%%%%%%%%%%%%
%                                                               %
%                                                               %
\section{Example B: The NRL Pump}
%                                                               %
%                                                               %
%%%%%%%%%%%%%%%%%%%%%%%%%%%%%%%%%%%%%%%%%%%%%%%%%%%%%%%%%%%%%%%%%

In order to check for the absence of illegal information flows in the \aemilia\ specification
{\tt nrl\_pump.aem} of Sect.~\ref{nrl_pump}, we classify its action types through the following
{\tt nrl\_pump.sec}:

	\begin{verbatim}
    HIGH_SECURITY HW.ALL_OBS_NRESTR

    LOW_SECURITY  LW.ALL_OBS_NRESTR 
	\end{verbatim}

\noindent where all the action types of the high (resp.\ low) wrapper are declared to be high (resp.\ low)
security.

The following is the result of the non-interference analysis:

	\begin{verbatim}
    NRL_Pump_Type violates the non-interference property
    as demonstrated by the following modal logic formula
    satisfied when the high security actions are hidden
    but not when the high security actions are restricted:

      EXISTS_WEAK_TRANS(
        LABEL(LW.send_conn_request#MT.receive_conn_request); 
        REACHED_STATE_SAT(
          EXISTS_WEAK_TRANS(
            LABEL(LW.receive_conn_valid#MT.send_conn_valid); 
            REACHED_STATE_SAT(TRUE)
          ) /\
          NOT(EXISTS_WEAK_TRANS(
                LABEL(LW.receive_conn_valid#MT.send_conn_valid); 
                REACHED_STATE_SAT(
                  EXISTS_WEAK_TRANS(
                    LABEL(LW.receive_conn_grant#TLT.send_conn_grant); 
                    REACHED_STATE_SAT(
                      NOT(EXISTS_WEAK_TRANS(
                            LABEL(LW.send_msg#TLT.receive_msg); 
                            REACHED_STATE_SAT(
                              EXISTS_WEAK_TRANS(
                                LABEL(LW.receive_low_ack#TLT.send_low_ack); 
                                REACHED_STATE_SAT(
                                  EXISTS_WEAK_TRANS(
                                    LABEL(LW.send_conn_close#TLT.receive_conn_close); 
                                    REACHED_STATE_SAT(TRUE)
                                  )
                                )
                              )
                            )
                          )
                      )
                    )
                  )
                )
              )
          ) /\
          NOT(EXISTS_WEAK_TRANS(
                LABEL(LW.receive_conn_reject#MT.send_conn_reject); 
                REACHED_STATE_SAT(TRUE)
              )
          )
        )
      )
	\end{verbatim}

\noindent The negative outcome above shows that, because of the acknowledgment mechanism, all the
connections are always aborted in the absence of high users, whereas some connections can successfully
terminate in the presence of high users. This reveals the existence of a covert channel related to a
connect/disconnect strategy, which a malicious high user may set up to pass confidential information to some
low user.

Since {\tt nrl\_pump.aem} does not satisfy non-interference, it cannot satisfy non-deducibility on
composition, either. In fact, the following is the result of the non-deducibility on composition analysis:

\newpage

	\begin{verbatim}
    NRL_Pump_Type violates the non-deducibility on composition property
    as demonstrated by the following modal logic formula
    satisfied by global state 45 with the high security actions restricted
    but not by global state 27 with the high security actions restricted:
	
      NOT(EXISTS_WEAK_TRANS(
            LABEL(LW.send_conn_close#TLT.receive_conn_close); 
            REACHED_STATE_SAT(TRUE)
          )
      )
	\end{verbatim}
