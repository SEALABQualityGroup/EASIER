%%%%%%%%%%%%%%%%%%%%%%%%%%%%%%%%%%%%%%%%%%%%%%%%%%%%%%%%%%%%%%%%%
%                                                               %
%                                                               %
%                                                               %
\chapter{The Model Checker}
%                                                               %
%                                                               %
%                                                               %
%%%%%%%%%%%%%%%%%%%%%%%%%%%%%%%%%%%%%%%%%%%%%%%%%%%%%%%%%%%%%%%%%


%%%%%%%%%%%%%%%%%%%%%%%%%%%%%%%%%%%%%%%%%%%%%%%%%%%%%%%%%%%%%%%%%
%                                                               %
%                                                               %
\section{Introduction}
%                                                               %
%                                                               %
%%%%%%%%%%%%%%%%%%%%%%%%%%%%%%%%%%%%%%%%%%%%%%%%%%%%%%%%%%%%%%%%%

TwoTowers~5.1 is able to check whether certain functional properties, each expressed as a LTL
formula~\cite{CGP} in a {\tt .ltl} file, are satisfied by a correct \aemilia\ specification, in which all
the possible variable formal parameters and local variables are of type bounded integer, boolean, or array
or record based on the two previous basic types. The verification is carried out by invoking the symbolic
model checker NuSMV~2.2.5~\cite{CCOPR} and the result of the check is written to a {\tt .mcr} file, together
with a counterexample for each property that is not satisfied.



%%%%%%%%%%%%%%%%%%%%%%%%%%%%%%%%%%%%%%%%%%%%%%%%%%%%%%%%%%%%%%%%%
%                                                               %
%                                                               %
\section{Syntax of {\tt .ltl} Specifications}
%                                                               %
%                                                               %
%%%%%%%%%%%%%%%%%%%%%%%%%%%%%%%%%%%%%%%%%%%%%%%%%%%%%%%%%%%%%%%%%

A {\tt .ltl} specification is a non-empty sequence of semicolon-separated property definitions, each of the
following form:

        \begin{verbatim}
    <property_def> ::= "PROPERTY" <identifier> ["[" <expr> "]"] "IS" <property_expr>
                    |  "FOR_ALL" <identifier> "IN" <expr> ".." <expr>
                       "PROPERTY" <identifier> "[" <expr> "]" "IS" <property_expr>
        \end{verbatim}

In its simpler form, a property definition contains the identifier of the property, a possible
integer-valued expression enclosed in square brackets, which represents a selector and must be free of
undeclared identifiers and invocations to pseudo-random number generators, and the property expression. The
only identifiers that can occur in the possible selector expression are the ones of the constant formal
parameters declared in the architectural type header of the \aemilia\ specification to which the model
checking is applied.

The second form is useful to concisely define several variants of the same property through an indexing
mechanism. This additionally requires the specification of the index identifier, which can then occur in the
selector expression and in the property expression, together with its range, which is given by two
integer-valued expressions. These two expressions must be free of undeclared identifiers and invocations to
pseudo-random number generators, with the value of the first expression being not greater than the value of
the second expression.

We observe that the identifier of a property can be augmented with a selector expression also in the simpler
form of property definition. This is useful whenever it is desirable to define a set of indexed variants of
the same property, but only some of them have a common selector expression.

The property expression is based on propositional and LTL operators and has the following verbose syntax:

        \begin{verbatim}
    <property_expr> ::= "TRUE"
                     |  "FALSE"
                     |  <property_expr> "/\" <property_expr>
                     |  <property_expr> "\/" <property_expr>
                     |  "NOT" "(" <property_expr> ")"
                     |  <property_expr> "\_/" <property_expr>
                     |  <property_expr> "->" <property_expr>
                     |  <property_expr> "<->" <property_expr>
                     |  "DEADLOCK_FREE"
                     |  <identifier> ["[" <expr> "]"] "." <identifier>
                     |  "NEXT_STATE_SAT" "(" <property_expr> ")"
                     |  "ALL_FUTURE_STATES_SAT" "(" <property_expr> ")"
                     |  "SOME_FUTURE_STATE_SAT" "(" <property_expr> ")"
                     |  <property_expr> "UNTIL" <property_expr>
                     |  <property_expr> "RELEASES" <property_expr>
                     |  "PREV_STATE_SAT" "(" <property_expr> ")"
                     |  "ALL_PAST_STATES_SAT" "(" <property_expr> ")"
                     |  "SOME_PAST_STATE_SAT" "(" <property_expr> ")"
                     |  <property_expr> "SINCE" <property_expr>
                     |  <property_expr> "TRIGGERED" <property_expr>
        \end{verbatim}

\noindent where the satisfaction relation with respect to a given state of the functional semantic model of
an \aemilia\ specification is defined as follows:

	\begin{itemize}

\item Constant {\tt TRUE} is satisfied in every state, while constant {\tt FALSE} is never satisfied.

\item The logical conjunction (\verb+/\+) of two properties is satisfied in a given state if so are both
properties.

\item The logical disjunction (\verb+\/+) of two properties is satisfied in a given state if so is at least
one of the two properties.

\item The logical negation ({\tt NOT}) of a property is satisfied in a given state if the property is not
satisfied in that state.

\item The logical exclusive disjunction (\verb+\_/+) of two properties is satisfied in a given state if so
is exactly one of the two properties.

\item The logical implication (\verb+->+) between two properties is satisfied in a given state if it is not
the case that the first property is satisfied while the second one is not.

\item The logical equivalence (\verb+<->+) between two properties is satisfied in a given state if both
properties are satisfied or none of them is.

\item Constant {\tt DEADLOCK\_FREE} is satisfied in a given state if no computation path starting from that
state deadlocks.

\item The property expression after {\tt DEADLOCK\_FREE} in the syntax above represents an action type,
which is denoted through its identifier preceded by the identifier of the AEI possibly augmented with an
integer-valued selector expression enclosed in square brackets, which must be free of undeclared identifiers
and invocations to pseudo-random number generators. The AEI must be declared in the \aemilia\ specification
to which the model checking is applied, and the action type must occur in the behavior of the type of the
AEI. This property is satisfied in a given state if, along every computation path starting from that state,
the state executes an action with the specified type (or a synchronizing type involving it, or a type
renaming it).

\item Operator {\tt NEXT\_STATE\_SAT} expresses with respect to a given state the fact that, for every
computation path traversing that state, the next state along the path satisfies a certain property.

\item Operator {\tt ALL\_FUTURE\_STATES\_SAT} expresses with respect to a given state the fact that, for
every computation path traversing that state, the traversed state and all the subsequent ones along the path
satisfy a certain property.

\item Operator {\tt SOME\_FUTURE\_STATE\_SAT} expresses with respect to a given state the fact that, for
every computation path traversing that state, there exists a state among the traversed one and all the
subsequent ones along the path that satisfies a certain property.

\item Operator {\tt UNTIL} expresses with respect to a given state the fact that, for every computation path
traversing that state, there exists a state among the traversed one and all the subsequent ones along the
path that satisfies the second property, with all the states between the traversed one and the one that
satisfies the second property satisfying the first property.

\item Operator {\tt RELEASES} expresses with respect to a given state the fact that, for every computation
path starting from that state, all the states among the traversed one and the subsequent ones along the path
satisfy the second property up to and including the first state (if any) that satisfies the first property.

\item Operator {\tt PREV\_STATE\_SAT} expresses with respect to a given state the fact that, for every
computation path traversing that state, the previous state along the path satisfied a certain property.

\item Operator {\tt ALL\_PAST\_STATES\_SAT} expresses with respect to a given state the fact that, for every
computation path traversing that state, the traversed state and all the previous ones along the path
satisfied a certain property.

\item Operator {\tt SOME\_PAST\_STATE\_SAT} expresses with respect to a given state the fact that, for every
computation path traversing that state, there exists a state among the traversed one and all the previous
ones along the path that satisfied a certain property.

\item Operator {\tt SINCE} expresses with respect to a given state the fact that, for every computation path
traversing that state, there exists a state among the traversed one and all the previous ones along the path
that satisfied the second property, with all the states between the traversed one and the one that satisfied
the second property satisfying the first property.

\item Operator {\tt TRIGGERED} expresses with respect to a given state the fact that, for every computation
path traversing that state, all the states among the traversed one and the previous ones along the path
satisfied the second property up to and including the first state (if any) that satisfied the first
property.

	\end{itemize}

The infix temporal operators {\tt UNTIL}, {\tt RELEASES}, {\tt SINCE}, and {\tt TRIGGERED} take precedence
over the logical conjunction operator, which takes precedence over the two logical disjunction operators,
which takes precedende over the logical equivalence operator, which takes precedence over the logical
implication operator. All the mentioned infix operators are left associative, except for the logical
implication one, which is right associative. The precedence and associativity of such operators can be
altered using parentheses {\tt ( )}.

When checked against an \aemilia\ specification, a property expressed in a {\tt .ltl} file holds if it is
satisfied by the initial global state of the functional semantic model of the \aemilia\ specification.



%%%%%%%%%%%%%%%%%%%%%%%%%%%%%%%%%%%%%%%%%%%%%%%%%%%%%%%%%%%%%%%%%
%                                                               %
%                                                               %
\section{Example A: The Alternating Bit Protocol}
%                                                               %
%                                                               %
%%%%%%%%%%%%%%%%%%%%%%%%%%%%%%%%%%%%%%%%%%%%%%%%%%%%%%%%%%%%%%%%%

The correctness of the \aemilia\ specification {\tt abp.aem} of Sect.~\ref{abp} can be checked against the
following {\tt abp.ltl}:

	\begin{verbatim}
    PROPERTY deadlock_freedom IS
      DEADLOCK_FREE;

    PROPERTY always_consumes_after_generating IS
      ALL_FUTURE_STATES_SAT(R.consume_msg -> SOME_PAST_STATE_SAT(S.generate_msg));

    PROPERTY correct_alternation IS
      ALL_FUTURE_STATES_SAT(
        (S.generate_msg -> (NEXT_STATE_SAT((NOT(S.generate_msg) UNTIL R.consume_msg) \_/
                                           ALL_FUTURE_STATES_SAT(NOT(S.generate_msg) /\
                                                                 NOT(R.consume_msg))))) /\
        (R.consume_msg -> (NEXT_STATE_SAT((NOT(R.consume_msg) UNTIL S.generate_msg) \_/
                                          ALL_FUTURE_STATES_SAT(NOT(R.consume_msg) /\
                                                                NOT(S.generate_msg))))))
	\end{verbatim}

\noindent The first property ensures that the protocol never causes a deadlock to occur. The second property
guarantees that, whenever a message is consumed at the receiver side, then it must have been previously
generated at the sender side. The third property establishes that message generations and consumptions
correctly alternate. Whenever a message is generated at the sender side, then along every computation path
it must be the case the either no new message is generated until the considered one is consumed at the
receiver side, or no further message generations and consumptions take place. Likewise, whenever a message
is consumed at the receiver side, then along every computation path it must be the case the either no
message is consumed until a new one is generated at the sender side, or no further message generations and
consumptions take place.

The following is the result of the model checking:

	\begin{verbatim}
    Validity of the properties for ABP_Type:

    - Property "deadlock_freedom" is satisfied.

    - Property "always_consumes_after_generating" is satisfied.

    - Property "correct_alternation" is satisfied.
	\end{verbatim}



%%%%%%%%%%%%%%%%%%%%%%%%%%%%%%%%%%%%%%%%%%%%%%%%%%%%%%%%%%%%%%%%%
%                                                               %
%                                                               %
\section{Example C: Dining Philosophers}
%                                                               %
%                                                               %
%%%%%%%%%%%%%%%%%%%%%%%%%%%%%%%%%%%%%%%%%%%%%%%%%%%%%%%%%%%%%%%%%

The correctness of the \aemilia\ specification {\tt dining\_philosophers.aem} of Sect.~\ref{dp} can be
checked against the following {\tt dining\_philosophers.ltl}:

	\begin{verbatim}
    PROPERTY deadlock_freedom IS
      DEADLOCK_FREE;

    FOR_ALL i IN 0..philosopher_num - 1
      PROPERTY starvation_freedom[i] IS
        ALL_FUTURE_STATES_SAT(SOME_FUTURE_STATE_SAT(P[i].eat));

    FOR_ALL i IN 0..philosopher_num - 1
      PROPERTY no_adjacent_philosopher_simultaneously_eating[i] IS
        ALL_FUTURE_STATES_SAT(
          P[i].eat -> NEXT_STATE_SAT(
                        NOT(P[mod(i + 1, philosopher_num)].eat \/
                            P[mod((philosopher_num + i) - 1, philosopher_num)].eat)
                        UNTIL P[i].put_down_left))
	\end{verbatim}

\noindent The first property ensures that the algorithm avoids deadlock, in the sense that whenever several
philosophers are hungry, at least one of them manages to get the chopsticks and eat. The second set of
properties guarantees that no philosopher starves, i.e.\ whenever a philosopher is hungry, it eventually
manages to eat. The third set of properties establishes the mutual exclusive usage of the chopsticks, in the
sense that whenever a philosopher eats, then none of its two neighbors can eat until the philosopher
releases both chopsticks.

What follows is the result of the model checking, where for each unsatisfied property a computation path
violating it is printed as the sequence of the types of the actions executed along the path, together with
the indication of possible loops in the path:

	\begin{verbatim}
    Validity of the properties for LR_Dining_Philosophers_Type:

    - Property "deadlock_freedom" is satisfied.

    - Property "starvation_freedom[0]" isn't satisfied
      as demonstrated by the following execution sequence:
        <<loop starts here>>
        P[1].think
        P[1].flip_tail
        C[2].pick_up_first.2#P[1].pick_up_left_first
        C[1].pick_up_then.1#P[1].pick_up_right_then
        P[1].eat
        C[1].put_down.1#P[1].put_down_right
        C[2].put_down.2#P[1].put_down_left
        P[1].think

    - Property "starvation_freedom[1]" isn't satisfied
      as demonstrated by the following execution sequence:
        P[1].think
        P[1].flip_head
        C[1].pick_up_first.1#P[1].pick_up_right_first
        C[2].pick_up_then.2#P[1].pick_up_left_then
        P[2].think
        P[2].flip_head
        P[0].think
        <<loop starts here>>
        P[0].flip_head
        C[0].pick_up_first.1#P[0].pick_up_right_first
        C[0].put_down.1#P[0].put_down_right
        P[0].flip_head

    - Property "starvation_freedom[2]" isn't satisfied
      as demonstrated by the following execution sequence:
        <<loop starts here>>
        P[1].think
        P[1].flip_tail
        C[2].pick_up_first.2#P[1].pick_up_left_first
        C[1].pick_up_then.1#P[1].pick_up_right_then
        P[1].eat
        C[1].put_down.1#P[1].put_down_right
        C[2].put_down.2#P[1].put_down_left
        P[1].think

    - Property "no_adjacent_philosopher_simultaneously_eating[0]" is satisfied.

    - Property "no_adjacent_philosopher_simultaneously_eating[1]" is satisfied.

    - Property "no_adjacent_philosopher_simultaneously_eating[2]" is satisfied.
	\end{verbatim}
