%%%%%%%%%%%%%%%%%%%%%%%%%%%%%%%%%%%%%%%%%%%%%%%%%%%%%%%%%%%%%%%%%
%                                                               %
%                                                               %
%                                                               %
\chapter{The Performance Evaluator}
%                                                               %
%                                                               %
%                                                               %
%%%%%%%%%%%%%%%%%%%%%%%%%%%%%%%%%%%%%%%%%%%%%%%%%%%%%%%%%%%%%%%%%


%%%%%%%%%%%%%%%%%%%%%%%%%%%%%%%%%%%%%%%%%%%%%%%%%%%%%%%%%%%%%%%%%
%                                                               %
%                                                               %
\section{Introduction}
%                                                               %
%                                                               %
%%%%%%%%%%%%%%%%%%%%%%%%%%%%%%%%%%%%%%%%%%%%%%%%%%%%%%%%%%%%%%%%%

TwoTowers~5.1 is able to evaluate the performance of correct \aemilia\ specifications in two different ways.

In the first case instant-of-time, stationary/transient performance measures, which are defined through
state and transitions rewards~\cite{How} in a {\tt .rew} file, are computed by solving the Markov chain
underlying the \aemilia\ specification. The value of each such performance measure, which is written to a
{\tt .val} file, is given by the sum of the stationary/transient state probabilities and transition
frequencies of the Markov chain, each weighed by the corresponding state reward or transition reward,
respectively. A state reward represents the rate at which gain is cumulated while staying in a certain
state, whereas a transition reward represents the gain that is instantaneously earned when executing a
certain transition. In TwoTowers~5.1 three methods are available for solving Markov chains: Gaussian
elimination, an adaptive variant of symmetric stochastic over-relaxation, and uniformization~\cite{Ste1}.
Gaussian elimination is an exact method for computing the stationary solution of small Markov chains (up to
a few thousands of states), while symmetric stochastic over-relaxation is an iterative method for computing
the stationary solution of larger Markov chains. On the contrary, uniformization is an iterative method for
computing the transient solution of Markov chains. The state probability distribution representing the
solution of a Markov chain is written to a {\tt .dis} file. The Markovian performance evaluation can be
applied only to (correct and) performance closed \aemilia\ specifications in which all the possible variable
formal parameters and local variables are of type bounded integer, boolean, or array or record based on the
two previous basic types.

In the second case the method of independent replications~\cite{Wel}, based on simulation experiments
described in a {\tt .sim} file, is applied to estimate with 90\% confidence level the mean, variance or
distribution of performance measures, which are specified through an extension of state and transition
rewards in the same {\tt .sim} file. The discrete event simulation can be trace driven, which means that
certain values are taken from a {\tt .trc} file instead of being sampled from pseudo-random number
generators. The result of the simulation is written to a {\tt .est} file. Unlike the Markovian performance
evaluation, the simulation-based performance evaluation can be applied to any (correct) \aemilia\
specification with no open and deadlock states, thus making it possible the estimation of the performance
measures of systems with generally distributed delays. Besides the compile-time crashes mentioned in
Sect.~\ref{ct_crashes}, the discrete event simulation of an \aemilia\ specification can be interrupted --
during the construction of the portion of the integrated semantic model that is necessary to make the
simulation advance -- because of the generation of a deadlock state, the generation of an open state, or the
absence of sufficiently many values to be read from a {\tt .trc} file.



%%%%%%%%%%%%%%%%%%%%%%%%%%%%%%%%%%%%%%%%%%%%%%%%%%%%%%%%%%%%%%%%%
%                                                               %
%                                                               %
\section{Syntax of {\tt .rew} Specifications}
%                                                               %
%                                                               %
%%%%%%%%%%%%%%%%%%%%%%%%%%%%%%%%%%%%%%%%%%%%%%%%%%%%%%%%%%%%%%%%%

A {\tt .rew} specification is a non-empty sequence of semicolon-separated measure definitions, each of the
following form:

        \begin{verbatim}
    <measure_def> ::= "MEASURE" <identifier> ["[" <expr> "]"] "IS" <reward_structure>
                   |  "FOR_ALL" <identifier> "IN" <expr> ".." <expr>
                      "MEASURE" <identifier> "[" <expr> "]" "IS" <reward_structure>
        \end{verbatim}

In its simpler form, a measure definition contains the identifier of the measure, a possible
integer-valued expression enclosed in square brackets, which represents a selector and must be free of
undeclared identifiers and invocations to pseudo-random number generators, and the reward structure. The
only identifiers that can occur in the possible selector expression are the ones of the constant formal
parameters declared in the architectural type header of the \aemilia\ specification to which the Markovian
performance evaluation is applied.

The second form is useful to concisely define several variants of the same measure through an indexing
mechanism. This additionally requires the specification of the index identifier, which can then occur in the
selector expression and in the reward structure, together with its range, which is given by two
integer-valued expressions. These two expressions must be free of undeclared identifiers and invocations to
pseudo-random number generators, with the value of the first expression being not greater than the value of
the second expression.

We observe that the identifier of a measure can be augmented with a selector expression also in the simpler
form of measure definition. This is useful whenever it is desirable to define a set of indexed variants of
the same measure, but only some of them have a common selector expression.

The reward structure is a non-empty sequence of reward assignments, each of the following form:

        \begin{verbatim}
    <reward_assign> ::= "ENABLED" "(" <identifier> ["[" <expr> "]"] "." <identifier> ")"
                        "->" <reward_type> "(" <expr> ")"
                     |  "FOR_ALL" <identifier> "IN" <expr> ".." <expr>
                        "ENABLED" "(" <identifier> "[" <expr> "]" "." <identifier> ")"
                        "->" <reward_type> "(" <expr> ")"
      <reward_type> ::= "STATE_REWARD"
                     |  "TRANS_REWARD"
        \end{verbatim}

In its simpler form, a reward assignment contains the identifier of an action type preceded by the
identifier of an AEI possibly augmented with an integer-valued selector expression enclosed in square
brackets, which must be free of undeclared identifiers and invocations to pseudo-random number generators.
The AEI must be declared in the \aemilia\ specification to which the Markovian performance evaluation is
applied, and the action type must occur in the behavior of the type of the AEI within non-passive actions.
The only identifiers that can occur in the possible selector expression are the ones of the constant formal
parameters declared in the architectural type header of the \aemilia\ specification.

The meaning is that, whenever an action with the specified type is enabled in a state, then that state
(resp.\ the transition that leaves that state and is originated from the considered action) is associated a
state (resp.\ transition) reward given by the value of the expression following symbol {\tt ->}. The reward
expression must be real valued as well as free of undeclared identifiers and invocations to pseudo-random
number generators. The only identifiers that can occur in the reward expression are the ones of the constant
formal parameters declared in the architectural type header of the \aemilia\ specification, together with
the index possibly occurring at the beginning of the measure definition. An action type can occur at most
once in the reward structure specified within a measure definition.

The second form is useful to concisely define several variants of the same reward assignment through an
indexing mechanism. This additionally requires the specification of the index identifier, which can then
occur in the selector expression and in the reward expression, together with its range, which is given by
two integer-valued expressions. These two expressions must be free of undeclared identifiers and invocations
to pseudo-random number generators, with the value of the first expression being not greater than the value
of the second expression. The index for the reward assignment must be different from the index possibly
occurring at the beginning of the measure definition.



%%%%%%%%%%%%%%%%%%%%%%%%%%%%%%%%%%%%%%%%%%%%%%%%%%%%%%%%%%%%%%%%%
%                                                               %
%                                                               %
\section{Syntax of {\tt .sim} Specifications}
%                                                               %
%                                                               %
%%%%%%%%%%%%%%%%%%%%%%%%%%%%%%%%%%%%%%%%%%%%%%%%%%%%%%%%%%%%%%%%%

A {\tt .sim} specification is composed of five sections:

        \begin{verbatim}
    <clock_act_type>
    <sim_run_length>
    <sim_run_number>
    <measure_def_sequence>
    [<trace_def_sequence>]
        \end{verbatim}


%%%%%%%%%%%%%%%%%%%%%%%%%%%%%%%%%%%%%%%%%%%%%%%%%%%%%%%%%%%%%%%%%
%                                                               %
\subsection{Clock Action Type}
%                                                               %
%%%%%%%%%%%%%%%%%%%%%%%%%%%%%%%%%%%%%%%%%%%%%%%%%%%%%%%%%%%%%%%%%

The clock action type represents the action type on the basis of which time is assumed to progress during
the simulation. Every execution of a transition labeled with the clock action type (or a type involving or
renaming it) corresponds to a clock tick.

The clock action type is defined through the following syntax:

        \begin{verbatim}
    "RUN_LENGTH_ON_EXEC" <identifier> ["[" <expr> "]"] "." <identifier>
        \end{verbatim}

\noindent which contains the identifier of an action type preceded by the identifier of an AEI possibly
augmented with an integer-valued selector expression enclosed in square brackets, which must be free of
undeclared identifiers and invocations to pseudo-random number generators. The AEI must be declared in the
\aemilia\ specification to which the simulation-based performance evaluation is applied, and the action type
must occur in the behavior of the type of the AEI within non-passive actions. The action type cannot be
hidden or restricted in the \aemilia\ specification. The only identifiers that can occur in the possible
selector expression are the ones of the constant formal parameters declared in the architectural type header
of the \aemilia\ specification.


%%%%%%%%%%%%%%%%%%%%%%%%%%%%%%%%%%%%%%%%%%%%%%%%%%%%%%%%%%%%%%%%%
%                                                               %
\subsection{Simulation Run Length}
%                                                               %
%%%%%%%%%%%%%%%%%%%%%%%%%%%%%%%%%%%%%%%%%%%%%%%%%%%%%%%%%%%%%%%%%

The simulation run length specifies the number of times that a transition labeled with the clock action type
(or a type involving or renaming it) must be executed in order for a simulation run to be considered
terminated.

The simulation run length is defined through the following syntax:

        \begin{verbatim}
    "RUN_LENGTH" <expr>
        \end{verbatim}

\noindent where the expression must be integer valued as well as free of undeclared identifiers and
invocations to pseudo-random number generators. The only identifiers that can occur in the expression are
the ones of the constant formal parameters declared in the architectural type header of the \aemilia\
specification to which the simulation-based performance evaluation is applied. The value of the expression
must be greater than zero.


%%%%%%%%%%%%%%%%%%%%%%%%%%%%%%%%%%%%%%%%%%%%%%%%%%%%%%%%%%%%%%%%%
%                                                               %
\subsection{Simulation Run Number}
%                                                               %
%%%%%%%%%%%%%%%%%%%%%%%%%%%%%%%%%%%%%%%%%%%%%%%%%%%%%%%%%%%%%%%%%

The simulation run number specifies the number of independent simulation runs that have to be carried out
in order for the simulation to be considered terminated.

The simulation run number is defined through the following syntax:

        \begin{verbatim}
    "RUN_NUMBER" <expr>
        \end{verbatim}

\noindent where the expression must be integer valued as well as free of undeclared identifiers and
invocations to pseudo-random number generators. The only identifiers that can occur in the expression are
the ones of the constant formal parameters declared in the architectural type header of the \aemilia\
specification to which the simulation-based performance evaluation is applied. The value of the expression
must be between~1 and~30.


%%%%%%%%%%%%%%%%%%%%%%%%%%%%%%%%%%%%%%%%%%%%%%%%%%%%%%%%%%%%%%%%%
%                                                               %
\subsection{Measure Definition Sequence}
%                                                               %
%%%%%%%%%%%%%%%%%%%%%%%%%%%%%%%%%%%%%%%%%%%%%%%%%%%%%%%%%%%%%%%%%

The fourth section is a non-empty sequence of semicolon-separated measure definitions, each of the following
form:

        \begin{verbatim}
    <measure_def> ::= "MEASURE" <identifier> ["[" <expr> "]"] "IS" <measure_expr>
                   |  "FOR_ALL" <identifier> "IN" <expr> ".." <expr>
                      "MEASURE" <identifier> "[" <expr> "]" "IS" <measure_expr>
        \end{verbatim}

In its simpler form, a measure definition contains the identifier of the measure, a possible
integer-valued expression enclosed in square brackets, which represents a selector and must be free of
undeclared identifiers and invocations to pseudo-random number generators, and the measure expression. The
only identifiers that can occur in the possible selector expression are the ones of the constant formal
parameters declared in the architectural type header of the \aemilia\ specification to which the
simulation-based performance evaluation is applied.

The second form is useful to concisely define several variants of the same measure through an indexing
mechanism. This additionally requires the specification of the index identifier, which can then occur in the
selector expression and in the measure expression, together with its range, which is given by two
integer-valued expressions. These two expressions must be free of undeclared identifiers and invocations to
pseudo-random number generators, with the value of the first expression being not greater than the value of
the second expression.

We observe that the identifier of a measure can be augmented with a selector expression also in the simpler
form of measure definition. This is useful whenever it is desirable to define a set of indexed variants of
the same measure, but only some of them have a common selector expression.

The measure expression has the following syntax:

        \begin{verbatim}
    <measure_expr> ::= "MEAN" "(" <sim_expr> "," <expr> ".." <expr> ")"
                    |  "VARIANCE" "(" <sim_expr> "," <expr> ".." <expr> ")"
                    |  "DISTRIBUTION" "(" <sim_expr> "," <expr> ".." <expr> "," <expr> ")"
        \end{verbatim}

\noindent where {\tt <sim\_expr>} is the expression whose mean, variance, or distribution has to be
estimated during the simulation, while the other expressions, which must be integer valued as well as free
of undeclared identifiers and invocations to pseudo-random number generators, delimit the observation
interval within a simulation run. The first expression represents the beginning of the observation interval,
whose value must be between zero and the simulation run length decremented by one. The second expression
represents the end of the observation interval, whose value must be between one and the simulation run
length, and greater than the value of the previous expression. The third expression, which is present only
in the case of the distribution, represents the width of the subintervals -- within the observation interval
-- at the end of which the distribution must be estimated. Its value must be greater than zero and a divisor
of the length of the observation interval, which is given by the difference between the values of the two
previous expressions. The only identifiers that can occur in these expressions are the ones of the constant
formal parameters declared in the architectural type header of the \aemilia\ specification to which the
simulation-based performance evaluation is applied, together with the index possibly occurring at the
beginning of the measure definition.

The syntax for {\tt <sim\_expr>} is the same as the syntax for {\tt <expr>}, with in addition the following
reward-based production:

        \begin{verbatim}
      <sim_expr> ::= "REWARD" "(" "EXECUTED" "(" <identifier> ["[" <expr> "]"] "."
                     <identifier> ")" "->" <reward_expr> "," <cumulative> ")"
    <cumulative> ::= "CUMULATIVE"
                  |  "NON_CUMULATIVE"
        \end{verbatim}

\noindent which contains the identifier of an action type preceded by the identifier of an AEI possibly
augmented with an integer-valued selector expression enclosed in square brackets, which must be free of
undeclared identifiers and invocations to pseudo-random number generators. The AEI must be declared in the
\aemilia\ specification to which the simulation-based performance evaluation is applied, and the action type
must occur in the behavior of the type of the AEI within non-passive actions. The only identifiers that can
occur in the possible selector expression are the ones of the constant formal parameters declared in the
architectural type header of the \aemilia\ specification, together with the index possibly occurring at the
beginning of the measure definition.

The meaning is that, whenever a transition labeled with the specified action type (or a type involving or
renaming it) is executed, then a reward is earned whose value is given by the expression following symbol
{\tt ->}. All the values of the reward expression collected during a run are summed up at the end of the run
-- with this sum being divided by the number of collected values if the measure is not cumulative -- then
all the sums are involved in a statistical inference process at the end of the simulation in order to derive
the measure estimate with 90\% confidence level. The reward expression must be real valued as well as free
of undeclared identifiers and invocations to pseudo-random number generators. The only identifiers that can
occur in the reward expression are the ones of the constant formal parameters declared in the architectural
type header of the \aemilia\ specification, together with the index possibly occurring at the beginning of
the measure definition.

The syntax for {\tt <reward\_expr>} is the same as the syntax for {\tt <expr>}, with in addition the
following variable-based production:

        \begin{verbatim}
    <reward_expr> ::= <identifier> ["[" <expr> "]"] "." <identifier> "." <identifier>
        \end{verbatim}

\noindent which contains the identifier of a variable formal parameter or local variable preceded by the
identifier of a behavior, which is in turn preceded by the identifier of an AEI possibly augmented with an
integer-valued selector expression enclosed in square brackets, which must be free of undeclared identifiers
and invocations to pseudo-random number generators. The AEI must be declared in the \aemilia\ specification
to which the simulation-based performance evaluation is applied, the behavior must be defined within the
type of the AEI, and the variable formal parameter or local variable must be declared in the behavior
header. Upon evaluation, this denotes the value hold in the variable formal parameter or local variable at
the time of the evaluation. The only identifiers that can occur in the possible selector expression are the
ones of the constant formal parameters declared in the architectural type header of the \aemilia\
specification, together with the index possibly occurring at the beginning of the measure definition.


%%%%%%%%%%%%%%%%%%%%%%%%%%%%%%%%%%%%%%%%%%%%%%%%%%%%%%%%%%%%%%%%%
%                                                               %
\subsection{Trace Definition Sequence}
%                                                               %
%%%%%%%%%%%%%%%%%%%%%%%%%%%%%%%%%%%%%%%%%%%%%%%%%%%%%%%%%%%%%%%%%

The fifth section, which is optional, is a possibly empty list of semicolon-separated trace definitions,
each of the following form:

        \begin{verbatim}
    <trace_def> ::= "DRAW" <trace_expr> "FROM" <trace_file_path> ["[" <expr> "]"] ".trc"
                 |  "FOR_ALL" <identifier> "IN" <expr> ".." <expr>
                    "DRAW" <trace_expr> "FROM" <trace_file_path> "[" <expr> "]" ".trc"
        \end{verbatim}

In its simpler form, a trace definition contains the trace expression to be sampled, the trace file path --
without extension -- from which the values for the trace expression are to be read during simulation, a
possible integer-valued expression enclosed in square brackets, which represents a selector and must be free
of undeclared identifiers and invocations to pseudo-random number generators, and the {\tt .trc} extension.
The only identifiers that can occur in the possible selector expression are the ones of the constant formal
parameters declared in the architectural type header of the \aemilia\ specification to which the
simulation-based performance evaluation is applied.

The second form is useful to concisely define several variants of the same trace through an indexing
mechanism. This additionally requires the specification of the index identifier, which can then occur in the
selector expression, together with its range, which is given by two integer-valued expressions. These two
expressions must be free of undeclared identifiers and invocations to pseudo-random number generators, with
the value of the first expression being not greater than the value of the second expression.

The path of a trace file is relative to the directory containing the {\tt .sim} specification, must start
with {\tt ./}, and must contain {\tt /} (rather than \verb+\+) as separator for directory names. The trace
file path can be augmented with a selector expression also in the simpler form of trace definition. This is
useful whenever it is desirable to associate different trace files with a set of indexed variants of the
same trace expression belonging to different AEIs, but only some of them have a common selector expression.

The syntax for {\tt <trace\_expr>} is the same as the syntax for {\tt <expr>}, except that it must start
with an invocation to a pseudo-random number generator and has the following constant-based production in
place of the production for a simple identifier:

        \begin{verbatim}
    <trace_expr> ::= <identifier> ["[" <expr> "]"] "." <identifier>
        \end{verbatim}

\noindent which contains the identifier of a constant formal parameter preceded by the identifier of an AEI
possibly augmented with an integer-valued selector expression enclosed in square brackets, which must be
free of undeclared identifiers and invocations to pseudo-random number generators. The AEI must be declared
in the \aemilia\ specification to which the simulation-based performance evaluation is applied, and the
constant formal parameter must be declared in the header of the type of the AEI. The overall trace
expression must occur in the behavior of the type of some AEI of the \aemilia\ specification, and at most
one trace file can be associated with it. Note that no variable formal parameter or local variable can occur
in the trace expression. The only identifiers that can occur in the possible selector expression are the
ones of the constant formal parameters declared in the architectural type header of the \aemilia\
specification, together with the index possibly occurring at the beginning of the trace definition.



%%%%%%%%%%%%%%%%%%%%%%%%%%%%%%%%%%%%%%%%%%%%%%%%%%%%%%%%%%%%%%%%%
%                                                               %
%                                                               %
\section{Syntax of {\tt .trc} Specifications}
%                                                               %
%                                                               %
%%%%%%%%%%%%%%%%%%%%%%%%%%%%%%%%%%%%%%%%%%%%%%%%%%%%%%%%%%%%%%%%%

A {\tt .trc} file must contain sufficiently many real numbers in fixed point notation, each starting at the
beginning of a new line.



%%%%%%%%%%%%%%%%%%%%%%%%%%%%%%%%%%%%%%%%%%%%%%%%%%%%%%%%%%%%%%%%%
%                                                               %
%                                                               %
\section{Example A: The Alternating Bit Protocol}
%                                                               %
%                                                               %
%%%%%%%%%%%%%%%%%%%%%%%%%%%%%%%%%%%%%%%%%%%%%%%%%%%%%%%%%%%%%%%%%

In this section we evaluate the performance of the alternating bit protocol.


%%%%%%%%%%%%%%%%%%%%%%%%%%%%%%%%%%%%%%%%%%%%%%%%%%%%%%%%%%%%%%%%%
%                                                               %
\subsection{Markovian Performance Evaluation}
%                                                               %
%%%%%%%%%%%%%%%%%%%%%%%%%%%%%%%%%%%%%%%%%%%%%%%%%%%%%%%%%%%%%%%%%

The performance measures of interest for the \aemilia\ specification {\tt abp.aem} of Sect.~\ref{abp} can be
defined through the following {\tt abp.rew}:

        \begin{verbatim}
    MEASURE throughput IS
      ENABLED(S.generate_msg) -> TRANS_REWARD(1);

    MEASURE utilization IS
      ENABLED(LM.propagate_0) -> STATE_REWARD(1)
      ENABLED(LM.propagate_1) -> STATE_REWARD(1)
        \end{verbatim}

\noindent The throughput represents the number of messages that are delivered per unit of time. It is
obtained by associating an instantaneous unit reward with the transitions originated from
{\tt S.generate\_msg}. Equivalently, it could have been obtained by associating a reward rate equal to the
rate of {\tt S.generate\_msg} with every state in which {\tt S.generate\_msg} is enabled. The utilization
represents instead the percentage of time during which the channel is busy because of message propagation.
It is obtained by associating a unit reward rate with every state in which {\tt LM.propagate\_0} or
{\tt LM.propagate\_1} is enabled.

The following is the result of the Markovian performance evaluation at steady state:

        \begin{verbatim}
    Stationary value of the performance measures for ABP_Type:

    - Value of measure "throughput":
        1.88226

    - Value of measure "utilization":
        0.26291
        \end{verbatim}

\noindent As we can see, the throughput is much less than the maximum potential value (i.e.\ the rate of
{\tt S.generate\_msg}) and the utilization is about~26\%.

Similar results are obtained when considering the value passing \aemilia\ specification {\tt abp\_vp.aem}
of Sect.~\ref{abp_vp} in place of {\tt abp.aem}, provided that the following {\tt abp\_vp.rew} is used:

        \begin{verbatim}
    MEASURE throughput IS
      ENABLED(S.generate_msg) -> TRANS_REWARD(1);

    MEASURE utilization IS
      ENABLED(LM.propagate) -> STATE_REWARD(1)
        \end{verbatim}


%%%%%%%%%%%%%%%%%%%%%%%%%%%%%%%%%%%%%%%%%%%%%%%%%%%%%%%%%%%%%%%%%
%                                                               %
\subsection{Simulation-Based Performance Evaluation}
%                                                               %
%%%%%%%%%%%%%%%%%%%%%%%%%%%%%%%%%%%%%%%%%%%%%%%%%%%%%%%%%%%%%%%%%

The mean, variance, and distribution of the same performance measures as before can be defined for the
\aemilia\ specification {\tt abp\_gd.aem} with general delays of Sect.~\ref{abp_gd} through the following
{\tt abp\_gd.sim}:

        \begin{verbatim}
    RUN_LENGTH_ON_EXEC
      C.elapse_tick

    RUN_LENGTH
      10000

    RUN_NUMBER
      20

    MEASURE throughput_avg IS
      MEAN
      {
        REWARD(EXECUTED(MG.generate_msg) -> 1,
               CUMULATIVE) / 10,
        0..10000
      };

    MEASURE throughput_var IS
      VARIANCE
      {
        REWARD(EXECUTED(MG.generate_msg) -> 1,
               CUMULATIVE) / 10,
        0..10000
      };

    MEASURE throughput_distr IS
      DISTRIBUTION
      {
        REWARD(EXECUTED(MG.generate_msg) -> 1,
               CUMULATIVE) / 1,
        0..10000,
        1000
      };

    MEASURE utilization_avg IS
      MEAN
      {
        REWARD(EXECUTED(LM.propagate) -> 1,
               CUMULATIVE) / 10000,
        0..10000
      };

    MEASURE utilization_var IS
      VARIANCE
      {
        REWARD(EXECUTED(LM.propagate) -> 1,
               CUMULATIVE) / 10000,
        0..10000
      };

    MEASURE utilization_distr IS
      DISTRIBUTION
      {
        REWARD(EXECUTED(LM.propagate) -> 1,
               CUMULATIVE) / 1000,
        0..10000,
        1000
      }
        \end{verbatim}

\noindent Note that the duration of each run corresponds to 10000~msec of execution of the protocol and that
the throughput is expressed in number of messages delivered per second, hence the division by~10 instead
of~10000. The distributions of the throughput and of the utilization are estimated at the end of each of the
10~seconds.

The following is the result of the simulation-based performance evaluation, where the 90\% confidence
intervals are shown in square brackets:

        \begin{verbatim}
    90% confidence estimate of the performance measures for ABP_GD_Type:

    - Estimate of measure "throughput_avg":
        3.005
                [2.75757, 3.25243]

    - Estimate of measure "throughput_var":
        0.455237
                [0.258599, 0.651874]

    - Estimate of measure "throughput_distr":
        3.25
                [2.80682, 3.69318]
        2.5
                [1.98826, 3.01174]
        3.05
                [2.46139, 3.63861]
        2.8
                [2.20989, 3.39011]
        3.25
                [2.64479, 3.85521]
        3
                [2.31652, 3.68348]
        3.2
                [2.65999, 3.74001]
        2.6
                [1.84376, 3.35624]
        3.75
                [3.24697, 4.25303]
        2.65
                [2.02863, 3.27137]

    - Estimate of measure "utilization_avg":
        0.33745
                [0.31401, 0.36089]

    - Estimate of measure "utilization_var":
        0.00408571
                [0.00233904, 0.00583238]

    - Estimate of measure "utilization_distr":
        0.30735
                [0.255711, 0.358989]
        0.3047
                [0.253016, 0.356384]
        0.2883
                [0.219288, 0.357312]
        0.336
                [0.277625, 0.394375]
        0.36775
                [0.30015, 0.43535]
        0.3422
                [0.283319, 0.401081]
        0.373
                [0.317583, 0.428417]
        0.3199
                [0.253941, 0.385859]
        0.4162
                [0.36628, 0.46612]
        0.3191
                [0.262867, 0.375333]
        \end{verbatim}



%%%%%%%%%%%%%%%%%%%%%%%%%%%%%%%%%%%%%%%%%%%%%%%%%%%%%%%%%%%%%%%%%
%                                                               %
%                                                               %
\section{Example B: The NRL Pump}
%                                                               %
%                                                               %
%%%%%%%%%%%%%%%%%%%%%%%%%%%%%%%%%%%%%%%%%%%%%%%%%%%%%%%%%%%%%%%%%

In order to measure the bandwidth of the covert channel of the \aemilia\ specification {\tt nrl\_pump.aem}
of Sect.~\ref{nrl_pump}, we use the following {\tt nrl\_pump.rew}:

        \begin{verbatim}
    MEASURE closed_connections_per_time_unit IS
      ENABLED(LW.send_conn_close) -> TRANS_REWARD(1);

    MEASURE aborted_connections_per_time_unit IS
      ENABLED(TLT.send_conn_exit) -> TRANS_REWARD(1)
        \end{verbatim}

\noindent The two measures above are strictly related to the connect/disconnect strategy that a malicious
high user may exploit to pass confidential information to low users. The number of connections that can be
closed or aborted per unit of time represents an estimate of how many bits can be leaked in a certain
period. We recall that the low users can deduce the presence of the high users only if some connections are
correctly terminated, as in that case the high users must have sent acknowledgments.

The following is the result of the Markovian performance evaluation at steady state:

        \begin{verbatim}
    Stationary value of the performance measures for NRL_Pump_Type:

    - Value of measure "closed_connections_per_time_unit":
        4.37617

    - Value of measure "aborted_connections_per_time_unit":
        2.27526
        \end{verbatim}

\noindent This means that a malicious high user can set up a one-bit covert channel by means of which the
high user can leak about 6~bits per second.



%%%%%%%%%%%%%%%%%%%%%%%%%%%%%%%%%%%%%%%%%%%%%%%%%%%%%%%%%%%%%%%%%
%                                                               %
%                                                               %
\section{Example C: Dining Philosophers}
%                                                               %
%                                                               %
%%%%%%%%%%%%%%%%%%%%%%%%%%%%%%%%%%%%%%%%%%%%%%%%%%%%%%%%%%%%%%%%%

In the case of the \aemilia\ specification {\tt dining\_philosophers.aem} of Sect.~\ref{dp}, it is
interesting to assess the degree of concurrency between non-adjacent philosophers, which can be expressed
through the following {\tt dining\_philosophers.rew}:

        \begin{verbatim}
    MEASURE mean_number_eating_philosophers IS
      FOR_ALL i IN 0..philosopher_num - 1
        ENABLED(P[i].eat) -> STATE_REWARD(1)
        \end{verbatim}

\noindent The degree of concurrency is obtained by counting the number of philosophers that are eating in
each state.

The following is the result of the Markovian performance evaluation conducted with the adaptive variant of
symmetric stochastic over-relaxation at steady state when there are 10~philosophers:

        \begin{verbatim}
    Stationary value of the performance measures for LR_Dining_Philosophers_Type:

    - Value of measure "mean_number_eating_philosophers":
        4.16684
        \end{verbatim}

\noindent This means that on average there are about 4~non-adjacent philosophers that are simultaneously
eating at each instant.
