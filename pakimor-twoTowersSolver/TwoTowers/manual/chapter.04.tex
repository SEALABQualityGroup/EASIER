%%%%%%%%%%%%%%%%%%%%%%%%%%%%%%%%%%%%%%%%%%%%%%%%%%%%%%%%%%%%%%%%%
%                                                               %
%                                                               %
%                                                               %
\chapter{The Equivalence Verifier}
%                                                               %
%                                                               %
%                                                               %
%%%%%%%%%%%%%%%%%%%%%%%%%%%%%%%%%%%%%%%%%%%%%%%%%%%%%%%%%%%%%%%%%


%%%%%%%%%%%%%%%%%%%%%%%%%%%%%%%%%%%%%%%%%%%%%%%%%%%%%%%%%%%%%%%%%
%                                                               %
%                                                               %
\section{Introduction}
%                                                               %
%                                                               %
%%%%%%%%%%%%%%%%%%%%%%%%%%%%%%%%%%%%%%%%%%%%%%%%%%%%%%%%%%%%%%%%%

TwoTowers~5.1 is able to check whether two correct \aemilia\ specifications -- in which all the possible
variable formal parameters and local variables are of type bounded integer, boolean, or array or record
based on the two previous basic types -- are equivalent. The verification is carried out by applying the
partition refinement algorithm by Kanellakis and Smolka~\cite{KS} and the result of the check is written to
a {\tt .evr} file, together with a distinguishing modal logic formula~\cite{Cle} -- in the case of
non-equivalence -- expressed in a verbose variant of the Hennessy-Milner logic~\cite{HM} or one of its
probabilistic extensions~\cite{LS,CGH}.

The equivalence verifier allows a comparative study of two \aemilia\ specifications to be conducted, aiming
at establishing whether they possess the same functional, security and performance properties in general.
Should the two \aemilia\ specifications be equivalent, in order to know whether they satisfy a particular
functional property, security requirement, or performance guarantee, it is necessary to apply to one of them
the model checker, the security analyzer, or the performance evaluator, respectively.



%%%%%%%%%%%%%%%%%%%%%%%%%%%%%%%%%%%%%%%%%%%%%%%%%%%%%%%%%%%%%%%%%
%                                                               %
%                                                               %
\section{Bisimulation-Based Behavioral Equivalences}
%                                                               %
%                                                               %
%%%%%%%%%%%%%%%%%%%%%%%%%%%%%%%%%%%%%%%%%%%%%%%%%%%%%%%%%%%%%%%%%

Four different bisimulation-based behavioral equivalences are available in TwoTowers~5.1: strong
bisimulation equivalence, weak bisimulation equivalence, strong (extended) Markovian bisimulation
equivalence, and weak (extended) Markovian bisimulation equivalence~\cite{Mil,BB2,Ber3}.

Strong bisimulation equivalence relates two \aemilia\ specifications if they are able to simulate each
other's functional behavior. For each pair of strongly bisimilar states of the integrated semantic models of
the two \aemilia\ specifications, it must be the case that, whenever one of the two states is able to
perform an action of a certain type, then the other state is able to perform an action of the same type,
with the two reached states being again strongly bisimilar.

Weak bisimulation equivalence is a coarser variant of the previous equivalence, in which it is possible --
to some extent -- to abstract from the execution of invisible actions, i.e.\ those actions whose type has
been hidden in the behavioral section of the \aemilia\ specifications to which equivalence checking is
applied. In essence, given a pair of weakly bisimilar states of the integrated semantic models of two
\aemilia\ specifications, it must be the case that, whenever one of the two states is able to perform an
action of a certain type, then the other state is able to perform an action of the same type possibly
preceded and followed by the execution of invisible actions, with the two reached states being again weakly
bisimilar.

Strong Markovian bisimulation equivalence is a finer variant of the first equivalence, which takes into
account timing/probabilistic aspects as well. Given a pair of strongly Markovian bisimilar states of the
integrated semantic models of two \aemilia\ specifications, it must be the case that, whenever one of the
two states is able to perform a set of actions of a certain type and priority, then the other state is able
to perform a set of actions of the same type and priority, with both states reaching the same equivalence
class of states with the same aggregated rate.

Weak Markovian bisimulation equivalence is a coarser variant of the previous equivalence, in which it is
possible -- to some extent -- to abstract from the execution of invisible immediate actions in the
continuous-time case. Basically, given a pair of weakly Markovian bisimilar states of the integrated
semantic models of two \aemilia\ specifications, it must be the case that, whenever one of the two states is
able to perform a set of actions of a certain type and priority, then the other state is able to perform a
set of actions of the same type and priority, each possibly preceded and followed by the execution of
invisible immediate actions, with both states reaching the same equivalence class of states with the same
aggregated rate.

Two \aemilia\ specifications are equivalent in accordance with one of the four equivalences above whenever
so are the initial global states of their integrated semantic models. We recall that each of the four
bisimulation-based equivalences is deadlock sensitive, i.e.\ it never equates a deadlock-free \aemilia\
specification to an \aemilia\ specification that can deadlock, and that the two Markovian ones comply with
the notion of exact aggregation for Markov chains known as ordinary lumping~\cite{Buc,Hil}.



%%%%%%%%%%%%%%%%%%%%%%%%%%%%%%%%%%%%%%%%%%%%%%%%%%%%%%%%%%%%%%%%%
%                                                               %
%                                                               %
\section{Syntax of Distinguishing Formulas}\label{disting_formula}
%                                                               %
%                                                               %
%%%%%%%%%%%%%%%%%%%%%%%%%%%%%%%%%%%%%%%%%%%%%%%%%%%%%%%%%%%%%%%%%

Whenever two \aemilia\ specifications are not found to be equivalent, a distinguishing modal logic formula
is produced, which has the following syntax:

	\begin{verbatim}
      <formula_expr> ::= "TRUE"
                      |  "FALSE"
                      |  <formula_expr> "/\" <formula_expr>
                      |  <formula_expr> "\/" <formula_expr>
                      |  "NOT" "(" <formula_expr> ")"
                      |  "EXISTS_TRANS" "(" "LABEL" "(" <action_type_label> ")" ";"
                         "REACHED_STATE_SAT" "(" <formula_expr> ")" ")"
                      |  "EXISTS_WEAK_TRANS" "(" "LABEL" "(" <action_type_label> ")" ";"
                         "REACHED_STATE_SAT" "(" <formula_expr> ")" ")"
                      |  "FOR_ALL_TRANS" "(" "LABEL" "(" <action_type_label> ")" ";"
                         "REACHED_STATE_SAT" "(" <formula_expr> ")" ")"
                      |  "FOR_ALL_WEAK_TRANS" "(" "LABEL" "(" <action_type_label> ")" ";"
                         "REACHED_STATE_SAT" "(" <formula_expr> ")" ")"
                      |  "EXISTS_TRANS_SET" "(" "LABEL" "(" <action_type_label> ")" ";"
                         <min_value_type> "(" <pos_real_number> ")" ";"
                         "REACHED_STATES_SAT" "(" <formula_expr> ")" ")"
                      |  "EXISTS_WEAK_TRANS_SET" "(" "LABEL" "(" <action_type_label> ")" ";"
                         <min_value_type> "(" <pos_real_number> ")" ";"
                         "REACHED_STATES_SAT" "(" <formula_expr> ")" ")"
                      |  "FOR_ALL_TRANS_SETS" "(" "LABEL" "(" <action_type_label> ")" ";"
                         <min_value_type> "(" <pos_real_number> ")" ";"
                         "REACHED_STATES_SAT" "(" <formula_expr> ")" ")"
                      |  "FOR_ALL_WEAK_TRANS_SETS" "(" "LABEL" "(" <action_type_label> ")" ";"
                         <min_value_type> "(" <pos_real_number> ")" ";"
                         "REACHED_STATES_SAT" "(" <formula_expr> ")" ")"
    <min_value_type> ::= "MIN_AGGR_EXP_RATE"
                      |  "MIN_AGGR_GEN_PROB"
                      |  "MIN_AGGR_REA_PROB"
        \end{verbatim}

\noindent where the satisfaction relation with respect to a given state of the integrated semantic model of
an \aemilia\ specification is defined as follows:

        \begin{itemize}

\item Constant {\tt TRUE} is satisfied in every state, while constant {\tt FALSE} is never satisfied.

\item The logical conjunction (\verb+/\+) of two properties is satisfied in a given state if so are both
properties.

\item The logical disjunction (\verb+\/+) of two properties is satisfied in a given state if so is at least
one of the two properties.

\item The logical negation ({\tt NOT}) of a property is satisfied in a given state if the property is not
satisfied in that state.

\item Quantifier {\tt EXISTS\_TRANS} expresses, with respect to a given state, the fact that from that state
it is possible to reach a state satisfying a certain property by traversing a transition that is labeled
with a certain action type.

\item Quantifier {\tt EXISTS\_WEAK\_TRANS} expresses, with respect to a given state, the fact that from that
state it is possible to reach a state satisfying a certain property by traversing a sequence of transitions,
with one of them being labeled with a certain action type and all the others being invisible.

\item Quantifier {\tt FOR\_ALL\_TRANS} expresses, with respect to a given state, the fact that whenever a
transition labeled with a certain action type can be traversed from that state, the reached state
necessarily satisfies a certain property.

\item Quantifier {\tt FOR\_ALL\_WEAK\_TRANS} expresses, with respect to a given state, the fact that
whenever a sequence of transitions, with one of them being labeled with a certain action type and all the
others being invisible, can be traversed from that state, the reached state necessarily satisfies a certain
property.

\item Quantifier {\tt EXISTS\_TRANS\_SET} expresses, with respect to a given state, the fact that from that
state it is possible to reach a set of states satisfying a certain property by traversing a set of
transitions that are labeled with a certain action type of a certain priority, whose aggregated rate is
above a certain threshold.

\item Quantifier {\tt EXISTS\_WEAK\_TRANS\_SET} expresses, with respect to a given state, the fact that from
that state it is possible to reach a set of states satisfying a certain property by traversing a set of
sequences of transitions whose aggregated rate is above a certain threshold, with every sequence having one
transition labeled with a certain action type of a certain priority and all the other transitions in the
sequence being invisible and immediate.

\item Quantifier {\tt FOR\_ALL\_TRANS\_SETS} expresses, with respect to a given state, the fact that
whenever a set of transitions, labeled with a certain action type of a certain priority and having an
aggregated rate above a certain threshold, can be traversed from that state, the reached states necessarily
satisfy a certain property.

\item Quantifier {\tt FOR\_ALL\_WEAK\_TRANS\_SETS} expresses, with respect to a given state, the fact that
whenever a set of sequences of transitions whose aggregated rate is above a certain threshold, with every
sequence having one transition labeled with a certain action type of a certain priority and all the other
transitions in the sequence being invisible and immediate, can be traversed from that state, the reached
states necessarily satisfy a certain property.

        \end{itemize}

\noindent The last four formulas specify a lower bound for the aggregated rate of the considered set of
transitions or transition sequences, which is a positive real number interpreted as an aggregated
exponential rate, a generative probability, or a reactive probability depending on whether the considered
action type refers to an exponentially timed, immediate or passive action, respectively.

When concerned with an \aemilia\ specification, a modal logic formula expressed in the syntax above holds if
it is satisfied by the initial global state of the integrated semantic model of the \aemilia\ specification.



%%%%%%%%%%%%%%%%%%%%%%%%%%%%%%%%%%%%%%%%%%%%%%%%%%%%%%%%%%%%%%%%%
%                                                               %
%                                                               %
\section{Example A: The Alternating Bit Protocol}
%                                                               %
%                                                               %
%%%%%%%%%%%%%%%%%%%%%%%%%%%%%%%%%%%%%%%%%%%%%%%%%%%%%%%%%%%%%%%%%

From an abstract point of view, the system behavior enforced by the application of the alternating bit
protocol is the same as the behavior of a one-position buffer, in which messages are alternately generated
and consumed. This can be formalized through the following {\tt abp\_spec.aem}:

	\begin{verbatim}
    ARCHI_TYPE ABP_Spec_Type(const rate msg_gen_rate := 5)

    ARCHI_ELEM_TYPES

      ELEM_TYPE One_Pos_Buffer_Type(const rate msg_gen_rate)

        BEHAVIOR

          One_Pos_Buffer(void;
                         void) =
            <generate_msg, exp(msg_gen_rate)> . <consume_msg, inf> . One_Pos_Buffer()

        INPUT_INTERACTIONS

          UNI generate_msg

        OUTPUT_INTERACTIONS

          UNI consume_msg

    ARCHI_TOPOLOGY

      ARCHI_ELEM_INSTANCES

        OPB : One_Pos_Buffer_Type(msg_gen_rate)

      ARCHI_INTERACTIONS

        OPB.generate_msg;
        OPB.consume_msg

      ARCHI_ATTACHMENTS

        void

    END
	\end{verbatim}

Now the question is whether the \aemilia\ specification {\tt abp.aem} of Sect.~\ref{abp} is equivalent to
the \aemilia\ specification {\tt abp\_spec.aem}, which would imply the correctness of {\tt abp.aem}. Before
applying equivalence verification, we need to modify both specifications to get rid of the dot notation for
the observable action types, and in {\tt abp.aem} we have to hide all the action types not related to
message generation or consumption, as these are the only ones occurring in {\tt abp\_spec.aem} -- hence the
only ones that can be matched in the bisimulation setting. Therefore, we rewrite {\tt abp.aem} into
{\tt abp\_impl.aem} by defining a new architectural type called {\tt ABP\_Impl\_Type} with the same
parameters, AETs, and architectural topology as before, and in addition the following behavioral variations:

	\begin{verbatim}
    BEHAV_VARIATIONS

      BEHAV_HIDINGS

        HIDE LM.ALL;
        HIDE LA.ALL;
        HIDE S.timeout

      BEHAV_RENAMINGS

        RENAME S.generate_msg AS generate_msg;
        RENAME R.consume_msg  AS consume_msg
	\end{verbatim}

\noindent Then we add the following behavioral variations to {\tt abp\_spec.aem}:

	\begin{verbatim}
    BEHAV_VARIATIONS

      BEHAV_RENAMINGS

        RENAME OPB.generate_msg AS generate_msg;
        RENAME OPB.consume_msg  AS consume_msg
	\end{verbatim}

The result of the strong bisimulation equivalence verification is the following:

	\begin{verbatim}
    ABP_Impl_Type isn't strongly bisimulation equivalent to ABP_Spec_Type
    as demonstrated by the following modal logic formula
    satisfied by ABP_Impl_Type but not by ABP_Spec_Type:

      NOT(EXISTS_TRANS(
            LABEL(generate_msg); 
            REACHED_STATE_SAT(
              EXISTS_TRANS(
                LABEL(consume_msg); 
                REACHED_STATE_SAT(TRUE)
              )
            )
          )
      )
	\end{verbatim}

\noindent which means that the second specification can consume a message right after its generation, while
this is not possible in the case of the first specification because of its actions explicitly modeling
message transmission, propagation, loss, and delivery. In other words, we cannot expect the two \aemilia\
specifications to be strongly bisimulation equivalent, as the invisible actions -- modeling details related
to the message flow -- of {\tt abp\_impl.aem} cannot be matched by any of the actions of
{\tt abp\_spec.aem}.

The result of the weak bisimulation equivalence verification is the following:

	\begin{verbatim}
    ABP_Impl_Type is weakly bisimulation equivalent to ABP_Spec_Type.
	\end{verbatim}

\noindent This ensures that the two specifications have the same functional behavior up to invisible
actions. So, it is now worth investigating whether the two specifications guarantees the same performance.

The following are the results of the strong and weak Markovian bisimulation equivalence verifications:

	\begin{verbatim}
    ABP_Impl_Type isn't strongly Markovian bisimulation equivalent to ABP_Spec_Type
    as demonstrated by the following modal logic formula
    satisfied by ABP_Impl_Type but not by ABP_Spec_Type:

      NOT(EXISTS_TRANS_SET(
            LABEL(generate_msg); 
            MIN_AGGR_EXP_RATE(5.000000); 
            REACHED_STATES_SAT(
              EXISTS_TRANS_SET(
                LABEL(consume_msg); 
                MIN_AGGR_GEN_PROB(1.000000); 
                REACHED_STATES_SAT(TRUE)
              )
            )
          )
      )

    ABP_Impl_Type isn't weakly Markovian bisimulation equivalent to ABP_Spec_Type
    as demonstrated by the following modal logic formula
    satisfied by ABP_Impl_Type but not by ABP_Spec_Type:

      NOT(EXISTS_WEAK_TRANS_SET(
            LABEL(generate_msg); 
            MIN_AGGR_EXP_RATE(5.000000); 
            REACHED_STATES_SAT(
              EXISTS_WEAK_TRANS_SET(
                LABEL(consume_msg); 
                MIN_AGGR_GEN_PROB(1.000000); 
                REACHED_STATES_SAT(TRUE)
              )
            )
          )
      )
	\end{verbatim}

\noindent As we should have expected, the two \aemilia\ specifications are not equivalent from the
performance viewpoint. The reason is that {\tt abp\_spec.aem} completely abstracts from message losses as
well as propagation delays.

The same results are obtained when considering the value passing \aemilia\ specification {\tt abp\_vp.aem}
of Sect.~\ref{abp_vp} -- enriched with suitable behavioral variations -- in place of {\tt abp\_impl.aem}.
